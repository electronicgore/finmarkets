%%% License: Creative Commons Attribution Share Alike 4.0 (see https://creativecommons.org/licenses/by-sa/4.0/)
%%% Slides are based heavily on earlier versions of this course taught by Jesper Rudiger.

\documentclass[english,10pt
%,handout
,aspectratio=169
]{beamer}
%%% License: Creative Commons Attribution Share Alike 4.0 (see https://creativecommons.org/licenses/by-sa/4.0/)
%%% Slides are based heavily on earlier versions of this course taught by Jesper Rudiger and Peter Norman Sorensen.

\DeclareGraphicsExtensions{.eps, .pdf,.png,.jpg,.mps,}
\usetheme{reMedian}
\usepackage{parskip}
\makeatother

\renewcommand{\baselinestretch}{1.1} 

\usepackage{amsmath, amssymb, amsfonts, amsthm}
\usepackage{enumerate}
\usepackage{hyperref}
\usepackage{url}
\usepackage{bbm}
\usepackage{color}

\usepackage{tikz}
\usepackage{tikzscale}
\newcommand*\circled[1]{\tikz[baseline=(char.base)]{
		\node[shape=circle,draw, inner sep=-20pt] (char) {#1};}}
\usetikzlibrary{automata,positioning}
\usetikzlibrary{decorations.pathreplacing}
\usepackage{pgfplots}
\usepgfplotslibrary{fillbetween}
\usepackage{graphicx}

\usepackage{setspace}
%\thinmuskip=1mu
%\medmuskip=1mu 
%\thickmuskip=1mu 


\usecolortheme{default}
\usepackage{verbatim}
\usepackage[normalem]{ulem}

\usepackage{apptools}
\AtAppendix{
	\setbeamertemplate{frame numbering}[none]
}
\usepackage{natbib}



\title{Financial Markets Microstructure \\ Lecture 19}

\subtitle{Markets and public information}

\author{Egor Starkov}

\date{K{\o}benhavns Unversitet \\
	Spring 2021}


\begin{document}
	\AtBeginSection[]{
		\frame<beamer>{
			\frametitle{This lecture:}
			\tableofcontents[currentsection,currentsubsection]
	}}
	\frame[plain]{\titlepage}


\begin{frame}{Previously on FMM}
	\begin{itemize}
		\item \textbf{High-Frequency Trading} generates informational asymmetries between traders
		\item If the markets are already reasonably good at matching traders with opportunities, fast trading may be strictly bad for welfare
		\item This inefficiency may be a question of market design
		\item Use better market design (frequent auctions) to improve this
	\end{itemize}
\end{frame}


\begin{frame}{News}
	\begin{itemize}
		\item Incomplete information has been pivotal in many of the models we have considered
		\item Roughly speaking, two types of models
		\begin{itemize}
			\item \structure<1>{Symmetric uncertainty}: e.g. Parlour, DGP
			\item \alert<1>{Asymmetric uncertainty}: e.g. Glosten-Milgrom
		\end{itemize}
		%\item How does uncertainty affect prices/trade?
		%\item Who would gain from reducing uncertainty in different models?
		\pause
		\item When \structure<2>{private news} arrive to insiders, they \structure<2>{trade}, everyone else learns from trades
		\item When \structure<3>{public} news arrive: \only<2>{...?}
		\pause
		\begin{itemize}
			\item (in the presence of HFTs, public news $\approx$ HFTs' private news)
			\item all orders are repriced, but \structure<3>{no trade} \alert<4>{should} take place
			\pause
			\item in reality, a lot of trade after public announcements
		\end{itemize}
	\end{itemize}
\end{frame}


\begin{frame}{Trade after public news}
	\begin{itemize}
		\item So why do public announcements generate high trading volumes?
		\item \textbf{Potential explanations} include:
		\begin{itemize}
			\item Announcements are made up of public \structure{and private} signals
			\item Or traders have \structure{heterogeneous priors} and can therefore `agree to disagree'; announcements then can amplify or mitigate these initial disagreements and so generate trade.
		\end{itemize}
		\item \textbf{\cite{kondor_more_2012}} provides a fresh look at this: focus on \structure{higher-order beliefs} 
	\end{itemize}
\end{frame}


\begin{frame}{Higher-order beliefs}
	Suppose two players are interested in $\theta$, which is unobserved. However, each observe a signal $x_{i}$. Consider player 1's beliefs
	\begin{itemize}
		\item First-order beliefs: $\mathbb{E}(\theta|x_{1})$
		\item Second-order beliefs: $\mathbb{E}(\mathbb{E}(\theta|x_{2})|x_{1})$
		\item Third-order beliefs: $\mathbb{E}(\mathbb{E}(\mathbb{E}(\theta|x_{1})|x_{2})|x_{1})$
		\item Etc.
	\end{itemize}
	Higher-order beliefs are beliefs about the beliefs of others.
	
	\pause\bigskip
	In most models so far, traders make decisions based on expectation of fundamental value: \\ no resale $\rightarrow$ higher-order beliefs are irrelevant.
	
	Even with resale, they are often irrelevant: $\mathbb{E}(\mathbb{E}(\theta|x_{2})|x_{1}) = \mathbb{E}(\theta|x_{1})$.
\end{frame}


%\begin{frame}{Public Information in a `standard model'}
%	Let's think about public information in a Glosten-Milgrom model
%	
%	\begin{itemize}
%		\item \textbf{Value}: $\theta=\theta_1 + \theta_2$, where $\theta_1, \theta_2 \in \{-1,0,1\}$ are distributed independently with equal probability
%		\item \textbf{Traders}:
%		\begin{itemize}
%			\item With prob. $\pi$: risk-neutral \structure{speculator} who observes $\theta_2$
%			\item With prob. $1-\pi$: \structure{noise trader}, buy/sell with equal probability
%		\end{itemize}
%		\item \textbf{Public signal}: $ y = \theta_1$
%		\only<2-3|handout:0>{
%			\item \alert{Bliz Quiz}: how does the public signal affect outcomes?
%			\begin{itemize}
%				\item nothing changes
%				\item \structure<3>{midprice changes, spread ($\approx$ volume) doesn't}
%				\item spread changes, midprice doesn't
%				\item both midprice and spread change
%			\end{itemize}
%		}
%	\end{itemize}
%\end{frame}
%
%
%\begin{frame}{Public Information in a `standard model' (2)}
%	\begin{itemize}
%		\item \textbf{Equilibrium ask price}: 
%		\[
%		%a=\theta_1+\frac{\frac{1}{3}(\pi+(1-\pi)\frac{1}{2})}{\frac{1}{3}(\pi+(1-\pi)\frac{1}{2})+\frac{2}{3}(1-\pi)\frac{1}{2}}=\theta_1+\frac{1+\pi}{1+\pi+2(1-\pi)}.
%		a = \mathbb{E}[v|\theta_1,\text{buy}] = \theta_1 + \frac{ \pi/3 }{ \pi/3 + (1-\pi)/2}
%		\]
%		\item \textbf{Role of public information}: Whereas prices react to public information, trading probability is independent
%		\item \textbf{Beliefs}: Higher-order beliefs are of no importance in this model: market makers and speculators care only about $\mathbb{E}[\theta]$
%		\item \textbf{Why?} \pause This is a model without resale, so you only care about the fundamental value, not about a potential resale price
%	\end{itemize}
%\end{frame}
%
%
%\begin{frame}{Kondor's contribution}
%	\begin{itemize}
%		\item \textbf{Idea} 
%		\begin{itemize}
%			\item Short-term traders with different trading times often interact
%			\item Eg. currency traders often trade during the opening hours of their local exchange and then `close their positions' at the end of the day
%			\item Example: London traders offload their position to NY traders 
%			\item Thus, the London traders (only) care about the re-sale value, which depends on the NY traders' valuations: \structure{second-order beliefs} matter
%		\end{itemize}
%		\item \textbf{Public announcements}
%		\begin{itemize}
%			\item Cause traders to agree more on the fundamental value of assets
%			\item But may cause them to disagree about the valuations of other traders
%		\end{itemize}
%		\item \textbf{Disagreement}: Diverging second-order beliefs lead traders to trade
%	\end{itemize}
%\end{frame}


\begin{frame}{\cite{kondor_more_2012}: Example}
	\begin{itemize}
		\item Two groups of traders, $I$ and $J$
		\item Fundamental value has two components: $\theta=\theta_{I}+\theta_{J}$
		\item I-trader signal: $x^{i} = \theta_{I}+\epsilon^{i}$
		\item J-trader signal: $z^{j} = \theta_{J}+\epsilon^{j}$
		\item Public signal: $y = \theta + \eta$
		\item Suppose $\theta_{I},\theta_{J}, \epsilon^{i}, \epsilon^{j}, \eta$ are independent and normal with zero mean
	\end{itemize}
	Feature that generates disagreement: private signals about \\different aspects of the fundamentals, $\theta_I$ and $\theta_J$.
\end{frame}


\begin{frame}{Example: Before public announcement}
	\textbf{No public news} ($y$ not observed)
	\begin{itemize}
		\item Traders' beliefs about $\theta$ are
		\begin{align*}
			\mathbb{E}(\theta|z^{j}) & =\mathbb{E}(\theta_{J}|z^{j})+\mathbb{E}(\theta_{I}|z^{j})=a_J z^{j}+0, a_J>0 \\
			\mathbb{E}(\theta|x^{i}) &=\mathbb{E}(\theta_{J}|x^{i})+\mathbb{E}(\theta_{I}|x^{i})=a_I x^{i}+0, a_I>0
		\end{align*}
		\item $I$-trader's second-order belief is
		\[
		\mathbb{E}(\mathbb{E}(\theta|z^{j})|x^{i})=\mathbb{E}(\theta) = 0,
		\]
		where the first equality holds because $x^{i}$ and $z^{j}$ are independent
	\end{itemize}
\end{frame}


\begin{frame}{Example: With public announcement}
	\textbf{Given public signal $y$}
	\begin{itemize}
		\item Traders' beliefs about $\theta$ are
		\begin{align*}
			\mathbb{E}(\theta|z^{j}, y) & =\mathbb{E}(\theta_{J}|z^{j},y)+\mathbb{E}(\theta_{I}|z^{j},y)=(b_J z^{j}+c_Jy)+d_J y,  \\
			\mathbb{E}(\theta|x^{i}, y) & =\mathbb{E}(\theta_{J}|x^{i},y)+\mathbb{E}(\theta_{I}|x^{i},y)=(b_I x^{i}+c_I y)+d_I y,
		\end{align*}
		where $b_k,c_k,d_k>0$ and $b_k<a_k$. First-order beliefs \structure{converge}
		\item But $I$-agent's second-order belief  is
		\begin{align*}
			\mathbb{E}(\mathbb{E}(\theta|z^{j},y)|x^{i},y)
			&=b_J\mathbb{E}(z^{j}|x^{i},y)+(c_J+d_J)y, \\
			&=b_J(ey-fx^{i})+(c_J+d_J)y 
		\end{align*}
		where $e,f>0$. 
		\alert{Decreasing in $x_i$} -- second-order beliefs \structure{diverge}
	\end{itemize}
\end{frame}


\begin{frame}{Example: Conclusion}
	\begin{itemize}
		\item Claim: divergence leads to trade (will show later)
		\item Further, the divergence grows the more precise public info is
		\begin{itemize}
			\item Recall: no divergence in the absence of a public signal
		\end{itemize}
	\end{itemize}
\end{frame}


\begin{frame}{\cite{kondor_more_2012}: Full(er) Model}
	\begin{itemize}
		\item \textbf{Timing}: 
		\begin{enumerate}
			\item I-traders observe their information and trade
			\item I-traders liquidate all their positions and sell to J-traders
			\item $\theta$ (distributed as before) is realized and J-traders consume asset
		\end{enumerate}
		\item \textbf{Traders}: Price takers, $i, j\sim u(0,1)$, demand $d_i$, util $u(W_i)=-e^{-\gamma W_i}$ and   
		\begin{align*}
			W_I & = {d_I}(p_2-p_1); &
			W_J & = {d_J}(\theta-p_2).
		\end{align*}
		\item \textbf{Supply}: Time-$t$ asset supply  $u_t$: \\$u_1 \sim \mathcal{N}(0,1/\delta^2_1)$, $u_2\sim \mathcal{N}(0,1/\delta^2_2)$ ($u_2 \equiv u_1 + \varDelta u_2$)
	\end{itemize}
\end{frame}


\begin{frame}{Analysis: Trader maximization}
	\begin{itemize}
		\item (NOTE: some parameter names are different to paper)
		\item \textbf{Random supply}: Implies that prices are not completely informative.
		\item \textbf{I traders}: Solve
		\[
			\max_{d_I} \mathbb{E}\left[ -e^{-\gamma W_I}| x^i, y, p_1 \right]
		\]
		\begin{itemize}
			\item CARA utility and normal distributions $\Rightarrow$ can rewrite $I$ traders' problem as
		\end{itemize}
		\[
			\max_{d_I} \left\{\mathbb{E}\left[W_I| x^i, y, p_1\right]-\frac{\gamma}{2} \mathbb{V}\left[W_I| x^i, y, p_1\right]\right\}
		\]
		\item \textbf{J traders}: Solve
		\[
			\max_{d_J} \left\{\mathbb{E}\left[W_J| z^j, y, p_1, p_2\right]-\frac{\gamma}{2} \mathbb{V}\left[W_J| z^j, y, p_1, p_2\right] \right\}
		\]
	\end{itemize}
\end{frame}


\begin{frame}{Analysis: Trader maximization (2)}
	\begin{itemize}
		\item Taking the FOC and solving for the demands we get 
		\begin{align}
			d^*_{1,i} &= \frac{\tau^2_{p_2}}{\gamma}(\mathbb{E}[p_2| x^i, y, p_1]-p_1), \label{eqfoc1} \\
			d^*_{2,j} &= \frac{\tau^2_\theta}{\gamma} (\mathbb{E}[\theta| z^j, y, p_1, p_2]-p_2), \label{eqfoc2}
		\end{align}
		where $\tau^2_{p_2}=1/\mathbb{V}(p_2| x^i, y, p_1)$ and $\tau^2_{\theta}=\mathbb{V}(\theta| z^j, y, p_1, p_2)$
		\item In order to calculate expectations, need to make a conjecture about prices
	\end{itemize}
\end{frame}


\begin{frame}{Analysis: Linear prices and price signals}
	\begin{itemize}
		\item \textbf{Equilibrium}: Find equilibrium with linear price function/demand
		\begin{align}
			p_1 & = \frac{1}{e_1}\left[a_1 \theta_I + c_1 y - u_1 \right]\label{price1}\\
			p_2 & =\frac{1}{e_2}\left[ b_2 \theta_J + c_2 y +  g_2 q_1-u_2 \right], \label{price2}
		\end{align}
		for some $a_1,b_2,c_1,c_2,e_1,e_2,g_2$, 
		where $q_1$ is the \structure{price signal} of $\theta_I$
		\item \textbf{Price signal}: This tells us the information contained in prices:
		\begin{align}
			q_1 & = \mathbb{E}[\theta_I|p_1,y]= \frac{e_1 p_1-c_1 y}{a_1} = \theta_I - \frac{1}{a_1} u_1 ; \label{signal1} \\
			q_2 &=\mathbb{E}[\theta_J|p_1,p_2,y]= \frac{e_2 p_2-c_2 y - g_2 q_1}{b_2} = \theta_J - \frac{1}{b_2} u_2. \label{signal2}
		\end{align}
	\end{itemize}
\end{frame}


\begin{frame}{Analysis: Reformulating in terms of price signals}
	\begin{itemize}
		\item \textbf{Rewrite expectation}. All variables jointly normal $\rightarrow$ linear expressions
		\begin{align}
		\mathbb{E}[p_2| x^i, y, p_1] &=\mathbb{E}[p_2| x^i, y, q_1] = a^e_1 x^i+b^e_1 y +c^e_1 q_1;   \label{exp1} \\
		\mathbb{E}[\theta| z^j, y, p_1, p_2] &= \mathbb{E}[\theta| z^j, y, q_1, q_2] =  a^e_2 z^j+b^e_2 y +c^e_2 q_1 +d^e_2 q_2. \label{exp2}
		\end{align}
		\item \textbf{Rewrite FOC}. Plugging \eqref{exp1} and \eqref{exp2} into \eqref{eqfoc1} and \eqref{eqfoc2} we get
		\begin{align}
		d^*_{1,i} &= \frac{\tau^2_{p_2}}{\gamma}(a^e_1 x^i+b^e_1 y +c^e_1 q_1 - p_1); \label{eqfocb1} \\
		d^*_{2,j} &= \frac{\tau^2_\theta}{\gamma}(a^e_2 z^j+b^e_2 y +c^e_2 q_1 +d^e_2 q_2 - p_2). \label{eqfocb2}
		\end{align}
		\item \textbf{Market clearing}: $u_1=\int_0^1 d^*_{1,i} di$ and $u_2=\int_0^1 d^*_{2,j} dj$
	\end{itemize}
\end{frame}


\begin{frame}{Analysis: Equilibrium}
	\begin{itemize}
		\item \textbf{Matching coefficients}: From market clearing, can show that $p_t$ is linear function as conjectured
		\item \textbf{Equilibrium demand}: Matching up all the coefficients, we can then show that
		\begin{align}
		d^*_{1,i} &=  a_1 x^i+c_1 y -e_1 p_1; \label{eqdem1} \\
		d^*_{2,j} &= b_2 z^j+c_2 y + g_2q_1-e_2 p_2. \label{eqdem2}
		\end{align} 
		\item Demand is increasing in private signal ($x^i$/$z^j$), in public signal ($y$), in price signal ($q_1$), and decreasing in price ($p_t$) 
		(recall that traders are price takers)
	\end{itemize}
\end{frame}


\begin{frame}{Results: Demand period 2}
	\begin{itemize}
		\item Let's look at what drives the agents' demands. Rewrite period-2 demand as 
		\begin{block}{}
			\begin{equation} \label{eqdem2z}
			d^*_{2,j} = b_2 (z^j - q_2) 
			\end{equation}
		\end{block}
		\item Notice that $z^j$ is $j$'s private signal and  $q_2$ is a noisy signal of all the other agents' signals
		\item Thus, if $j$ believes to have received a better signal than everybody else, he will buy, otherwise sell
		\item This is a standard story: $J$-traders trade due to a \textbf{difference in opinion} -- they think the asset is worth more/less than others
	\end{itemize}
\end{frame}


\begin{frame}{Results: Demand period 1}
	\begin{itemize}
		\item Market clearing in period 2 together with \eqref{eqfoc2} implies
		\[
		p_2 = \int_0^1 \mathbb{E}[\theta|z^j, y , q_1, q_2] dj - \frac{\gamma}{\tau^2_\theta}u_2
		\]
		\item Rewrite period-1 demand using this:
		\[
			d^*_{1,i} = \frac{\tau^2_{p_2}}{\gamma} \left( \underbrace{\mathbb{E}\left[ \int_0^1 \mathbb{E}[\theta \mid z^j, y , q_1, q_2 ] dj -\frac{\gamma}{\tau^2_\theta}u_2 \mid x^i,y,q_1 \right]}_{\text{2nd order expectation}}-p_1 \right)
		\]
		\item $I$-trader demand in period 1 is thus a function of a \textbf{second-order expectation}: 
		The more $i$ expects $J$ traders to value the asset, the more he buys 
	\end{itemize}
\end{frame}


\begin{frame}{Results}
	\begin{itemize}
		\item In the paper, Kondor considers a more general information structure where there is a common factor about which $I$ and $J$ both learn.
		\item He then defines \textit{weakly correlated} information structures in which the common factor  is not too important
		\item In the above, we have disregarded the common factor, so what we analyzed is automatically a weakly correlated information structure
		\begin{block}{Main result}
			If the information structure is \textit{weakly correlated}, then trading intensity, volume and informational content of prices \structure{increase} in both periods when there is more public info.
		\end{block}
		\item Public signals  create trade, due to their effect on second-order beliefs
	\end{itemize}
\end{frame}


\begin{frame}{Model 2}
	\textbf{Heterogenous trading horizons}
	\begin{itemize}
		\item Timing
		\begin{enumerate}
			\item I-traders and J-traders trade
			\item I-traders sell all their holdings to J-traders
			\item $\theta$ is realized and J-traders consume
		\end{enumerate}
		\item Let $\mu$ be the proportion of J-traders
	\end{itemize}
	\textbf{Interpretation}
	\begin{itemize}
		\item Traders with different trading horizons co-exist in the market
		\item For instance day-traders and pension savers
	\end{itemize}
\end{frame}


\begin{frame}{Model 2: Results}
	\begin{itemize}
		\item When $\mu$ is high, most traders trade with each other: the market is well-integrated
		%\item Public information will make J-traders trade more
		%\begin{itemize}
		%	\item When prices become more correlated with fundamentals, J-traders can better forecast them
		%\end{itemize}
		\item When $\mu$ is small, the results of model 1 are close to those of model 2
		\item But when $\mu$ is high, public information crowds out private information, and public signals have the usual effect
		\begin{itemize}
			\item I.e., there will be less disagreement and less trade
		\end{itemize}
		\item Thus, integration is key to the results (what happens to ST speculation as market becomes more integrated?)
	\end{itemize}
\end{frame}


\begin{frame}{Relation to empirics}
	\begin{itemize}
		\item In general, the model provides an explanation for trade after public announcements
		\item \citet*{bailey_economic_2006} find that price volatility and trading volumes increase after earnings announcements
		\begin{itemize}
			\item They find that the effect is larger for cross-listed stock
		\end{itemize}
		\item Kondor argues that cross-listing is roughly equivalent to lower market integration: lower $\mu$
	\end{itemize}
\end{frame}


\begin{frame}{Kondor: Conclusion}
	\begin{itemize}
		\item The introduction of public announcements that affect second-order beliefs is new in this setting
		\item The results are quite surprising, and go against many standard intuitions
		\item In particular, public announcements may increase trading volumes and price volatility
		\item Rather than being a welfare analysis, this goes toward explaining some empirical puzzles 
		\item In particular, it should allow us to predict better which stocks will react strongly to announcements
	\end{itemize}
\end{frame}

\appendix
\begin{frame}[allowframebreaks]{References}
	\bibliography{../teaching}
	\bibliographystyle{abbrvnat}
\end{frame}


\end{document} 