%%% License: Creative Commons Attribution Share Alike 4.0 (see https://creativecommons.org/licenses/by-sa/4.0/)
%%% Slides are based heavily on earlier versions of this course taught by Jesper Rudiger.

\documentclass[english,10pt
%,handout
,aspectratio=169
]{beamer}
%%% License: Creative Commons Attribution Share Alike 4.0 (see https://creativecommons.org/licenses/by-sa/4.0/)
%%% Slides are based heavily on earlier versions of this course taught by Jesper Rudiger and Peter Norman Sorensen.

\DeclareGraphicsExtensions{.eps, .pdf,.png,.jpg,.mps,}
\usetheme{reMedian}
\usepackage{parskip}
\makeatother

\renewcommand{\baselinestretch}{1.1} 

\usepackage{amsmath, amssymb, amsfonts, amsthm}
\usepackage{enumerate}
\usepackage{hyperref}
\usepackage{url}
\usepackage{bbm}
\usepackage{color}

\usepackage{tikz}
\usepackage{tikzscale}
\newcommand*\circled[1]{\tikz[baseline=(char.base)]{
		\node[shape=circle,draw, inner sep=-20pt] (char) {#1};}}
\usetikzlibrary{automata,positioning}
\usetikzlibrary{decorations.pathreplacing}
\usepackage{pgfplots}
\usepgfplotslibrary{fillbetween}
\usepackage{graphicx}

\usepackage{setspace}
%\thinmuskip=1mu
%\medmuskip=1mu 
%\thickmuskip=1mu 


\usecolortheme{default}
\usepackage{verbatim}
\usepackage[normalem]{ulem}

\usepackage{apptools}
\AtAppendix{
	\setbeamertemplate{frame numbering}[none]
}
\usepackage{natbib}



\title{Financial Markets Microstructure \\ Lecture 19}

\subtitle{Markets and public information}

\author{Egor Starkov}

\date{K{\o}benhavns Unversitet \\
	Spring 2022}


\begin{document}
	\AtBeginSection[]{
		\frame<beamer>{
			\frametitle{This lecture:}
			\tableofcontents[currentsection,currentsubsection]
	}}
	\frame[plain]{\titlepage}


\begin{frame}{Previously on FMM}
	\begin{itemize}
		\item \textbf{High-Frequency Trading} generates informational asymmetries between traders
		\item If the markets are already reasonably good at matching traders with opportunities, fast trading may be strictly bad for welfare
		\item While HFTs can provide liquidity, more HFTs does not necessarily improve liquidity
		\item So it might be optimal to eliminate the speed game, e.g. by moving away from continuous markets to frequent batch auctions
	\end{itemize}
\end{frame}


\begin{frame}{News}
	\begin{itemize}
		\item Incomplete information has been pivotal in many of the models we have considered
		\item But we really focused on \structure<1>{asymmetric information}
		\begin{itemize}
			\item When \structure<1>{private news} arrive to insiders, they \structure<1>{trade}, everyone else learns from trades
		\end{itemize}
		\pause
		\item What about \structure<2>{symmetric uncertainty}? When \structure<2>{public} news arrive: \only<2>{...?}
		\pause
		\begin{itemize}
			\item all orders are repriced, but \structure<3>{no trade} \alert<4>{should} take place
			\pause
			\item in reality, a lot of trade after public announcements
		\end{itemize}
		\item Let's look at how information is aggregated and how this depends on public info!
	\end{itemize}
\end{frame}


%\begin{frame}{Trade after public news}
%	\begin{itemize}
%		\item So why do public announcements generate high trading volumes?
%		\item \textbf{Potential explanations} include:
%		\begin{itemize}
%			\item Announcements are made up of public \structure{and private} signals
%			\item Or traders have \structure{heterogeneous priors} and can therefore `agree to disagree'; announcements then can amplify or mitigate these initial disagreements and so generate trade.
%		\end{itemize}
%		\item \textbf{\cite{kondor_more_2012}} provides a fresh look at this: focus on \structure{higher-order beliefs} 
%	\end{itemize}
%\end{frame}
%
%
%\begin{frame}{Higher-order beliefs}
%	Suppose two players are interested in $\theta$, which is unobserved. However, each observe a signal $x_{i}$. Consider player 1's beliefs
%	\begin{itemize}
%		\item First-order beliefs: $\mathbb{E}(\theta|x_{1})$
%		\item Second-order beliefs: $\mathbb{E}(\mathbb{E}(\theta|x_{2})|x_{1})$
%		\item Third-order beliefs: $\mathbb{E}(\mathbb{E}(\mathbb{E}(\theta|x_{1})|x_{2})|x_{1})$
%		\item Etc.
%	\end{itemize}
%	Higher-order beliefs are beliefs about the beliefs of others.
%	
%	\pause\bigskip
%	In most models so far, traders make decisions based on expectation of fundamental value: \\ no resale $\rightarrow$ higher-order beliefs are irrelevant.
%	
%	Even with resale, they are often irrelevant: $\mathbb{E}(\mathbb{E}(\theta|x_{2})|x_{1}) = \mathbb{E}(\theta|x_{1})$.
%\end{frame}



\section{Context 1: \cite{hellwig_aggregation_1980}}

\begin{frame}{Context: \cite{hellwig_aggregation_1980}}
	\begin{itemize}
		\item Suppose there is a continuum of traders $i \in [0,1]$ with CARA preferences:
		\begin{align*}
			U(W_i)&=-e^{-\gamma W_i}, &
			W_i & = (\theta - p){d_i}
		\end{align*}
		\item One asset, fundamental value $\theta \sim \mathcal{N}(0,1/\tau_\theta)$
		\item Every trader gets a private signal $x_i = \theta + \epsilon_i$ with $\epsilon_i \sim \mathcal{N}(0,1/\tau_\epsilon)$ and submits a price-contingent demand schedule $d_i(p)$
		\item Aggregate supply $u_1 \sim \mathcal{N}(0, \sigma^2_u)$ (provided by noise traders)
		\item Price $p$ is set to clear the market
		% kinda Kyle, but many informed traders, non-strategic dealer
	\end{itemize}
\end{frame}


\begin{frame}{Hellwig: eqm}
	\begin{itemize}
		\item Conjecture a linear eqm, where price is a linear function of $\theta$ and $u$ (and aggregate $\bar{x} = \int x_i di$, which cancels out in the end)
		\item $\theta | p,d_i$ is normal with some mean and variance $\Rightarrow$ so is $W_i$
		(without noise trades, $\theta|p$ is degenerate)
		
		\item Then trader $i$'s problem is:
		\begin{align*}
			&\max_{d_i} \left\{ \mathbb{E} \left[ -e^{-\gamma W_i} | x_i, p \right] \right\} 
			\\
			\text{($W_i$ normal)} \iff 
			&\max_{d_i} \left\{\mathbb{E} \left[ W_i| x_i, p\right] - \frac{\gamma}{2} \mathbb{V}\left[W_i| x_i, p \right] \right\}
			\\
			\Rightarrow & d_i(x_i,p) = \frac{\mathbb{E} \left[ \theta | x_i, p\right] - p}{\gamma \mathbb{V} \left[ \theta| x_i, p\right]}, \quad p= \mathbb{E} \Big[ \mathbb{E} [\theta|x_i,p] \mid p \Big] - \gamma \mathbb{V} \left[ \theta| x_i, p\right] u
		\end{align*}
	\end{itemize}
\end{frame}


\begin{frame}{Hellwig: conclusion}
	\begin{itemize}
		\item $\mathbb{E} \left[ \theta | x_i, p\right] = a x_i + b p$ for $a,b>0$
		\item Traders \structure{respond to private signals}: high $x_i$ $\Rightarrow$ buy, low $x_i$ $\Rightarrow$ sell
		\item This is because $i$ believes that when $x_i > p$, this might be because high supply $u$ depressed price, hence a purchase is justified (traders \structure{disagree about the fundamental} value $\theta$)
		\pause
		\item What if the traders care not about $\theta$, but the resale value instead?
	\end{itemize}
\end{frame}



\section{Context 2: \cite{brown_technical_1989}}

\begin{frame}{\cite{brown_technical_1989}}
	\begin{itemize}
		\item Now consider the same economy but in dynamics: there are two periods $t=1,2$, two populations of traders $i,j \in [0,1]$ with CARA preferences, and early traders need to offload their asset holdings to late traders:
		\begin{align*}
			W_i & = (p_2 - p_1) d_i, &
			W_j & = (\theta - p_2) d_j
		\end{align*}
		\item Private signals $x_i = \theta + \epsilon_i$, $z_j = \theta + \epsilon_j$
		\item Asset aggregate supply $u_1,u_2$ i.i.d. normal
	\end{itemize}
\end{frame}


\begin{frame}{Brown-Jennings: eqm}
	\begin{itemize}
		\item \alert<1>{In $t=2$}, same as Hellwig: 
		\begin{align*}
			d_j(x_j,p_2,p_1) &= \frac{\mathbb{E} \left[ \theta | z_j, p_2, p_1\right] - p_2}{\gamma \mathbb{V} \left[ \theta| z_j, p_2, p_1\right]},
			& 
			p_2 &= \mathbb{E} \Big[ \mathbb{E} [\theta|z_j, p_2, p_1] \mid p_2, p_1 \Big] - \gamma \mathbb{V} \left[ \theta| z_j, p_2, p_1\right] u_2
		\end{align*}
		\pause
		\item \alert{In $t=1$}, instead of $(\theta|x_i, p_1)$, traders now care about $(p_2|x_i, p_1)$:
		\begin{align*}
			d_i(x_i,p_1) &= \frac{\mathbb{E} \left[ p_2|x_i, p_1\right] - p_1}{\gamma \mathbb{V} \left[ \theta| p_2|x_i, p_1\right]}
		\end{align*}
		i.e., their \alert{demand depends on} $\mathbb{E} \left[ \mathbb{E} [\theta|z_j, p_2, p_1] | x_i, p_1\right]$ -- their expectation of later investors' expectation of $\theta$.
	\end{itemize}
\end{frame}


\begin{frame}{Brown-Jennings: conclusions}
	\begin{itemize}
		\item \structure{Second-order beliefs} are important for trading volumes in dynamic settings.
		\item In this specific model, early traders have no reason to believe that late traders' estimate differs from their own:
		\begin{align*}
			\mathbb{E} \left[ \mathbb{E} [\theta|z_j, p_2, p_1] | x_i, p_1\right] = \mathbb{E} \left[ \theta | x_i, p_1\right],
		\end{align*}
		hence early traders only trade (amongst themselves) for the same reason as before -- they \alert{disagree about the resale value}.
	\end{itemize}
\end{frame}


\begin{frame}{Brown-Jennings: public signals}
	\begin{itemize}
		\item Revealing a \alert{public signal $y$} about $\theta$ at $t=1$ would make traders \structure{agree more} (they would put less weight on $x_i$), hence there would be \structure{less trade} at $t=1$; the less trade, the more informative $y$ is.
		\item But in reality, there is a lot of trading when public news are revealed (\citet*{bailey_economic_2006}). \textbf{Potential explanations} include:
		\begin{itemize}
			\item Announcements are made up of public \structure{and private} signals \\
			(in the presence of HFTs, public news $\approx$ HFTs' private news)
			\item Or traders have \structure{heterogeneous priors} and can therefore `agree to disagree'; announcements then can amplify or mitigate these initial disagreements and so generate trade.
		\end{itemize}
		\item Turns out, there's another explanation: if you craft a more elaborate information structure, you can generate disagreement from public news!
	\end{itemize}
\end{frame}



\section{\cite{kondor_more_2012}: beliefs}

\begin{frame}{\cite{kondor_more_2012}: Example}
	\begin{itemize}
		\item Two groups of traders again, $I$ and $J$
		\item Fundamental value has two components: $\theta=\theta_{I}+\theta_{J}$
		\item I-trader signal: $x_i = \theta_{I}+\epsilon^{i}$
		\item J-trader signal: $z_j = \theta_{J}+\epsilon^{j}$
		\item Public signal: $y = \theta + \eta$
		\item Suppose $\theta_{I},\theta_{J}, \epsilon^{i}, \epsilon^{j}, \eta$ are independent and normal with zero mean
	\end{itemize}
	Main features: public signal about all of $\theta$, private signals about \emph{different aspects} of the fundamentals, $\theta_I$ and $\theta_J$.
\end{frame}


\begin{frame}{Example: Before public announcement}
	\textbf{No public news} ($y$ not observed)
	\begin{itemize}
		\item Traders' beliefs about $\theta$ are
		\begin{align*}
			\mathbb{E}(\theta|z_j) & =\mathbb{E}(\theta_{J}|z_j)+\mathbb{E}(\theta_{I}|z_j)=a_J z_j+0, a_J>0 \\
			\mathbb{E}(\theta|x_i) &=\mathbb{E}(\theta_{J}|x_i)+\mathbb{E}(\theta_{I}|x_i)=a_I x_i+0, a_I>0
		\end{align*}
		\item $I$-trader's second-order belief is
		\[
		\mathbb{E}(\mathbb{E}(\theta|z_j)|x_i)=\mathbb{E}(a_J z_j | x_i) = 0,
		\]
		i.e., $\mathbb{E}(\mathbb{E}(\theta|z_j)|x_i) = \mathbb{E}(\theta)$ because $x_i$ and $z_j$ are independent
	\end{itemize}
\end{frame}


\begin{frame}{Example: With public announcement}
	\textbf{Given public signal $y$}
	\begin{itemize}
		\item Traders' beliefs about $\theta$ are
		\begin{align*}
			\mathbb{E}(\theta|z_j, y) & =\mathbb{E}(\theta_{J}|z_j,y)+\mathbb{E}(\theta_{I}|z_j,y)=(b_J z_j+c_Jy)+d_J y,  \\
			\mathbb{E}(\theta|x_i, y) & =\mathbb{E}(\theta_{J}|x_i,y)+\mathbb{E}(\theta_{I}|x_i,y)=(b_I x_i+c_I y)+d_I y,
		\end{align*}
		where $b_k,c_k,d_k>0$ and $b_k<a_k$. \\
		First-order beliefs of $i$-traders \structure{converge} due to public signal
		\pause
		\item But $I$-agent's second-order belief  is
		\begin{align*}
			\mathbb{E}(\mathbb{E}(\theta|z_j,y)|x_i,y)
			&=b_J\mathbb{E}(z_j|x_i,y)+(c_J+d_J)y, \\
			&=b_J(ey-fx_i)+(c_J+d_J)y 
		\end{align*}
		where $e,f>0$. 
		
	\end{itemize}
\end{frame}


\begin{frame}{Example: Conclusion}
	\begin{itemize}
		\item $\mathbb{E}(\mathbb{E}(\theta|z_j,y)|x_i,y)$ is \alert{decreasing in $x_i$} and the weight on $x_i$ \emph{increases} with the precision of $y$
		\item i.e., second-order beliefs \structure{diverge} among $i$-traders: the more precise $y$ is, the less $I$-traders agree about the resale value of the asset $\Rightarrow$ more trade among $i$!
		\item This disagreement generates trade after public signals.
		\item The remainder of the slides presents the Kondor's trading model and derivations in \emph{slightly} more detail.
	\end{itemize}
\end{frame}



\section{\cite{kondor_more_2012}: trade model}


\begin{frame}{\cite{kondor_more_2012}: Full(er) Model}
	\begin{itemize}
		\item \textbf{Timing}: 
		\begin{enumerate}
			\item I-traders observe their information and trade
			\item I-traders liquidate all their positions and sell to J-traders
			\item $\theta$ (distributed as before) is realized and J-traders consume asset
		\end{enumerate}
		\item \textbf{Traders}: Price takers, $i, j \sim U(0,1)$, demand $d_i(p_t)$, util $u(W_i)=-e^{-\gamma W_i}$ and  wealth 
		\begin{align*}
			W_I & = (p_2-p_1){d_I}; &
			W_J & = (\theta-p_2){d_J}.
		\end{align*}
		\item \textbf{Supply}: Time-$t$ asset supply  $u_t$ (from noise traders): \\$u_1 \sim \mathcal{N}(0,1/\delta^2_1)$, $u_2\sim \mathcal{N}(0,1/\delta^2_2)$ ($u_2 \equiv u_1 + \varDelta u_2$)
	\end{itemize}
\end{frame}


\begin{frame}{Analysis: Trader maximization}
	\begin{itemize}
		\item (NOTE: some parameter names are different to paper)
		\item \textbf{Random supply}: Implies that prices are not completely informative.
		\item \textbf{I traders}: Solve
		\[
			\max_{d_I} \mathbb{E}\left[ -e^{-\gamma W_I}| x_i, y, p_1 \right]
		\]
		\begin{itemize}
			\item CARA utility and normal distributions $\Rightarrow$ can rewrite $I$ traders' problem as
		\end{itemize}
		\[
			\max_{d_I} \left\{\mathbb{E}\left[W_I| x_i, y, p_1\right]-\frac{\gamma}{2} \mathbb{V}\left[W_I| x_i, y, p_1\right]\right\}
		\]
		\item \textbf{J traders}: Solve
		\[
			\max_{d_J} \left\{\mathbb{E}\left[W_J| z_j, y, p_1, p_2\right]-\frac{\gamma}{2} \mathbb{V}\left[W_J| z_j, y, p_1, p_2\right] \right\}
		\]
	\end{itemize}
\end{frame}


\begin{frame}{Analysis: Trader maximization (2)}
	\begin{itemize}
		\item Taking the FOC and solving for the demands we get 
		\begin{align}
			d^*_{1,i} &= \frac{\tau^2_{p_2}}{\gamma}(\mathbb{E}[p_2| x_i, y, p_1]-p_1), \label{eqfoc1} \\
			d^*_{2,j} &= \frac{\tau^2_\theta}{\gamma} (\mathbb{E}[\theta| z_j, y, p_1, p_2]-p_2), \label{eqfoc2}
		\end{align}
		where $\tau^2_{p_2}=1/\mathbb{V}(p_2| x_i, y, p_1)$ and $\tau^2_{\theta}=\mathbb{V}(\theta| z_j, y, p_1, p_2)$
		\item In order to calculate expectations, need to make a conjecture about prices
	\end{itemize}
\end{frame}


\begin{frame}{Analysis: Linear prices and price signals}
	\begin{itemize}
		\item \textbf{Equilibrium}: Look for equilibrium with linear price function/demand
		\begin{align}
			p_1 & = \frac{1}{e_1}\left[a_1 \theta_I + c_1 y - u_1 \right]\label{price1}\\
			p_2 & =\frac{1}{e_2}\left[ b_2 \theta_J + c_2 y +  g_2 q_1-u_2 \right], \label{price2}
		\end{align}
		for some $a_1,b_2,c_1,c_2,e_1,e_2,g_2$, 
		where $q_1$ is the \structure{price signal} of $\theta_I$
		\item \textbf{Price signal}: This tells us the information contained in prices:
		\begin{align}
			q_1 & = \mathbb{E}[\theta_I|p_1,y]= \frac{e_1 p_1-c_1 y}{a_1} = \theta_I - \frac{1}{a_1} u_1 ; \label{signal1} \\
			q_2 &=\mathbb{E}[\theta_J|p_1,p_2,y]= \frac{e_2 p_2-c_2 y - g_2 q_1}{b_2} = \theta_J - \frac{1}{b_2} u_2. \label{signal2}
		\end{align}
	\end{itemize}
\end{frame}


\begin{frame}{Analysis: Reformulating in terms of price signals}
	\begin{itemize}
		\item \textbf{Rewrite expectation}. All variables jointly normal $\rightarrow$ linear expressions
		\begin{align}
		\mathbb{E}[p_2| x_i, y, p_1] & = a^e_1 x_i+b^e_1 y +c^e_1 q_1;   \label{exp1} \\
		\mathbb{E}[\theta| z_j, y, p_1, p_2] & =  a^e_2 z_j+b^e_2 y +c^e_2 q_1 +d^e_2 q_2. \label{exp2}
		\end{align}
		\item \textbf{Rewrite FOC}. Plugging \eqref{exp1} and \eqref{exp2} into \eqref{eqfoc1} and \eqref{eqfoc2} we get
		\begin{align}
		d^*_{1,i} &= \frac{\tau^2_{p_2}}{\gamma}(a^e_1 x_i+b^e_1 y +c^e_1 q_1 - p_1); \label{eqfocb1} \\
		d^*_{2,j} &= \frac{\tau^2_\theta}{\gamma}(a^e_2 z_j+b^e_2 y +c^e_2 q_1 +d^e_2 q_2 - p_2). \label{eqfocb2}
		\end{align}
		\item \textbf{Market clearing}: $u_1=\int_0^1 d^*_{1,i} di$ and $u_2=\int_0^1 d^*_{2,j} dj$ determine $p_1$ and $p_2$ resp.
	\end{itemize}
\end{frame}


\begin{frame}{Analysis: Equilibrium}
	\begin{itemize}
		\item \textbf{Matching coefficients}: From market clearing, can show that $p_t$ is linear function as conjectured
		\item \textbf{Equilibrium demand}: Matching up all the coefficients, we can then show that
		\begin{align}
		d^*_{1,i} &=  a_1 x_i+c_1 y -e_1 p_1; \label{eqdem1} \\
		d^*_{2,j} &= b_2 z_j+c_2 y + g_2q_1-e_2 p_2. \label{eqdem2}
		\end{align} 
		\item Demand is increasing in private signal ($x_i$/$z_j$), in public signal ($y$), in price signal ($q_1$), and decreasing in price ($p_t$) 
		(recall that traders are price takers)
	\end{itemize}
\end{frame}


\begin{frame}{Results: Demand period 2}
	\begin{itemize}
		\item Let's look at what drives the agents' demands. Rewrite period-2 demand as 
		\begin{block}{}
			\begin{equation} \label{eqdem2z}
			d^*_{2,j} = b_2 (z_j - q_2) 
			\end{equation}
		\end{block}
		\item Notice that $z_j$ is $j$'s private signal and  $q_2$ is a noisy signal of all the other agents' signals
		\item Thus, if $j$ believes to have received a better signal than everybody else, he will buy, otherwise sell
		\item This is a standard story: $J$-traders trade due to a \textbf{difference in opinion} -- they think the asset is worth more/less than others
	\end{itemize}
\end{frame}


\begin{frame}{Results: Demand period 1}
	\begin{itemize}
		\item Market clearing in period 2 together with \eqref{eqfoc2} implies
		\[
		p_2 = \int_0^1 \mathbb{E}[\theta|z_j, y , q_1, q_2] dj - \frac{\gamma}{\tau^2_\theta}u_2
		\]
		\item Rewrite period-1 demand using this:
		\[
			d^*_{1,i} = \frac{\tau^2_{p_2}}{\gamma} \left( \underbrace{\mathbb{E}\left[ \int_0^1 \mathbb{E}[\theta \mid z_j, y , q_1, q_2 ] dj -\frac{\gamma}{\tau^2_\theta}u_2 \,\bigg|\, x_i,y,q_1 \right]}_{\text{2nd order expectation}}-p_1 \right)
		\]
		\item $I$-trader demand in period 1 is thus a function of a \textbf{second-order expectation}: 
		The more $i$ expects $J$ traders to value the asset, the more he buys 
	\end{itemize}
\end{frame}


\begin{frame}{Results}
	\begin{itemize}
		\item In the paper, Kondor considers a more general information structure where there is a common factor about which $I$ and $J$ both learn.
		\item He then defines \textit{weakly correlated} information structures in which the common factor  is not too important
		\item In the above, we have disregarded the common factor, so what we analyzed is automatically a weakly correlated information structure
		\begin{block}{Main result}
			If the information structure is \textit{weakly correlated}, then trading intensity, volume and informational content of prices \structure{increase} in both periods when there is more public info.
		\end{block}
		\item Public signals  create trade, due to their effect on second-order beliefs
	\end{itemize}
\end{frame}


\begin{frame}{Model 2}
	\textbf{Heterogenous trading horizons}
	\begin{itemize}
		\item Timing
		\begin{enumerate}
			\item I-traders and J-traders trade
			\item I-traders sell all their holdings to J-traders
			\item $\theta$ is realized and J-traders consume
		\end{enumerate}
		\item Let $\mu$ be the proportion of J-traders
	\end{itemize}
	\textbf{Interpretation}
	\begin{itemize}
		\item Traders with different trading horizons co-exist in the market
		\item For instance day-traders and pension savers
	\end{itemize}
\end{frame}


\begin{frame}{Model 2: Results}
	\begin{itemize}
		\item When $\mu$ is high, most traders trade with each other: the market is well-integrated
		%\item Public information will make J-traders trade more
		%\begin{itemize}
		%	\item When prices become more correlated with fundamentals, J-traders can better forecast them
		%\end{itemize}
		\item When $\mu$ is small, the results of model 1 are close to those of model 2
		\item But when $\mu$ is high, public information crowds out private information, and public signals have the usual effect
		\begin{itemize}
			\item I.e., there will be less disagreement and less trade
		\end{itemize}
		\item Thus, integration is key to the results (what happens to ST speculation as market becomes more integrated?)
	\end{itemize}
\end{frame}


\begin{frame}{Relation to empirics}
	\begin{itemize}
		\item In general, the model provides an explanation for trade after public announcements
		\item \citet*{bailey_economic_2006} find that price volatility and trading volumes increase after earnings announcements
		\begin{itemize}
			\item They find that the effect is larger for cross-listed stock
		\end{itemize}
		\item Kondor argues that cross-listing is roughly equivalent to lower market integration: lower $\mu$
	\end{itemize}
\end{frame}


\begin{frame}{Kondor: Conclusion}
	\begin{itemize}
		\item Public announcements can affect second-order beliefs, thereby generating trade and increasing price volatility!
		\item This requires some very specific assumptions on the information structure in the market though...
		\item The model goes primarily towards explaining some empirical puzzles; not clear whether we should base welfare analysis on it.
		\item But it should allow us to predict better which stocks will react strongly to announcements
	\end{itemize}
\end{frame}

\appendix
\begin{frame}[allowframebreaks]{References}
	\bibliography{../teaching}
	\bibliographystyle{abbrvnat}
\end{frame}


\end{document} 