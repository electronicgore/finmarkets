%%% License: Creative Commons Attribution Share Alike 4.0 (see https://creativecommons.org/licenses/by-sa/4.0/)
%%% Slides are based heavily on earlier versions of this course taught by Jesper Rudiger.

\documentclass[english,10pt]{beamer}
%\documentclass[english,10pt,handout]{beamer}
%%% License: Creative Commons Attribution Share Alike 4.0 (see https://creativecommons.org/licenses/by-sa/4.0/)
%%% Slides are based heavily on earlier versions of this course taught by Jesper Rudiger and Peter Norman Sorensen.

\DeclareGraphicsExtensions{.eps, .pdf,.png,.jpg,.mps,}
\usetheme{reMedian}
\usepackage{parskip}
\makeatother

\renewcommand{\baselinestretch}{1.1} 

\usepackage{amsmath, amssymb, amsfonts, amsthm}
\usepackage{enumerate}
\usepackage{hyperref}
\usepackage{url}
\usepackage{bbm}
\usepackage{color}

\usepackage{tikz}
\usepackage{tikzscale}
\newcommand*\circled[1]{\tikz[baseline=(char.base)]{
		\node[shape=circle,draw, inner sep=-20pt] (char) {#1};}}
\usetikzlibrary{automata,positioning}
\usetikzlibrary{decorations.pathreplacing}
\usepackage{pgfplots}
\usepgfplotslibrary{fillbetween}
\usepackage{graphicx}

\usepackage{setspace}
%\thinmuskip=1mu
%\medmuskip=1mu 
%\thickmuskip=1mu 


\usecolortheme{default}
\usepackage{verbatim}
\usepackage[normalem]{ulem}

\usepackage{apptools}
\AtAppendix{
	\setbeamertemplate{frame numbering}[none]
}
\usepackage{natbib}




\title{Financial Markets Microstructure \\ Lecture 6}

\subtitle{Limit order book\\
	Chapter 6 of FPR}

\author{Egor Starkov}

\date{K{\o}benhavns Unversitet \\
	Spring 2020}




\begin{document}
\AtBeginSection[]{
\frame<beamer>{
\frametitle{This lecture:}
\tableofcontents[currentsection,currentsubsection]
}}
\frame[plain]{\titlepage}

\section{Revision, exercises and introduction}

\begin{frame}{Last time}
	\begin{itemize}
		\item The Kyle model helps us analyze market depth:
		\begin{itemize}
			\item Hence, it tells us something about how adverse selection causes spread to vary with trade size
			\item The model has batch clearing instead of the single-unit market of GM
		\end{itemize}
		\item We can use insights from theory to estimate the importance of different components of the spread
		\begin{itemize}
			\item Perhaps surprisingly, order costs are by far the largest cost (but estimated on major stocks)
			\item Easley et al.: around 19\% of trading is informed
		\end{itemize}
	\end{itemize}
\end{frame}


\begin{frame}{Exercise}
	\begin{itemize}
		\item I uploaded two articles from the Economist, March 7, 2014, both on potential manipulation of prices
		\item Consider these two issues:
		\begin{itemize}
			\item What motivates a trader to manipulate an asset's price? 
			\item Is manipulation easier in a very liquid or a very illiquid market?
		\end{itemize}
	\end{itemize}
\end{frame}


\begin{frame}{Introduction}
	\begin{itemize}
		\item \structure{Price discrimination}:  LOB has discriminatory pricing of larger orders
		\begin{itemize}
			\item Difference to Kyle's (auction) model: in LOB, total order size is not observed, you only know that your limit order has been `hit'
			\item Glosten (1994) provides a Glosten-Milgrom style model of the price schedule offered (i.e., how much volume supplied at the limit prices)
		\end{itemize}
		\item \structure{Ticks}: Often prices are discrete and must lie at a tick (e.g. at NYSE, ticks are multiples of 1 cent) --  tick size is the increment btw prices:
		\begin{itemize}
			\item Prioritized limit orders become profitable when there are ticks 
			(since no `marginal undercutting')
			\item We will analyze the effects of changing ticks, priority rules
		\end{itemize}
		\item \structure{Interpretation}: One way of thinking of an LOB is that dealers post limit orders, as in the GM model
	\end{itemize}
\end{frame}



\section{Static analysis: \cite{glosten_is_1994}}

\begin{frame}{Continuous model: Limit order book}
	\textbf{Single asset}: Unknown value $v$ with cumulative distribution  $G(v)$
	\\
	\textbf{Limit order book}: 
	\begin{itemize}
		\item Prices vary with volume $q$. Posted limit orders: $P'(q)$
		\item A market buy order of size $q$ will `walk up the book' until the final bit of it is cleared at a limit price $P'(q)$
		\item The entire payment for buying volume $q$ is then
		\[
		P(q) = \int_0^q P'(\tilde{q}) \, d\tilde{q}.
		\]
		\item The average price is 
		\[
		p(q) = \frac{P(q)}{q}.
		\]
	\end{itemize}
\end{frame}


\begin{frame}{Continuous model: Marginal rate of substitution}
	\begin{itemize}
		\item Suppose trader $i$ has  marginal rate of substitution $\theta_i(q)$
		\item If an incoming trader $i$ chooses a market buy order $q$ and parts with cash $P(q)$, we should have: \[\theta_i(q) = P'(q).\]
		\item If incoming traders are to some extent informed, it's reasonable to suppose that for a given $q$, higher $\theta_i(q)$ indicates higher asset value $v$
		\item Thus, we assume that 
		\[
		\mathbb{E}[v|\theta_i(q) = \theta]
		\]
		is strictly increasing in $\theta$
	\end{itemize}
\end{frame}


\begin{frame}{Continuous model: Market makers}
	\begin{itemize}	
		\item Competitive limit orders posted by uninformed, risk-neutral market makers
		\item Limit price $P'(q)$ quoted for the $q$-th unit is relevant only if the trader places a market order of $q$ or larger, i.e. $\theta_i(q) \geq P'(q)$
		\item Hence, if supply is competitive then
		\[
		\structure{P'(q) = \mathbb{E}[v|\theta_i(q) \geq P'(q)]}
		\]
		\item Notice that all we know if limit price $P'(q)$ is `hit' is that the trader's MRS was \textit{at least} $P'(q)$
	\end{itemize}
\end{frame}


\begin{frame}{Continuous model: Equilibrium}
	Adverse selection implies
	\begin{align}
	P'(q) = \mathbb{E}[v|\theta_i(q) \geq P'(q)] > \mathbb{E}[v|\theta_i(q) = P'(q)]
	\end{align}
	\begin{block}{Implications}
		\begin{itemize}
			\item Thus, market makers (ex post) profit on the sale of the last units
			\item At small realized trades, $q \simeq 0$, MM always profit
			\item Even with continuous prices, there is a non-zero \structure{inside spread} between ask and bid prices as the order size goes to zero (contrast with Kyle)
			\item After a trade of size $q$, new expected asset value is below $P'(q)$. \pause \structure{New expected value = $\mathbb{E}[v|\theta_i(q)=P'(q)]<\mathbb{E}[v|\theta_i(q) \ge P'(q)]=P'(q)$}. Often new limit orders will be posted below $P'(q)$
		\end{itemize}
	\end{block}
\end{frame}


\begin{frame}{Discrete model: Setup}
	\begin{itemize}
		\item \textbf{Asset}: Continue with single asset with value $v \sim G$
		\item \textbf{Market order $q$}
		\begin{itemize}
			\item Cumulative distribution function $F(q)$
			\item Focus on the ask side of the book, $q>0$
		\end{itemize}
		\item \textbf{Discrete price grid}
		\begin{itemize}
			\item $A_1$ is lowest price tick above $\mu$
			\item $A_k-A_{k-1}>0$ is the tick size
		\end{itemize}
		\item \textbf{Limit orders}
		\begin{itemize}
			\item \textit{Time prioritized}: first posted, first executed
			\item \textit{Display cost}: $C$ per unit (paid regardless of whether order executes)
			\item Let $y_k$ denote the amount supplied at price $A_k$
			\item Let $Y_k=\sum^k_{j=1} y_j$ be cumulative volume (depth) supplied at or below $A_k$
		\end{itemize}
	\end{itemize}
\end{frame}


\begin{frame}{Discrete model: Equilibrium}
	Let $\mathbb{E}[v|q \geq Y_k]=\mathbb{E}[v|\theta_i(q) \geq P'(q)]$.
	\begin{itemize}
		\item \textbf{Competition}: Limit orders are supplied at each tick until the \structure{last order earns zero profit}
		\item \textbf{Zero-profit condition}:
		\[
		\mathbb{P}(q \geq Y_k)[A_k-\mathbb{E}[v|q \geq Y_k]] - C = 0,
		\]
		solved by
		\begin{block}{}
			\[
			A_k = \mathbb{E}[v| q \geq Y_k] + \frac{C}{\mathbb{P}(q\geq Y_k)}
			\]
		\end{block}
		\item \textbf{Adverse selection}: Captured by  $\mathbb{E}[v| q \geq Y_k] $
	\end{itemize}
\end{frame}


\begin{frame}{Example 1: Model} 
	\begin{itemize}
		\item \textbf{Asset}. Let $g$ be the marginal distribution of $G$ and
		\[
		g(v)=\left\{ \begin{aligned}
		1/2  & \text{ if } v=v^H; \\
		1/2 & \text{ if } v=v^L,
		\end{aligned}
		\right.
		\]
		with $v^{H}=\mu + \sigma $ and $v^{L}= \mu - \sigma $.
		\item \textbf{Traders}. Single trader, who uses a market order.
		\begin{itemize}
			\item Prob. $\pi$: risk-neutral speculator (\structure{S}) who knows $v$
			\item Prob $1-\pi$: noise trader (\structure{N})  who buys/sells with equal probability, and uses large ($q_L$) or small ($q_S<q_L$) order with equal probability:
			$$
			\mathbb{P}(q=q_S|\structure{N})= \mathbb{P}(q=q_L|\structure{N})=\mathbb{P}(q=-q_S|\structure{N})=\mathbb{P}(q=-q_L|\structure{N})=1/4
			$$
		\end{itemize}
		\item \textbf{No display cost}. Let $C=0$
	\end{itemize}
\end{frame}


\begin{frame}{Example 1: Equilibrium}
	\begin{itemize}
		\item \textbf{Separating equilibrium}: Look for eq. with $y_{1}=q_{S}$ and $y_{2}=q_{L}-q_{S}$ for some ask prices $v^L<A_1<A_2<v^H$ (hence $Y_1=q_S$ and $Y_2=q_L$)
		\item \textbf{Speculator}: 
		\begin{itemize}
			\item If $q \notin \{q_S,q_L\}$, speculator reveals himself
			\item 
			When observing $v=v^{H}$ speculator buys $q_{L}$ units, when $v=v^L$ don't buy (maybe sell)
		\end{itemize}
		\item \textbf{Price}. In equilibrium, price equals expected value given $q$. Hence,
		\begin{block}{}
			\[\begin{aligned}
			A_1 & =\mathbb{E}[v|q \geq q_{S}] = \mu + \pi \sigma,\\
			A_2 & = \mathbb{E}[v|q \geq q_{L}] = \mu + \frac{2\pi}{1+\pi} \sigma
			\end{aligned}
			\]
		\end{block}
		\item Straightforward that $v^L<A_1<A_2<v^H$. Thus: equilibrium.
	\end{itemize}
\end{frame}


\begin{frame}{Example 2: Model}
	\begin{itemize}
		\item \textbf{Market orders}: Exponential distribution, $f(q)=\frac{\theta}{2} e^{-\theta|q|}$
		\item \textbf{Asset}.  Assume `price impact' equation $\mathbb{E}[v|q=x] = \mu + \lambda x$ where $\lambda >0$ is a constant measuring informativeness of order flow
		\item Thus,  we are taking a short-cut and modeling adverse selection in a `reduced form': rather than modeling the informed traders, we model their price impact
	\end{itemize}
\end{frame}


\begin{frame}{Example 2: Equilibrium}
	\begin{itemize}
		\item Focus on  \textbf{ask side}: $Y_k>0$. For $q \geq Y_{k}$:
		\begin{align*}
		f(q|q \ge Y_k)&=\frac{f(q)}{ \mathbb{P}(q \geq Y_{k})} \\
		&= \frac{f(q)}{\int^\infty_{Y_k} f(q) dq}\\
		&= \frac{\frac{\theta}{2} \cdot e^{-\theta q}}{e^{-\theta Y_{k}}/2} \quad \quad \text{ ($|q|=q$ since $q \ge Y_k>0$)}\\
		& =\theta \cdot e^{-\theta(q-Y_{k})}\\
		& = e^{Y_k} \left[ \theta \cdot e^{-\theta q} \right]
		\end{align*}
	\end{itemize}
\end{frame}


\begin{frame}{Example 2: Equilibrium (2)}
	\begin{itemize}
		\item The expected value at tick $k$ becomes
		\begin{align*}
		\mathbb{E}[v|q \geq Y_{k}] & =  \mu + \lambda \mathbb{E}[q|q \geq Y_{k}]\\
		&=\mu+ \lambda \int^\infty_{Y_k} q \cdot f(q|q \geq Y_{k}) dq\\
		&=\mu+ \lambda e^{\theta Y_{k}} \int^\infty_{Y_k} q \cdot \theta \cdot e^{-\theta q} dq\\
		& =  \mu+ \lambda e^{\theta Y_{k}} \left\{\left[- q \cdot e^{-\theta q} \right]^\infty_{Y_k} -\int^\infty_{Y_k} -e^{-\theta q} dq   \right\}\text{ (int. by parts)} \\
		& =  \mu+ \lambda e^{\theta Y_{k}} \left\{\left[0+ Y_k \cdot  e^{-\theta Y_k} \right] -\frac{1}{\theta}[0-e^{-\theta Y_k}]   \right\} \\
		& = \mu + \lambda \left( \frac{1}{\theta }+ Y_{k} \right)
		\end{align*}
	\end{itemize}
\end{frame}


\begin{frame}{Example 2: Equilibrium (3)}
	\begin{itemize}
		\item Hence, the ask price at tick $k$ can be found by solving
		\[
		\underbrace{\structure{\mathbb{P}(q \geq Y_k)}}_{e^{-\theta Y_{k}}/2}[A_k-\underbrace{\structure{\mathbb{E}[v|q \geq Y_k]}}_{\mu + \lambda \left( \frac{1}{\theta }+ Y_{k} \right)}] - C = 0,
		\]
		which gives
		\begin{block}{}
			\[
			A_k=\mu + \lambda \left( \frac{1}{\theta }+ Y_{k}\right) + \frac{2C}{e^{-\theta Y_{k}}}.
			\]
		\end{block}
	\end{itemize}
\end{frame}



\section{Market design}

\begin{frame}{Market design: Tick size and priority rules}
	\begin{itemize}
		\item With \structure{time priority} (first limit order posted at tick executes first)
		\begin{itemize}
			\item The marginal order at price $A_{k}$ earns zero profit under `time priority'
			\item The order at the front of the queue executes with greater  probability than the order at the end of the queue: $\mathbb{P}(q > Y_{k-1}) \ge \mathbb{P}(q \geq Y_{k})$
			\item So inframarginal orders may earn strictly positive profits
			\item Note that this profit is reduced with smaller tick sizes
		\end{itemize}
		\item With \structure{pro-rata} (limit orders at given tick executed proportionally to their size):
		\begin{itemize}
			\item The expected profit of all orders at price $A_k$ must be zero
			\item Since profits are decreasing in the number of orders (lower execution probability), this implies that more orders are posted under pro-rata
			% not sure about this last part
		\end{itemize}
	\end{itemize}
\end{frame}


\begin{frame}[label=hybridorg]{Market design: Hybrid market}
	Hybrid market
	\begin{itemize}
		\item Suppose a dealer can compete with the limit order book, as follows
		\item The dealer may observe the size $q$ before serving the order (last-mover advantage)
		\item Can profit by pricing at $\mathbb{E}[v|q=x]$ rather than an average of $\mathbb{E}[v|q \geq y]$ for $y \leq x$ which is used by the LOB
		\item Especially profitable on small trades
		\item But the existence of such a dealer invalidates our analysis of the LOB
		\item Profitable limit orders are being picked off
	\end{itemize}
	\hyperlink{hybrid}{\beamerbutton{Example}}
\end{frame}


\begin{frame}{\cite{glosten_is_1994}: Conclusion}
	\begin{itemize}
		\item Ordinary traders can  use limit orders to compete with market makers
		\begin{itemize}
			\item Our model of market orders?
		\end{itemize}
		\item Limit order book is actually dynamic
		\begin{itemize}
			\item Consequences of priority rules?
		\end{itemize}
	\end{itemize}
	Glosten argued that if orders can be freely split between competing trading platforms (which need not have limit order books), then the aggregate pricing schedule must be that from the competitive order market
\end{frame}



\section{Dynamic analysis: the choice between limit and market orders}

\begin{frame}{Dynamic analysis}
	\begin{itemize}
		\item \cite{glosten_is_1994} models the static price schedule under adverse selection (with quote display costs)
		\item Traders must choose: place limit/market orders = make/take liquidity
		\begin{itemize}
			\item \cite{parlour_price_1998}: time priority of limit orders means it's more attractive to add limit orders when there's currently few of them
			\item \cite{foucault_order_1999}: limit order placed by traders whose private valuation makes current prices unattractive
			\item \citet*{foucault_limit_2005}: patient traders place limit orders inside the spread
		\end{itemize}
	\end{itemize}
\end{frame}


\begin{frame}{Model based on Parlour (1998)}
	\begin{itemize}
		\item \textbf{Exogenous prices}. Bid and ask prices exogenously given as $A>v>B$
		\item \textbf{Traders}. Arriving trader chooses btw limit or market order (one unit)
		\begin{itemize}
			\item Limit order valid one period. Choice depends on prob. of limit order being executed, i.e. `hit' by a market order from the next trader
			\item Valuation: $v+y$. $y$ is uniformly distributed on $(-Y,Y)$, unobserved and independent across traders. $v$ is known and common to all.
		\end{itemize}
		\item \textbf{Profits}. Let $P^B_M (P^S_M)$ be prop. of next-period market buy (sell) order 
		\begin{align*}
		\text{Market sell: } & B-(v+y) \\
		\text{Limit sell: } & (A-v-y)P^B_M \\
		\text{Limit buy: } &(v+y-B)P^S_M \\
		\text{Market buy: } &v+y-A
		\end{align*}
	\end{itemize}
\end{frame}


\begin{frame}{Analysis}
	\begin{itemize}
		\item Look for stationary eq. where  $A$, $B$, $P^B_M$, $P^S_M$ are constant 
		\item From the trader profits, we see that there must exist $\overline{y}$,  $\underline{y}$ and $\hat{y}$ such that the best response (optimal order) of the trader is
		\begin{equation*}
		BR=\left\{ \begin{aligned}
		&MSell		&& 	\text{ if } 	&&	-Y\le y \le \underline{y} \\
		&LSell 		&&	\text{ if } 	&&	\underline{y} \le y \le \hat{y} \\
		&LBuy		&&	\text{ if } 	&&	\hat{y} \le y \le \overline{y} \\
		&MBuy	&&	\text{ if } 	&&	\overline{y} \le y \le Y
		\end{aligned}
		\right.
		\end{equation*}
		\item Traders with greater urgency to trade (extreme $y$) use market orders
		\item Limit order more attractive when more likely to execute:  automatic tendency for the limit order book to be replenished (resiliency)
	\end{itemize}
\end{frame}


\begin{frame}{Equilibrium}
	\begin{itemize}
		\item From trader's BR:
		\begin{align*}
		P^S_M 	&=\mathbb{P}(-Y\le y \le \underline{y})=\frac{\underline{y}+Y}{2Y} \\
		P^B_M 	&=\mathbb{P}(\overline{y} \le y \le Y)=\frac{Y-\overline{y}}{2Y}
		\end{align*}
		\item At $\underline{y}$, indifferent btw $MSell$ and $LSell$. At $\hat{y}$, indifferent btw $LSell$ and $LBuy$. At $\overline{y}$, indifferent btw $LBuy$ and $MBuy$. Thus:
		\begin{align}
		B-(v+\underline{y}) 					& = (A-v-\underline{y})\frac{Y-\overline{y}}{2Y} \label{eq1}\\
		(A-v-\hat{y}) \frac{Y-\overline{y}}{2Y} 			& = (v+\hat{y}-B)\frac{\underline{y}+Y}{2Y} \label{eq2}\\
		(v+\overline{y}-B) \frac{\underline{y}+Y}{2Y}  	& = v+\overline{y}-A \label{eq3}
		\end{align}
	\end{itemize}
\end{frame}


\begin{frame}[label=adverseorg]{Equilibrium (2)}
	\begin{itemize}
		\item Solving \eqref{eq1}-\eqref{eq3} we obtain the thresholds. If we set $v=0$, $Y=2$, $A=1$ and $B=-1$ they become
		\begin{align*}
		\underline{p} 	& = \frac{1}{2}(3-\sqrt{33}) \simeq 1.4 \\
		\hat{p} 	& = 0 \\
		\overline{p} 		& = \frac{1}{2}(\sqrt{33}-3) \simeq -1.4 
		\end{align*}
		\item Thus, in equilibrium the probability of a market buy/sell order next period is $\frac{2-1.4}{2\cdot 2}=\frac{-1.4+2}{2 \cdot 2}=0.15$
		\item Given  large spread ($A-B=2$), limit order traders are willing to accept a low execution probability ($15\%$) in order to obtain better price
	\end{itemize}
	\hyperlink{adverse}{\beamerbutton{Adverse selection}}
\end{frame}


\begin{frame}{Conclusion}
	\begin{itemize}
		\item Trade occurs when seller and buyer meet and agree on terms
		\begin{itemize}
			\item Exchange brings sellers and buyers together
			\item Impatient traders search more actively (use market orders)
			\item Patient traders are more passive, offer liquidity (use limit orders)
		\end{itemize}
		\item Ordinary traders can use limit orders to compete with market makers
		\item Limit order book is dynamic: influence the choice of order
		\item In reality, many features to think about when placing orders
	\end{itemize}
\end{frame}


\begin{frame}{Exercise for next week}
	\begin{itemize}
		\item Solve exercise 1 on pages 232-233 in the textbook. Note that you need to use Bayes' rule to assess the conditional distribution over $v$ given a market order of size $q$
		\item Solve exercise 5 on page 235 in the textbook on the effect of fees charged for limit orders and market orders
		\item Suppose that there is no tick; quotes can  be placed in a continuous price space. Suppose that there is price priority. What is then the role of time priority, so that first-come quotes at identical prices are served first?
		\item If you could choose between trading at discriminatory prices in a limit order book, or to reveal your order size to a dealer, what would influence your choice?
	\end{itemize}
\end{frame}





\appendix
\begin{frame}[allowframebreaks]{References}
\bibliography{../teaching}
\bibliographystyle{abbrvnat}
\end{frame}


\begin{frame}[label=hybrid]{Example of hybrid market}
	\begin{itemize}
		\item Example 1 continued. Assume an uninformed dealer receives order $q$. Can either send order to LOB or execute himself (at better price) 
		\item Focus on ask side. Let $A^H_k$ be the hybrid ask price. When dealer observes $q=q_S$, he knows trader is noise trader and thus $\mathbb{E}[v|\structure{N}]=\mu$. 
		\begin{itemize}
			\item Can execute order at just below $A^H_1$ and earn profit $A^H_1-\mu$.
		\end{itemize}
		\item Hence, only large orders $q=q_L$ are sent to LOB. LOB traders will expect this, and will price as if any order arriving to the LOB is large:
		\begin{align*}
		\mathbb{E}[v|q \ge q_S]=\mathbb{E}[v|q \ge q_L]=\mu+\frac{2\pi}{1+\pi} \sigma,
		\end{align*}
		and thus $A^H_1=A^H_2=\mu+\frac{2\pi}{1+\pi} \sigma$.
		\item $A^H_1>A_1$ and $A^H_2=A_2$: hybrid market less liquid than normal market
	\end{itemize}
	\hyperlink{hybridorg}{\beamerbutton{Back}}
\end{frame}


\begin{frame}[label=adverse]{Dynamic model with adverse selection}
	\begin{itemize}
		\item Now, let us try to derive the time-$t$ prices $A_t$ and $B_t$ endogenously. 
		\item Suppose $v_t=v_{t-1}+\epsilon_t$ where $\epsilon_t \in \{-\sigma, \sigma \}$ with equal probability. $v_t$ is observed in period $t$. The period-$t$ trader has valuation $v_t+y_t$. 
		\item Again, we look for a stationary equilibrium. In particular:
		\begin{itemize}
			\item Suppose that $A_t-v_{t-1}=v_{t-1}-B_t=S$ is constant.
			\item Suppose there exist $\underline{y}(\epsilon_t),\overline{y}(\epsilon_t),\hat{y}(\epsilon_t)$  that describe the trader's strategy in \textit{each} period  as a function of the value innovation $\epsilon_t$ 
			\item Thus there exist time-invariant $P^S_M$ and $P^B_M$ that characterize the probability of a next-period market sell/buy order 
		\end{itemize}
	\end{itemize}
\end{frame}


\begin{frame}{Dynamic model with adverse selection (2)}
	\begin{itemize}
		\item We can write the equilibrium conditions as
		\begin{align*}
		B_t-(v_t+\underline{y}(\epsilon_t)) 					& = (A_{t+1}-\mathbb{E}_t[v_{t+1}|buy]-\underline{y}(\epsilon_t) 	)P^B_M \\
		(A_{t+1}-\mathbb{E}_t[v_{t+1}|buy]-\hat{y}(\epsilon_t))P^B_M 			& = (\mathbb{E}_t[v_{t+1}|sell]+\hat{y}(\epsilon_t)-B_{t+1})P^S_M  \\
		(\mathbb{E}_t[v_{t+1}|sell]+\overline{y}(\epsilon_t)-B_{t+1}) P^S_M  	& = v_t+\overline{y}(\epsilon_t)-A_t 
		\end{align*}
		\item Focus on the first and last equation. Use definition of $S$:
		\begin{align*}
		-S-\epsilon_t-\underline{y}(\epsilon_t) 					& = (S-\mathbb{E}_t[\epsilon_{t+1}|buy]-\underline{y}(\epsilon_t) 	)P^B_M \\
		(S+\mathbb{E}[\epsilon_{t+1}|sell]+\overline{y}(\epsilon_t)) P^S_M  	& =-S+\epsilon_t+\overline{y}(\epsilon_t)
		\end{align*}
		\item Solve for $\underline{y}(\epsilon_t)$ and $\overline{y}(\epsilon_t)$. Determine $S$ from $A_t=\mathbb{E}_t[v_t|buy]$ and $B_t= \mathbb{E}_t[v_t|sell]$. Notice adv. sel. affects only limit orders. \hyperlink{adverseorg}{\beamerbutton{Back}}
	\end{itemize}
\end{frame}


\end{document} 