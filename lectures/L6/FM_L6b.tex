%%% License: Creative Commons Attribution Share Alike 4.0 (see https://creativecommons.org/licenses/by-sa/4.0/)
%%% Slides are based heavily on earlier versions of this course taught by Jesper Rudiger.

\documentclass[english,10pt
%,handout
,aspectratio=169
]{beamer}
%%% License: Creative Commons Attribution Share Alike 4.0 (see https://creativecommons.org/licenses/by-sa/4.0/)
%%% Slides are based heavily on earlier versions of this course taught by Jesper Rudiger and Peter Norman Sorensen.

\DeclareGraphicsExtensions{.eps, .pdf,.png,.jpg,.mps,}
\usetheme{reMedian}
\usepackage{parskip}
\makeatother

\renewcommand{\baselinestretch}{1.1} 

\usepackage{amsmath, amssymb, amsfonts, amsthm}
\usepackage{enumerate}
\usepackage{hyperref}
\usepackage{url}
\usepackage{bbm}
\usepackage{color}

\usepackage{tikz}
\usepackage{tikzscale}
\newcommand*\circled[1]{\tikz[baseline=(char.base)]{
		\node[shape=circle,draw, inner sep=-20pt] (char) {#1};}}
\usetikzlibrary{automata,positioning}
\usetikzlibrary{decorations.pathreplacing}
\usepackage{pgfplots}
\usepgfplotslibrary{fillbetween}
\usepackage{graphicx}

\usepackage{setspace}
%\thinmuskip=1mu
%\medmuskip=1mu 
%\thickmuskip=1mu 


\usecolortheme{default}
\usepackage{verbatim}
\usepackage[normalem]{ulem}

\usepackage{apptools}
\AtAppendix{
	\setbeamertemplate{frame numbering}[none]
}
\usepackage{natbib}




\title{Financial Markets Microstructure \\ Lecture 11}

\subtitle{Limit order book, part 2\\
	Chapter 6.2-6.3 of FPR}

\author{Egor Starkov}

\date{K{\o}benhavns Unversitet \\
	Spring 2023}




\begin{document}
\AtBeginSection[]{
\frame<beamer>{
\frametitle{This lecture:}
\tableofcontents[currentsection,currentsubsection]
}}
\frame[plain]{\titlepage}


\begin{frame}{Last time}
	\begin{itemize}
		\item \structure{Glosten model}: see how the behavior of competitive liquidity providers in a LOB is different from dealers' behavior.
		\item Pricing rule: \alert{marginal price} of $q$th unit (on the ask side) is $$p(q) = \mathbb{E} [v | x \geq q].$$
		\item Reminder: in a dealer market, average price when $q$ units are traded is $$\bar{p}(q) = \mathbb{E} [v | x = q].$$
	\end{itemize}
\end{frame}


\section{Static Analysis: Glosten Model (discrete ticks)}

\begin{frame}{Ticks}
	\begin{itemize}
		\item The version of the Glosten model we've seen last time outlines some basic differences between LOB and dealer markets
		\begin{itemize}
			\item Adverse selection affects prices differently
		\end{itemize}
		\item It neglects the \structure{discreteness of prices}:
		\begin{itemize}
			\item Often prices are discrete and must lie at a tick --  tick size is the increment b/w prices
			\item E.g. at NYSE it was \$1/8 for stocks with prices over one dollar until June 1997, when, under regulatory pressure, it was reduced to \$1/16 and finally, in 2000, to one cent.
		\end{itemize}
		\item Q for today: how does this discreteness (and tick size in particular) affect market outcomes?
		\begin{itemize}
			\item Prioritized limit orders become profitable when there are ticks (since no `marginal undercutting')
		\end{itemize}
	\end{itemize}
\end{frame}


\begin{frame}{Discrete Glosten model: Setup}
	\begin{itemize}
		\item \textbf{Asset}: Continue with single asset with value $v \sim G$
		\item \textbf{Market order $x$} correlated with $v$ (reminder: notation different from the book)
		\begin{itemize}
			\item Unconditional c.d.f. $F(x)$
			\item Again, focus on the ask side of the book, $x>0$
		\end{itemize}
		\item \textbf{Discrete price grid}
		\begin{itemize}
			\item $A_1$ is lowest price tick above $\mu$
			\item $A_k-A_{k-1}>0$ is the tick size
		\end{itemize}
		\item \textbf{Limit orders}
		\begin{itemize}
			\item \textit{Time priority}: first posted, first executed
			\item \textit{Display cost}: $C$ per unit (paid regardless of whether order executes)
			% This display cost is here for the model to work even without adverse selection
			%\item Let $q_k$ denote the amount supplied at price $A_k$
			\item Let $q_k$ denote the \emph{cumulative} volume supplied (depth) at prices up to $A_k$
			% Y_k(A_k) is the inverse of P(q)
		\end{itemize}
	\end{itemize}
\end{frame}


\begin{frame}{Discrete model: Equilibrium}
	%Let $\mathbb{E}[v|q \geq Y_k]=\mathbb{E}[v|\theta_i(q) \geq P'(q)]$.
	%Let $\mathbb{E}[v|q \geq Y_k]=\mathbb{E}[v|\theta_i(q) \geq A_k]$.
	\begin{itemize}
		\item \textbf{Competition}: Limit orders are supplied at each tick until the \structure{last order earns zero profit}
		\item \textbf{Zero-profit condition}:
		\[
		\mathbb{P}(x \geq q_k) \cdot \big[ A_k-\mathbb{E}[v|x \geq q_k] \big] - C = 0,
		\]
		\pause
		solved by
		\begin{block}{}
			\[
			A_k = \underbrace{\mathbb{E}[v| x \geq q_k]}_{\text{\alert{Adverse selection}}} + \underbrace{\frac{C}{\mathbb{P}(x \geq q_k)}}_{\text{\alert{Execution risk}}}
			\]
		\end{block}
	\end{itemize}
	(Though we actually want to solve for endogenous depth $q_k$ given exogenous price ticks $A_k$)
\end{frame}


\begin{frame}{Example 1: Setup} 
	\begin{itemize}
		\item \textbf{Asset}. Let $g$ be the marginal distribution of $G$ and
		\[
		g(v)=\left\{ \begin{aligned}
		1/2  & \text{ if } v=v^H; \\
		1/2 & \text{ if } v=v^L,
		\end{aligned}
		\right.
		\]
		with $v^{H}=\mu + \sigma $ and $v^{L}= \mu - \sigma $.
		\item \textbf{Traders}. Single trader, who uses a market order.
		\begin{itemize}
			\item Prob. $\pi$: risk-neutral speculator (\structure{S}) who knows $v$
			\item Prob $1-\pi$: noise trader (\structure{N})  who buys/sells with equal probability, and uses large ($x_L$) or small ($x_S<x_L$) order with equal probability:
			$$
			\mathbb{P}(x=x_S|\structure{N})= \mathbb{P}(x=x_L|\structure{N})=\mathbb{P}(x=-x_S|\structure{N})=\mathbb{P}(x=-x_L|\structure{N})=1/4
			$$
		\end{itemize}
		\item \textbf{No display cost}. Let $C=0$
		\item \textbf{Continuous prices}.
	\end{itemize}
\end{frame}


\begin{frame}{Discrete Glosten model: comments}
	\begin{itemize}
		\item The pricing rule seems to be exactly the same as in the continuous model...
		\item ...but this is only for the marginal units! 
	\end{itemize}
	\bigskip 
	Either way, let's now look at a few examples to practice applying this pricing rule!
\end{frame}


\begin{frame}{Example 1: Equilibrium}
	\begin{itemize}
		\item \textbf{Equilibrium}: Look for eq. with $q_{1}=x_{S}$ and $q_{2}=x_{L}$ for some ask prices $v^L<A_1<A_2<v^H$. In this prb: $q_1,q_2$ given, we look for \structure{$A_1,A_2$}.
		\pause
		\item \textbf{Speculator}: 
		\begin{itemize}
			\item If $x \notin \{x_S,x_L\}$, speculator reveals himself -- never optimal
			\item Since $v^H > A_1,A_2 > v^L$, if $v=v^{H}$ then speculator buys $x_L$ units; if $v=v^L$ then sells $x_L$. \only<2>{\\ \alert{Why is it not optimal to shade the order?}}
			\pause
			\item (Shading order (strategically restricting trade size) is not optimal because buying more does not worsen the price of the previous units, unlike in dealer mkt)
		\end{itemize}
		\pause 
		\item \textbf{Price}. In equilibrium, price must equal $\mathbb{E}[v|x\geq q_k]$:
		\begin{align*}
			A_1 & =\mathbb{E}[v|x \geq x_{S}] = \mu + \pi \sigma,
			\\
			A_2 & = \mathbb{E}[v|x \geq x_{L}] = \mu + \frac{2\pi}{1+\pi} \sigma
		\end{align*}
		\item Obvious that $v^L<A_1<A_2<v^H$. Thus: equilibrium.
	\end{itemize}
\end{frame}


\begin{frame}{Example 1: Comment}
	\begin{itemize}
		\item Not the best example for discreteness:
		\begin{itemize}
			\item The example does \emph{not} assume fixed ticks...
			\item But they arise endogenously in equilibrium...
			\item Due to discreteness of noise traders' strategy
			\item (A very artificial assumption)
		\end{itemize}
		\pause
		\item But focus on adverse selection leading to limited depth:
		\begin{itemize}
			\item Price increases in order size
			\item Not due to informed trader having stronger info, as in Kyle model
			\item But due to noise traders' order becoming (relatively) less likely
		\end{itemize}
	\end{itemize}
\end{frame}


\begin{frame}{Example 2: Setup}
	\begin{itemize}
		\item \textbf{Market orders}: Exponential distribution, $f(x)=\frac{\theta}{2} e^{-\theta|x|}$
		\item \textbf{Asset}. \underline{Assume} `price impact' equation $\mathbb{E}[v|x] = \mu + \lambda x$, where $\lambda >0$ is a constant measuring informativeness of order flow
		\begin{itemize}
			\item Thus, we are taking a short-cut and modeling adverse selection in a `reduced form': rather than modeling the informed traders, we model their price impact
		\end{itemize}
		\item There is some order submission cost $C$
		\item \textbf{Goal}: find the equation connecting $A_k$ and $q_k$ (given arbitrary tick grid $A_1,\dots,A_k,\dots$)
	\end{itemize}
\end{frame}


\begin{frame}{Example 2: Equilibrium}
	\begin{itemize}
		\item Focus on \textbf{ask side}: $q_k>0$. For $x \geq q_k$:
		\begin{align*}
			f(x|x \geq q_k)&=\frac{f(x)}{ \mathbb{P}(x \geq q_k)} \\
			&= \frac{f(x)}{\int^\infty_{q_k} f(x) dx}\\
			&= \frac{\frac{\theta}{2} \cdot e^{-\theta x}}{e^{-\theta q_k}/2} \quad \quad \text{ ($|x|=x$ since $x \geq q_k>0$)}\\
			& =\theta \cdot e^{-\theta(x-q_k)}\\
			& = e^{\theta q_k} \left[ \theta \cdot e^{-\theta x} \right]
		\end{align*}
	\end{itemize}
\end{frame}


\begin{frame}{Example 2: Equilibrium (2)}
	\begin{itemize}
		\item The expected value at tick $k$ becomes
		\begin{align*}
		\mathbb{E}[v|x \geq q_k] & =  \mu + \lambda \mathbb{E}[x|x \geq q_k]\\
		&=\mu+ \lambda \int^\infty_{q_k} x \cdot f(x|x \geq q_k) dx\\
		&=\mu+ \lambda e^{\theta q_k} \int^\infty_{q_k} x \cdot \theta \cdot e^{-\theta x} dx\\
		& =  \mu+ \lambda e^{\theta q_k} \left\{\left[- x \cdot e^{-\theta x} \right]^\infty_{q_k} -\int^\infty_{q_k} -e^{-\theta x} dx   \right\}\text{ (int. by parts)} \\
		& =  \mu+ \lambda e^{\theta q_k} \left\{\left[0+ q_k \cdot  e^{-\theta q_k} \right] -\frac{1}{\theta}[0-e^{-\theta q_k}]   \right\} \\
		& = \mu + \lambda \left( \frac{1}{\theta }+ q_k \right)
		\end{align*}
	\end{itemize}
\end{frame}


\begin{frame}{Example 2: Equilibrium (3)}
	\begin{itemize}
		\item Hence, the ask price at tick $k$ can be found by solving
		\[
		\underbrace{\structure{\mathbb{P}(x \geq q_k)}}_{e^{-\theta q_k}/2}[A_k-\underbrace{\structure{\mathbb{E}[v|x \geq q_k]}}_{\mu + \lambda \left( \frac{1}{\theta }+ q_k \right)}] - C = 0,
		\]
		which gives
		\begin{block}{}
			\[
			A_k=\mu + \lambda \left( \frac{1}{\theta }+ q_k\right) + \frac{2C}{e^{-\theta q_k}}.
			\]
		\end{block}
		\item (Again: we actually need the opposite -- find $q_k$ for a given tick $A_k$ -- but it is hard to get a closed-form expression for that.)
	\end{itemize}
\end{frame}


\begin{frame}{Glosten: Empirical evidence}
	\begin{itemize}
		\item \textbf{\cite{sandas_adverse_2001}} estimates Glosten model (in a form similar to example 2 above) via GMM, using intraday snapshots of LOB from Stockholm Stock Exchange and data on market orders 
		\item Estimates the info content of market orders vs actual pricing schedules, so effectively the $\mathbb{E}[v|x \geq q]$ inferred from pricing schedule and the actual $\mathbb{E}[v|x = q]$ from the price dynamics.
		\item Zero profit condition is tested and rejected: LOB not deep enough to drive average expected profits to zero
		\item Also, estimated order execution costs are negative for the best bid and ask -- i.e., those limit traders have some intrinsic preference for trading (although these days many exchanges do offer negative execution fees to limit traders to incentivize liq-ty provision)
	\end{itemize}
\end{frame}


\begin{frame}{Glosten model: Conclusion}
\begin{itemize}
	\item Limit traders act in the same capacity as the dealer did before
	\begin{itemize}
		\item but face different \alert{informational environment}
		\item so act differently
		\item which leads to different market outcomes
	\end{itemize}
	\item With discrete ticks and time priority, even competitive limit traders can get positive expected profits
\end{itemize}
\end{frame}




\section{Market design}

\begin{frame}{Market design}
	\begin{itemize}
		\item There are many dimensions in which legislation or exhange rules can regulate trade
		\item Today's phrase of the day: ``\structure{unintended consequences}''
		\begin{itemize}
			\item Attempts to mitigate a particular inefficiency may have far-fetching consequences
			\item We will look at a few examples
		\end{itemize}
	\end{itemize}
\end{frame}


\begin{frame}{Tick size}
	\begin{itemize}
		\item Assume \structure{time priority} is the second order after price priority
		\begin{itemize}
			\item I.e., first limit order posted at tick executes first
		\end{itemize}
		\item Profit of the limit trader at price $A_k$ is:
		\begin{itemize}
			\item Zero for the marginal (last) limit order at $A_k$
			\item Strictly positive for inframarginal orders (if $C>0$) because order executes with higher probability
		\end{itemize}
		\item Q: what happens if we \alert{change the tick size}?
		\pause
		\visible<handout:0|1->{
			\item This profit is reduced with \alert{smaller tick sizes}
			\begin{itemize}
				\item Hence decreasing tick size drives away limit traders and reduces depth
				\item But it will also reduce spread (by design) and reduce trading costs for the opposite side of the market (liquidity demanders)
			\end{itemize}
		}
	\end{itemize}
\end{frame}


\begin{frame}{Tick size}
	\begin{itemize}
		\item Driving away limit traders intuitively also has dynamic repercussions
		\visible<handout:0>{
			\begin{itemize}
				\item LOB is replenished more slowly after trades -- so market orders traded more frequently against non-competitive prices
			\end{itemize}
		}
		\pause
		\item \cite{goldstein_eighths_2000} explored the NYSE 1997 case (tick size from $\$1/8$ to $\$1/16$)
		\begin{itemize}
			\item Trading costs decreased for small orders
			\item Unclear for large orders
			\item Aligns with our predictions (smaller spread, smaller depth)
		\end{itemize}
	\end{itemize}
\end{frame}


\begin{frame}{Priority rules}
	\begin{itemize}
		\item With \structure{pro-rata} allocation (limit orders at given tick executed proportionally to their size), as opposed to \alert{time priority}:
		\begin{itemize}
			\item The expected profit of all orders at price $A_k$ must be zero (as opposed to strictly positive)
			\item So execution probabilities must be lower for all orders
		\end{itemize}
		\item Lower profits lead to same dynamic consequences as with reducing tick size
		\item Pro-rata allocation rule used in the electronic futures markets for the leading short-term interest rate and for the two-year U.S. Treasuries.
	\end{itemize}
\end{frame}


\begin{frame}{Hybrid market}
	Hybrid market
	\begin{itemize}
		\item Suppose a dealer can compete with the limit order book, as follows
		\item The dealer may observe trade size $x$ before serving the order (and can fulfill the order before it is matched against LOB)
		\item Can profit by pricing at $\mathbb{E}[v|x=q]$ rather than an average of $\mathbb{E}[v|x \geq y]$ for $y \leq q$ which is used by the LOB
		\item Especially profitable on small trades
		\item But the existence of such a dealer invalidates our analysis of the LOB
		\begin{itemize}
			\item Profitable limit orders are being picked off
			\item So limit traders would gain negative profits if they follow the old strategy $\Rightarrow$ incentive to change their strategy (or quit the market)
		\end{itemize}
	\end{itemize}
\end{frame}


\begin{frame}{Example 1 with hybrid market}
	\begin{itemize}
		\item Example 1 from before continued. Assume an uninformed dealer receives order $x$. Can either send order to LOB or execute himself (at better price) 
		\pause
		\item Focus on ask side. Let $A^H_k$ be the hybrid ask price. When dealer observes $x=x_S$, he knows trader is noise trader and thus $\mathbb{E}[v|\structure{N}]=\mu$. 
		\begin{itemize}
			\item Can execute order at just below $A^H_1$ and earn profit $A^H_1-\mu$.
		\end{itemize}
		\pause
		\item Hence, only large orders $x=x_L$ are sent to LOB. LOB traders will expect this, and will price as if any order arriving to the LOB is large:
		\begin{align*}
			\mathbb{E}[v|x \ge x_S]=\mathbb{E}[v|x \ge x_L]=\mu+\frac{2\pi}{1+\pi} \sigma,
		\end{align*}
		and thus $A^H_1=A^H_2=\mu+\frac{2\pi}{1+\pi} \sigma$.
		\item $A^H_1>A_1$ and $A^H_2=A_2$: hybrid market less liquid than normal market
	\end{itemize}
\end{frame}


\begin{frame}{Hybrid market: conclusions}
	\begin{itemize}
		\item To be fair: adding a dealer to LOB market... 
		\begin{itemize}
			\item decreases liquidity in ``good times'', when there would've been a thick LOB
			\item but can help in bad times: if LOB is empty then adding a dealer has no adverse effects and will actually increase liquidity
			\item So in the end, adding a dealer is like a \structure{liquidity insurance} for the market
		\end{itemize}
		\item More analysis of hybrid markets with risk-averse traders: see \cite{viswanathan_market_2002}
		\item Also: the analysis of the example above relied on a bunch of implicit assumptions (which are not necessarily true):
		\begin{itemize}
			\item Assumed the dealer had time priority over (could undercut all of) the LOB. If MO-traders can trade against the LOB before the dealer can act, the conclusions are different.
			\item Assumed the market order revealed enough information. If MO-traders split their orders (trade one unit at a time), dealers no longer have any advantage. (\cite{back_working_2007})
		\end{itemize}
	\end{itemize}
\end{frame}


\begin{frame}{Market Design: conclusion}
	\begin{itemize}
		\item Regulation aimed at improving market liquidity can backfire by distorting agents' incentives
		\item Shameless plug:
		\begin{itemize}
			\item If you want to know more about market design, I teach a course on \structure{Mechanism Design} in Fall
			\item Explore how to design environments in which agents operate (i.e., distort their incentives) in order to generate desired behavior
		\end{itemize}
	\end{itemize}
\end{frame}




\begin{frame}{Next week}
	\begin{itemize}
		\item Dynamic LOB analysis: traders can choose between limit and market orders
	\end{itemize}
\end{frame}


\begin{frame}{Homework}
	\begin{itemize}
		\item Thinking in the framework of the discrete model: suppose tick size is actually zero; quotes can be placed in a continuous price space. Suppose that there is price priority. What then is the role of time priority, so that first-come quotes at identical prices are served first?
		% ex1 is on informed trading
		\item Solve exercise 1 after ch.6 (pages 232-233) in the textbook. Note that you need to use Bayes' rule to assess the conditional distribution over $v$ given a market order of size $x$ (and work through slightly different notation)
	\end{itemize}
\end{frame}





\appendix
\begin{frame}<handout:0>[allowframebreaks]{References}
	\bibliography{../teaching}
	\bibliographystyle{abbrvnat}
\end{frame}




\end{document} 