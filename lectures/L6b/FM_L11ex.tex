%%% License: Creative Commons Attribution Share Alike 4.0 (see https://creativecommons.org/licenses/by-sa/4.0/)
%%% Slides are based heavily on earlier versions of this course taught by Jesper Rudiger.

\documentclass[english,10pt
%,handout
,aspectratio=169
]{beamer}
%%% License: Creative Commons Attribution Share Alike 4.0 (see https://creativecommons.org/licenses/by-sa/4.0/)
%%% Slides are based heavily on earlier versions of this course taught by Jesper Rudiger and Peter Norman Sorensen.

\DeclareGraphicsExtensions{.eps, .pdf,.png,.jpg,.mps,}
\usetheme{reMedian}
\usepackage{parskip}
\makeatother

\renewcommand{\baselinestretch}{1.1} 

\usepackage{amsmath, amssymb, amsfonts, amsthm}
\usepackage{enumerate}
\usepackage{hyperref}
\usepackage{url}
\usepackage{bbm}
\usepackage{color}

\usepackage{tikz}
\usepackage{tikzscale}
\newcommand*\circled[1]{\tikz[baseline=(char.base)]{
		\node[shape=circle,draw, inner sep=-20pt] (char) {#1};}}
\usetikzlibrary{automata,positioning}
\usetikzlibrary{decorations.pathreplacing}
\usepackage{pgfplots}
\usepgfplotslibrary{fillbetween}
\usepackage{graphicx}

\usepackage{setspace}
%\thinmuskip=1mu
%\medmuskip=1mu 
%\thickmuskip=1mu 


\usecolortheme{default}
\usepackage{verbatim}
\usepackage[normalem]{ulem}

\usepackage{apptools}
\AtAppendix{
	\setbeamertemplate{frame numbering}[none]
}
\usepackage{natbib}




\title{Financial Markets Microstructure \\ Exercise class 3}

\author{Egor Starkov}

\date{K{\o}benhavns Unversitet \\
	Spring 2020}



\begin{document}


\subsection{ex 6.1}

\begin{frame}[label=ex1]{Exercise 6.1}
	This exercise considers a two-period model of an LOB.
	\begin{itemize}
		\item \textbf{Security}: Value $v \sim g(v)=\frac{1}{2\sigma} e^{-\frac{|v-\mu|}{\sigma}}$
		\item \textbf{Trader}: 
		\begin{itemize}
			\item Type \textcolor{blue}{$I$}. With prob. $\pi$, knows $v$ and maximizes exp. utility
			\item Type \textcolor{blue}{$N$}. With prob. $1-\pi$, buys/sells (with equal prob.) $x$ shares, with $x \sim f(x)=\frac{1}{2}\theta e^{-\theta |x|}$ 
		\end{itemize}
		\item \textbf{Tick}: Tick size is nil ($\Delta=0$)
	\end{itemize}
	Notice that $\mathbb{E}[v|v \ge z]=z+\sigma$.
\end{frame}


\begin{frame}{Exercise 6.1.a}
	\begin{exampleblock}{}
		(a) Let $Y(A)$ be the cumulative depth up to ask price $A$ in the book and $A^*$ be the lowest ask price in the LOB. Show that when $v \ge A^*$, the optimal strategy of the informed trader is to buy $Y(v)$ shares.
	\end{exampleblock}

	\pause

	\begin{itemize}
		\item Let $P'(q) \equiv \{A: q=Y(A)\}$ be marginal price.
		
		Notice $P'(q) = Y^{-1}(q)$.
		\item The trader's profit from buying $q$ units is $\int_0^q (v-P'(y)) \, dy.$
		\item Take the FOC wrt. $q$
		\[
		v-P'(q)=0 \Leftrightarrow P'(q)=v.
		\]
		\item Denote the optimal strategy by $q^*(v)$ and use the definition of $P'(q)$:
		\[
		P'(q^*)=v \Leftrightarrow Y(P'(q^*))=Y(v) \Leftrightarrow Y(Y^{-1}(q^*))=Y(v) \Leftrightarrow q^*= Y(v).
		\]
	\end{itemize}
\end{frame}


\begin{frame}{Exercise 6.1.b}
	\begin{exampleblock}{}
		(b) Using part (a) and the zero-profit condition show that in eqm:
		\begin{center}
			$
			Y(A)=\frac{1}{\theta} \left[\ln \left(\frac{1-\pi}{\pi}\right) + \ln \left(\frac{A-\mu}{\sigma}\right) + \frac{A-\mu}{\sigma}\right]\text{ if } A>A^*.
			$
		\end{center}
	\end{exampleblock}

	\pause

	\begin{itemize}
		\item Part (a) implies that $v=P'(q^*)$. Zero-profit condition is $P'(q) = \mathbb{E} [v | q^*(v) \geq q]$.
		
		\item Use the latter to write the marginal price:
		\[
			P'(q) = \alpha(q) \mathbb{E}[v|v \ge P'(q)]+(1-\alpha(q))\mathbb{E}[v],
		\]
		where 
		\begin{align*}
			\alpha(q) &\equiv \mathbb{P}(I|\text{buy} \geq q) = \frac{\mathbb{P}(I)\mathbb{P}(\text{buy} \geq q|I) }{\mathbb{P}(\text{buy} \geq q)} 
			\\
			&= \frac{\pi \mathbb{P}(v \ge P'(q))}{\pi \mathbb{P}(v \ge P'(q))+(1-\pi)\mathbb{P}(x \ge q)}
		\end{align*}.
	\end{itemize}
\end{frame}


\begin{frame}{Exercise 6.1.b (2)}
	\begin{itemize}
		\item Compute the ingredients of $\alpha(q)$:
		\begin{align*}
			\mathbb{P}(v \ge P'(q)) 	&=\int_{P'(q)}^\infty g(v) dv = \frac{1}{2} e^{-\frac{P'(q)-\mu}{\sigma}} \\
			\mathbb{P}(x \ge q)	&=\int_q^\infty f(x) dx = \frac{1}{2}e^{-\theta q}
		\end{align*}
		\item Plug them back into $\alpha(q)$:
		\begin{align*}
			\alpha(q)
			& =\frac{\pi \frac{1}{2} e^{-\frac{P'(q)-\mu}{\sigma}}}{\pi \frac{1}{2} e^{-\frac{P'(q)-\mu}{\sigma}} + (1-\pi)\frac{1}{2}e^{-\theta q}} 
			\\
			& = \frac{e^{\theta q-\frac{P'(q)-\mu}{\sigma}}}{e^{\theta q-\frac{P'(q)-\mu }{\sigma}} + \frac{1-\pi}{\pi}} 
		\end{align*}
	\end{itemize}
\end{frame}


\begin{frame}{Exercise 6.1.b (3)}
	\begin{itemize}
		\item Going back to the marginal price, we had
		\begin{align*}
			P'(q) &= \alpha(q) \mathbb{E}[v|v \ge P'(q)]+(1-\alpha(q))\mathbb{E}[v]
		\end{align*}
		\item Recall hint: $\mathbb{E}[v|v \ge P'(q)]=P'(q)+\sigma$. Plug this together with $\alpha(q)$ into the expression above:
		\begin{align*}
			&P'(q) = \alpha(q) (P'(q)+\sigma) + (1-\alpha(q)) \mu 
			\\
			& \Leftrightarrow \frac{1-\pi}{\pi} \cdot \frac{P'(q)-\mu}{\sigma} = e^{\theta q - \frac{P'(q)-\mu}{\sigma}}  \\
			& \Leftrightarrow \ln \left( \frac{1-\pi}{\pi} \right) + \ln \left( \frac{P'(q)-\mu}{\sigma} \right) = \theta q - \frac{P'(q)-\mu}{\sigma} 
		\end{align*}
		Finally, substitute $Y(A)=q(A)$ and $A=P'(q)$ and re-arrange to get result.
	\end{itemize}
\end{frame}


\begin{frame}{Exercise 6.1.c}
	\begin{exampleblock}{}
		(c) Show that the book becomes thinner on the ask side when (i) $\pi$ increases or (ii) $\sigma$ increases. What is the economic intuition for this result.
	\end{exampleblock}

	\pause

	$$
		Y(A)=\frac{1}{\theta} \left[\ln \left(\frac{1-\pi}{\pi}\right) + \ln \left(\frac{A-\mu}{\sigma}\right) + \frac{A-\mu}{\sigma}\right]\text{ if } A>A^*.
	$$
	\begin{itemize}
		\item \textbf{Informed trading}. 
		\[
		\frac{\partial Y(A)}{\partial \pi} = \frac{1}{\theta}\left[ -\frac{\pi}{1-\pi}\cdot \frac{1}{\pi^2}\right]=\frac{1}{\theta}\left[ -\frac{1}{\pi(1-\pi)}\right]<0
		\]
		More informed trading implies greater risk when posting orders
	\end{itemize}
\end{frame}


\begin{frame}{Exercise 6.1.c}
	\begin{exampleblock}{}
		(c) Show that the book becomes thinner on the ask side when (i) $\pi$ increases or (ii) $\sigma$ increases. What is the economic intuition for this result.
	\end{exampleblock}

	$$
		Y(A)=\frac{1}{\theta} \left[\ln \left(\frac{1-\pi}{\pi}\right) + \ln \left(\frac{A-\mu}{\sigma}\right) + \frac{A-\mu}{\sigma}\right]\text{ if } A>A^*.
	$$
	
	\begin{itemize}
		\item \textbf{Value uncertainty}. 
		\[
		\frac{\partial Y(A)}{\partial \sigma} = \frac{1}{\theta} \left[ -\frac{\sigma}{A-\mu}\frac{A-\mu}{\sigma^2} - \frac{A-\mu}{\sigma^2}\right]=\frac{1}{\theta} \left[ -\frac{1}{\sigma} - \frac{A-\mu}{\sigma^2}\right]<0
		\]
		More value uncertainty implies greater risk when posting orders
	\end{itemize}
\end{frame}



%\subsection{ex 6.5}
%
%\begin{frame}[label=ex5]{Exercise 6.5}
%	\begin{itemize}
%		\item Consider the LOB model of section 6.4.1 with $\sigma=0$ and $\tau=1$
%		\item Thus, there is no fundamental value uncertainty, and the asset is `infinitely lived'
%		\item Prior on fundamental value is $\mu_0$
%		\item The agents have private value $y \in \{-L,L\}$ with equal probability
%		\item Suppose limit orders last one period
%		\item Let $f_{mo}$ be the fee per share for market orders, and $f_{lo}$ the fee per share for limit orders (when the order executes). Let $f=f_{mo}+f_{lo}$
%		\item Tension: either trade now with market order, or post limit order and hope that it is hit by next trader
%		\item Look for stationary equilibrium: prices $A$ and $B$ are the same each period
%	\end{itemize}
%\end{frame}
%
%
%\begin{frame}{Exercise 6.5.a}
%	\textbf{Part (a)}. Compute bid and ask quotes in equilibrium
%	\smallskip
%	\smallskip
%	\begin{itemize}
%		\item Use the approach in chapter 6.4.2. Set prices such that limit orders are always executed if the next trader has the right type.
%		\item Suppose we set a price $A$ such that next trader will use a buy market order if he is type $L$. Given $B$, the highest $A$ that will satisfy this is
%		\begin{equation}\label{eq1}
%		\frac{1}{2}[\mu_0+L-B-f_{lo}]=\mu_0+L-A-f_{mo}.
%		\end{equation}
%		\item Similarly, given $A$, the lowest $B$ such that next trader will use a sell market order if he is type $-L$ is
%		\begin{equation} \label{eq2}
%		B-(\mu_0-L)-f_{mo}=\frac{1}{2}[A-(\mu_0-L)-f_{lo}]
%		\end{equation}
%	\end{itemize}
%\end{frame}
%
%
%\begin{frame}{Exercise 6.5.a (2)}
%	\begin{itemize}
%		\item Add 2 times \eqref{eq1} to \eqref{eq2} to get
%		\begin{align*}
%		A=\mu_0+\frac{L}{3}+\frac{f_{lo}-2f_{mo}}{3}\\
%		B=\mu_0-\frac{L}{3}-\frac{f_{lo}-2f_{mo}}{3}
%		\end{align*}
%		\item This gives a spread of
%		\[
%		S=\frac{2L}{3}+2\frac{f_{lo}-2f_{mo}}{3}.
%		\]
%	\end{itemize}
%\end{frame}
%
%
%\begin{frame}{Exercise 6.5.b}
%	\textbf{Part (b)}. Show that the bid-ask spread decreases in $f_{mo}$ and increases in $f_{lo}$
%	\smallskip
%	\smallskip
%	\begin{itemize}
%		\item Recall spread
%		\[
%		S=\frac{2L}{3}+2\frac{f_{lo}-2f_{mo}}{3}.
%		\]
%		\item Increasing in $f_{lo}$: when it is less attractive to post limit orders, bid (ask) price must be lower (higher) to attract limit order posters
%		\item Decreasing in $f_{mo}$: when it is less attractive to post market orders, bid (ask) price must be higher (lower) to attract market order posters
%	\end{itemize}
%\end{frame}
%
%
%\begin{frame}{Exercise 6.5.c}
%	\textbf{Part (c)}. Trading platforms often subsidize traders who submit limit orders. That is, they set $f_{lo}<0$ and $f_{mo}>0$, maintaining that this practice ultimately helps to narrow the spread and benefits traders submitting market orders. Holding the total trading fee fixed, is this argument correct?
%	\smallskip
%	\smallskip
%	\begin{itemize}
%		\item Suppose $f=f_{lo}+f_{mo}$. Then we can rewrite spread as
%		\[
%		S=\frac{2L}{3}-\frac{2f}{3}+2f_{lo}.
%		\]
%		\item This is increasing in $f_{lo}$, so setting $f_{lo}<0$ would be optimal
%	\end{itemize}
%\end{frame}
%
%
%\subsection{continuous tick}
%
%\begin{frame}{Tick size and time priority}
%	\begin{exampleblock}{}
%		Suppose that there is no tick; quotes can  be placed in a continuous price space. Suppose that there is price priority. What is then the role of time priority, so that first-come quotes at identical prices are served first?
%	\end{exampleblock}
%	
%	\centering
%	\emph{``...The size of the trading increment is a
%	powerful level that is underappreciated by
%	regulators. The \alert{finer the trading increment},
%	\structure{the more important price priority becomes
%	relative to time priority}. In other words, if the
%	market was designed to trade at continuous
%	(vs discrete) non‐penny increments you could
%	always win a trade by quoting the best price
%	and the \structure{speed game would be non existent}''}
%	
%	\begin{flushright}
%		--Narang, Manoj ``HFT 101 with Tradeworx'',
%		May 20, 2014
%	\end{flushright}
%\end{frame}
%
%
%\begin{frame}{Tick size and time priority}
%	\begin{itemize}
%		\item The role of the tick is essentially \textit{to balance price priority and time priority}
%		\item Larger tick gives more time priority, smaller tick gives more price priority
%		\item When there is no tick, there is essentially no time priority...
%		\item However, there is some evidence that ticks may arise \textit{endogenously}. In that case, time priority takes importance again
%	\end{itemize}
%\end{frame}





\end{document} 