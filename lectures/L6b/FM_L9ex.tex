%%% License: Creative Commons Attribution Share Alike 4.0 (see https://creativecommons.org/licenses/by-sa/4.0/)
%%% Slides are based heavily on earlier versions of this course taught by Jesper Rudiger.

\documentclass[english,10pt
,handout
,aspectratio=169
]{beamer}
%%% License: Creative Commons Attribution Share Alike 4.0 (see https://creativecommons.org/licenses/by-sa/4.0/)
%%% Slides are based heavily on earlier versions of this course taught by Jesper Rudiger and Peter Norman Sorensen.

\DeclareGraphicsExtensions{.eps, .pdf,.png,.jpg,.mps,}
\usetheme{reMedian}
\usepackage{parskip}
\makeatother

\renewcommand{\baselinestretch}{1.1} 

\usepackage{amsmath, amssymb, amsfonts, amsthm}
\usepackage{enumerate}
\usepackage{hyperref}
\usepackage{url}
\usepackage{bbm}
\usepackage{color}

\usepackage{tikz}
\usepackage{tikzscale}
\newcommand*\circled[1]{\tikz[baseline=(char.base)]{
		\node[shape=circle,draw, inner sep=-20pt] (char) {#1};}}
\usetikzlibrary{automata,positioning}
\usetikzlibrary{decorations.pathreplacing}
\usepackage{pgfplots}
\usepgfplotslibrary{fillbetween}
\usepackage{graphicx}

\usepackage{setspace}
%\thinmuskip=1mu
%\medmuskip=1mu 
%\thickmuskip=1mu 


\usecolortheme{default}
\usepackage{verbatim}
\usepackage[normalem]{ulem}

\usepackage{apptools}
\AtAppendix{
	\setbeamertemplate{frame numbering}[none]
}
\usepackage{natbib}





\begin{document}
\section{Kyle model}

\begin{frame}{Lecture 9}
	\begin{itemize}
		\item solve exercise 3 in chapter 4 (p.159) which explores competition among speculators
	\end{itemize}
\end{frame}


\begin{frame}{Exercise 4.3}
	\begin{itemize}
		\item Consider a Kyle model with many speculators $i \in \{1,2,...,N\}$.
		\item Every speculator uses a linear strategy
		$$ x_i = \beta (v-\mu) $$
		\item Everything else is same as in lecture
	\end{itemize}
\end{frame}


\begin{frame}{Exercise 4.3.a}
	\begin{exampleblock}{}
		(a) Find the equilibrium aggressiveness $\beta$, determine how it depends on $N$ and explain why.
	\end{exampleblock}
	
	\pause
	
	Speculator $i$ maximizes the expected profit $\mathbb{E} [x_i (v-p)]$.
	
	The clearing price is given by the zero-profit condition:
	\begin{align*}
		p = \mu + \lambda [x_i + (N-1) \beta (v-\mu) + u]
	\end{align*}
	Plugging it into the expected profit expression and maximizing it w.r.t. $x_i$ yields
	\begin{align*}
		x_i &= \underbrace{\frac{1 - \lambda \beta (N-1)}{2 \lambda}}_{=\beta} (v-\mu)
		\\
		\Rightarrow
		\beta &= \frac{1}{\lambda (N+1)}
	\end{align*}
\end{frame}


\begin{frame}{Exercise 4.3.a}
	\begin{align*}
	\beta &= \frac{1}{\lambda (N+1)}
	\end{align*}
	
	\begin{itemize}
		\item $\beta$ decreasing in $N$: the more traders, the smaller is one trader's share (as in Cournot oligopoly)
		\item $N \beta$ increasing in $N$: total order size increases in $N$ because with high $N$, each trader suffers less from higher price (due to lower output)
	\end{itemize}
\end{frame}


\begin{frame}{Exercise 4.3.b}
	\begin{exampleblock}{}
		(b) Derive the price impact coefficient $\lambda$ from the dealer's zero-profit condition
	\end{exampleblock}

	\pause
	
	\begin{align*}
		\lambda = \frac{\mathbb{C}(v,q)}{\mathbb{V}(q)} = \frac{N \beta \sigma^2_v}{(N \beta)^2 \sigma^2_v + \sigma^2_u}	
		% \frac{ \frac{N}{\lambda (N+1)} \sigma^2_v }{ \left( \frac{N}{\lambda (N+1)} \right)^2 \sigma^2_v + \sigma^2_u }
	\end{align*}
	Combining with the expression for $\beta$, we get
	\begin{align*}
		\lambda &= \frac{\sqrt{N}}{N+1} \frac{\sigma_v}{\sigma_u} 
		&
		\beta = \frac{1}{\sqrt{N}} \frac{\sigma_u}{\sigma_v}
	\end{align*}
\end{frame}


\begin{frame}{Exercise 4.3.c}
	\begin{exampleblock}{}
		(c) What is the market depth in equilibrium, and how is it affected by an increase in the number of informed traders, $N$? What is the economic intuition for this result?
		Do you think this result is robust?
	\end{exampleblock}
	
	\pause
	
	\begin{itemize}
		\item Depth is $\frac{1}{\lambda} = \frac{N+1}{\sqrt{N}} \frac{\sigma^2_u}{\sigma^2_v} $
		\item It is increasing in $N$ because traders become more aggressive (relative to their market share) in the presence of competition -- submit larger orders $|x_i|$ given any fixed $v$, -- price reacts less to any fixed order size.
		\item The conclusion is not 100\% robust...
	\end{itemize}
\end{frame}


%\begin{frame}{Exercise 4.3.c}
%	%TODO: rip the pic out of the textbook and elaborate on it -- Fig 4.2? or which one?
%\end{frame}


\begin{frame}{Exercise 4.3.d}
	\begin{exampleblock}{}
		(d) Compute the ex-ante expected profit of each informed investor. What is the effect of an increase in $N$ on the aggregate profit of informed investors?
	\end{exampleblock}

	\pause
	
	\begin{align*}
		\mathbb{E} [x_i(v-p)] = \frac{\sigma_u \sigma_v}{(N+1) \sqrt{N}}
	\end{align*}
	\begin{itemize}
		\item predictably, the profits of each speculator decrease in $N$
		\item aggregate profit of all speculators $\frac{\sigma_u \sigma_v \sqrt{N}}{N+1}$ also decreases in $N$ (because total aggressiveness $N \beta$ increases away from the monopolistic optimum)
	\end{itemize}
\end{frame}


\end{document} 