%%% License: Creative Commons Attribution Share Alike 4.0 (see https://creativecommons.org/licenses/by-sa/4.0/)
%%% Slides are based heavily on earlier versions of this course taught by Jesper Rudiger.

\documentclass[english,10pt
%,handout
,aspectratio=169
]{beamer}
%%% License: Creative Commons Attribution Share Alike 4.0 (see https://creativecommons.org/licenses/by-sa/4.0/)
%%% Slides are based heavily on earlier versions of this course taught by Jesper Rudiger and Peter Norman Sorensen.

\DeclareGraphicsExtensions{.eps, .pdf,.png,.jpg,.mps,}
\usetheme{reMedian}
\usepackage{parskip}
\makeatother

\renewcommand{\baselinestretch}{1.1} 

\usepackage{amsmath, amssymb, amsfonts, amsthm}
\usepackage{enumerate}
\usepackage{hyperref}
\usepackage{url}
\usepackage{bbm}
\usepackage{color}

\usepackage{tikz}
\usepackage{tikzscale}
\newcommand*\circled[1]{\tikz[baseline=(char.base)]{
		\node[shape=circle,draw, inner sep=-20pt] (char) {#1};}}
\usetikzlibrary{automata,positioning}
\usetikzlibrary{decorations.pathreplacing}
\usepackage{pgfplots}
\usepgfplotslibrary{fillbetween}
\usepackage{graphicx}

\usepackage{setspace}
%\thinmuskip=1mu
%\medmuskip=1mu 
%\thickmuskip=1mu 


\usecolortheme{default}
\usepackage{verbatim}
\usepackage[normalem]{ulem}

\usepackage{apptools}
\AtAppendix{
	\setbeamertemplate{frame numbering}[none]
}
\usepackage{natbib}




\title{Financial Markets Microstructure \\ Exercises}

\author{Egor Starkov}

\date{K{\o}benhavns Unversitet \\
	Spring 2025}



\begin{document}
%	\AtBeginSection[]{
%		\frame<beamer>{
%			\frametitle{This class:}
%			\tableofcontents[currentsection,currentsubsection]
%	}}
%	
%\frame[plain]{\titlepage}
%\addtocounter{framenumber}{-1}


%\begin{frame}{Plan}
%	\begin{itemize}
%		\item L7: Ch 6, ex 5 (p 235)
%		\item L8: Ch 7, ex 3 (p 276)
%		\item PS1: ex 3
%		%\item L9: 
%	\end{itemize}
%\end{frame}



%\section{Ch 6, ex 5}

\begin{frame}{Ch 6, ex 5}
	\textbf{Make/take fees and bid-ask spreads.} 
	
	Consider the Foucault (Parlour) model. Trading platforms often charge different fees for market and limit orders. 
	\begin{itemize}
		\item Let $f_{mo}$ be the fee per share paid by a market order placer and 
		\item $f_{lo}$ the fee per share for a limit order placer when the limit order \emph{executes} (there is no entry fee for limit orders).
		\item Finally let $f$ be the total fee earned by the platform on each trade, $f=f_{mo}+f_{lo}$.
	\end{itemize}
\end{frame}


\begin{frame}{Foucault model: lecture ver}
	The model we have seen \structure{in class}:
	\begin{itemize}
		\item \textbf{Exogenous prices}. Bid and ask prices exogenously given as $A>v>B$
		\item \textbf{Traders}. Arriving trader chooses btw limit or market order (one unit)
		\begin{itemize}
			\item Limit order valid one period. Choice depends on prob. of limit order being executed, i.e. `hit' by a market order from the next trader
			\item Valuation: $v+y$. $y$ is uniformly distributed on $(-Y,Y)$, unobserved and independent across traders. $v$ is known and common to all.
		\end{itemize}
		\item \textbf{Profits}. Let $P^B_M (P^S_M)$ be prob. of next-period market buy (sell) order
	\end{itemize}
\end{frame}


\begin{frame}{Foucault (Parlour) model: lecture vs book ver}
	\begin{itemize}
		\item Model from lecture differs slightly from model \alert{in the book} (Sections 6.4.1-2):
		\begin{itemize}
			\item we made $A$ and $B$ exogenous, the book derives them;
			\item we let $y_i \sim U[-Y,Y]$, the book assumes $y_i \in \{-Y,Y\}$;
			\item in 6.4.1 the textbook initially sets up a much more general model, but never actually solves it;
			\item in lecture, we had to assume that if LOB is empty on either side, it is filled by noise traders at same prices. The book's model doesn't need this
		\end{itemize}
		\item For the problem today we stick to the textbook version (Section 6.4.2)
	\end{itemize}
\end{frame}


\begin{frame}{Ch 6, ex 5, part a}
	\begin{exampleblock}{}
		(a) Compute bid and ask quotes in equilibrium
	\end{exampleblock}
	
	\only<1>{How do?}
	
	\pause
	
	How should traders behave in equilibrium?
	\begin{itemize}
		\item If $y_i = Y$ then buy
		\begin{itemize}
			\item indifferent between market buy (if available) and limit buy
		\end{itemize}
		\item If $y_i = -Y$ then sell
		\begin{itemize}
			\item indifferent between market sell (if available) and limit sell
		\end{itemize}
	\end{itemize}
\end{frame}


\begin{frame}{Ch 6, ex 5, part a}
	Consider $y_i = Y$. 
	\begin{itemize}
		\item Profit from market buy is $v+Y-A-f_{mo}$.
		\item Profit from limit buy is $(v+Y-B-f_{lo})P_M^S$.
		\item Indifference $\Rightarrow$ the two are equal. This gives a condition on $A,B$ given $v,Y,f_{mo},f_{lo},P_M^S$. 
		\item But $P_M^S$ is uncertain -- even if next trader has $y_i=-Y$, how does he choose between MS and LS?
		\begin{itemize}
			\item In equilibrium: $t+1$-trader always chooses a market order whenever possible.
			\item Idea: limit trader at $t$ can set a price that is $\epsilon$-better for $t+1$ than submitting a limit order. So anything different from the above cannot be an equilibrium.
		\end{itemize}
	\end{itemize}
\end{frame}


\begin{frame}{Ch 6, ex 5, part a}
	So the indifference condition when $y_i=Y$ is:
	\[ v+Y-A-f_{mo} = (v+Y-B-f_{lo})1/2 \]
	Same for trader with $y_i = -Y$:
	\[ B-(v-Y)-f_{mo} = (A-(v-Y)-f_{lo})1/2 \]
	Solve the two for $A,B$ to get:
	\begin{align*}
		A &= v + \frac{1}{3}(Y+f_{lo}-2f_{mo})
		\\
		B &= v - \frac{1}{3}(Y+f_{lo}-2f_{mo})
	\end{align*}
\end{frame}


\begin{frame}{Ch 6, ex 5, part b}
	\begin{exampleblock}{}
		(b) Show that the bid-ask spread decreases in $f_{mo}$ and increases in $f_{lo}$. Explain.
	\end{exampleblock}
	
	\pause
	
	\[ S = \frac{2}{3}(Y+f_{lo}-2f_{mo}) \]
	\begin{itemize}
		\item if limit orders expensive then the price improvement from LO compared to MO (=spread) must be large to offset this cost, make LO competitive with MO
		\item vice versa for $f_{mo}$
	\end{itemize}
\end{frame}


\begin{frame}{Ch 6, ex 5, part c}
	\begin{exampleblock}{}
		(c) Trading platforms often subsidize traders who submit limit orders. That is, they set $f_{lo}<0$ and $f_{mo}>0$, maintaining that this practice ultimately helps to narrow the spread and benefits traders submitting market orders. Holding the total trading fee fixed, is this argument correct?
	\end{exampleblock}
	
	\pause
	
	This \structure{does} narrow down the nominal spread, but does \alert{NOT} benefit market traders.
	
	Consider a MB order. Trader pays
	\[ A+f_{mo} = v + \frac{1}{3}(Y+f_{lo}+f_{mo}) \]
	which only depends on total $f=f_{lo}+f_{mo}$ and not on how it is split between $f_{lo}$ and $f_{mo}$. 
\end{frame}


\end{document} 