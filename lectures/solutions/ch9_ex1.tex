%%% License: Creative Commons Attribution Share Alike 4.0 (see https://creativecommons.org/licenses/by-sa/4.0/)
%%% Slides are based heavily on earlier versions of this course taught by Jesper Rudiger.

\documentclass[english,10pt
%,handout
,aspectratio=169
]{beamer}
%%% License: Creative Commons Attribution Share Alike 4.0 (see https://creativecommons.org/licenses/by-sa/4.0/)
%%% Slides are based heavily on earlier versions of this course taught by Jesper Rudiger and Peter Norman Sorensen.

\DeclareGraphicsExtensions{.eps, .pdf,.png,.jpg,.mps,}
\usetheme{reMedian}
\usepackage{parskip}
\makeatother

\renewcommand{\baselinestretch}{1.1} 

\usepackage{amsmath, amssymb, amsfonts, amsthm}
\usepackage{enumerate}
\usepackage{hyperref}
\usepackage{url}
\usepackage{bbm}
\usepackage{color}

\usepackage{tikz}
\usepackage{tikzscale}
\newcommand*\circled[1]{\tikz[baseline=(char.base)]{
		\node[shape=circle,draw, inner sep=-20pt] (char) {#1};}}
\usetikzlibrary{automata,positioning}
\usetikzlibrary{decorations.pathreplacing}
\usepackage{pgfplots}
\usepgfplotslibrary{fillbetween}
\usepackage{graphicx}

\usepackage{setspace}
%\thinmuskip=1mu
%\medmuskip=1mu 
%\thickmuskip=1mu 


\usecolortheme{default}
\usepackage{verbatim}
\usepackage[normalem]{ulem}

\usepackage{apptools}
\AtAppendix{
	\setbeamertemplate{frame numbering}[none]
}
\usepackage{natbib}




\title{Financial Markets Microstructure}

\author{Egor Starkov}

\date{K{\o}benhavns Unversitet \\
	Spring 2022}



\begin{document}
	
%\frame[plain]{\titlepage}
%\addtocounter{framenumber}{-1}

\begin{frame}{Ch 9, ex 1}
	\textbf{Liquidity premium in the presence of dividend income.}
	Consider a stock with a dividend yield $d$ per period and with fundamental
	value $\mu _{t}$ at date $t$ (equal to its midprice $m_{t}$ at that date).
	Investors hold the stock for one period and can trade it at a constant
	percentage bid-ask spread $s$ in each period. Their required rate of return
	on the stock is given, equal to $r$.
\end{frame}


\begin{frame}{Ch 9, ex 1, part a}
	\begin{exampleblock}{}
		(a) Define the gross-of-transaction-cost return $1+R$ in terms of $\mu _{t}$, $\mu _{t+1}$, and $d$.
	\end{exampleblock}
	
	\pause
	
	\[ 1+R = \frac{\mu_{t+1}}{\mu_t} + d \]
	
	Note that ``gross-of-transaction-cost return'' interpretation of $R$ differs from how we interpreted $R$ in lectures (which was ``rate of price growth'').
\end{frame}


\begin{frame}{Ch 9, ex 1, part b}
	\begin{exampleblock}{}
		(b) Determine the equilibrium gross return $1+R$ as a function of $r$, $s$, and $d$ alone.
	\end{exampleblock}
	
	\pause
	
	\begin{itemize}
		\item Asset bought at $t$ at price $a_t = (1 + s/2) \mu_t$,
		\item sold at $t+1$ at $b_{t+1} = (1 - s/2) \mu_{t+1}$,
		\item yields $d$ in the interim.
		\item Required return is $1 + r = \frac{b_{t+1} + d \mu_t}{a_t}$.
		\item Plugging in and rearranging:
		\[ 1+R = \frac{1}{1-\frac{s}{2}} \left[ (1+r)\left(1+\frac{s}{2}\right) - \frac{s}{2} d \right] \]
	\end{itemize}
\end{frame}


\begin{frame}{Ch 9, ex 1, part c}
	\begin{exampleblock}{}
		(c) How does the liquidity premium respond to an increase in the
		dividend yield $d$? What is the intuitive reason for this result?
	\end{exampleblock}
	
	\pause
	
	\begin{itemize}
		\item Liquidity premium is 
		\[ R-r = \left(1+r-\frac{d}{2}\right) \frac{s}{1-\frac{s}{2}} \]
		\item Decreasing in $d$
		\begin{itemize}
			\item dividends do not suffer from stock illiquidity
			\item if larger share of the return is generated by the dividend, investors lose less from illiquidity and require smaller liquidity premium.
		\end{itemize}
	\end{itemize}
\end{frame}

\end{document} 