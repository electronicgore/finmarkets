%%% License: Creative Commons Attribution Share Alike 4.0 (see https://creativecommons.org/licenses/by-sa/4.0/)
%%% Slides are based heavily on earlier versions of this course taught by Jesper Rudiger.

\documentclass[english,10pt
%,handout
,aspectratio=169
]{beamer}
%%% License: Creative Commons Attribution Share Alike 4.0 (see https://creativecommons.org/licenses/by-sa/4.0/)
%%% Slides are based heavily on earlier versions of this course taught by Jesper Rudiger and Peter Norman Sorensen.

\DeclareGraphicsExtensions{.eps, .pdf,.png,.jpg,.mps,}
\usetheme{reMedian}
\usepackage{parskip}
\makeatother

\renewcommand{\baselinestretch}{1.1} 

\usepackage{amsmath, amssymb, amsfonts, amsthm}
\usepackage{enumerate}
\usepackage{hyperref}
\usepackage{url}
\usepackage{bbm}
\usepackage{color}

\usepackage{tikz}
\usepackage{tikzscale}
\newcommand*\circled[1]{\tikz[baseline=(char.base)]{
		\node[shape=circle,draw, inner sep=-20pt] (char) {#1};}}
\usetikzlibrary{automata,positioning}
\usetikzlibrary{decorations.pathreplacing}
\usepackage{pgfplots}
\usepgfplotslibrary{fillbetween}
\usepackage{graphicx}

\usepackage{setspace}
%\thinmuskip=1mu
%\medmuskip=1mu 
%\thickmuskip=1mu 


\usecolortheme{default}
\usepackage{verbatim}
\usepackage[normalem]{ulem}

\usepackage{apptools}
\AtAppendix{
	\setbeamertemplate{frame numbering}[none]
}
\usepackage{natbib}




\title{Financial Markets Microstructure}

\author{Egor Starkov}

\date{K{\o}benhavns Unversitet \\
	Spring 2022}



\begin{document}
	
%\frame[plain]{\titlepage}
%\addtocounter{framenumber}{-1}

\begin{frame}{Ch 8, ex 2}
	\textbf{Price discovery and transparency. }Consider the model of
	\structure{post-trade transparency} described in section 8.2. Consider the time-averaged
	expected squared deviation between the transaction price and the true value
	of the security, that is
	
	\begin{equation*}
		\frac{E\left[ (p_{1}^{k}-v)^{2}\right] }{2}+\frac{E\left[ (p_{2}^{k}-v)^{2}%
			\right] }{2},
	\end{equation*}%
	where $p_{t}^{k}$ is the transaction price in period $t=1,2$ in regime $%
	k=T,O $ (transparent, opaque). Show that price discovery is more efficient
	in the transparent market. You may limit your analysis to the case $\pi >%
	\frac{1}{2} $ in which the equilibrium first-period spread is positive.
\end{frame}


\begin{frame}{Post-trade transparency}
	\begin{itemize}
		\item If orders arrive sequentially, what effect does information about \alert{past orders} have?
		\item \textbf{Value}: high $v^H$ or low $v^L$ with equal probability
		\begin{itemize}
			\item Mean: $\mu=(v^H+v^L)/2$
			\item \alert{Denote} $\sigma = (v^H-v^L)/2$
		\end{itemize}
		\item \textbf{Dealers}: set quotes, competitive, risk neutral
		\item \textbf{Traders}: two traders arrive, submit unit market orders
		\begin{itemize}
			\item With prob. $\pi$: both are informed
			\item With prob. $(1-\pi)/2$: both liquidity traders;  first  seller, then buyer
			\item With prob. $(1-\pi)/2$: both liquidity traders;  first  buyer, then seller
		\end{itemize}
		\item \textbf{Transparent market}: All dealers observe the first order $y_1$
		\begin{itemize}
			\item Set $a_1=\mu+\pi(v^{H}-\mu)$ and $a_{2y_1}=\mathbb{E}[v|y_1,buy]$
		\end{itemize}
	\end{itemize}
\end{frame}


\begin{frame}{Post-trade transparency: Period 2}
	\textbf{Opaque market}: First dealer gains informational advantage. Focus on \alert{ask side}
	\begin{itemize}
		\item \structure{Period 2}. Denote the dealer who observed period-1 trade by \alert{$I$}, and the other dealer by \alert{$U$}.
		\begin{itemize}
			\item For technical reasons, suppose $I$ sets price after observing $U$'s quote
			\item \structure{Dealer $U$}: How to quote if you didn't see the first trade and second trade is buy? 
			\begin{itemize}
				\item If you set ask $a^U_2<v^{H}$ you will be undercut by $I$  if first order was a sell
				\item You only get to trade if first order was buy: lose $v^{H}-a^U_2$
				\item Thus, uninformed dealers need to quote $a^U_2=v^{H}$
			\end{itemize}
			\item \structure{Dealer $I$}: Suppose you saw the first trade, and second trade is a buy:
			\begin{itemize}
				\item Set price at $a^I_{2s}=v^{H}$ if first trade was a sell, and $a^I_{2b} = v^H - \epsilon$ if buy
				\item  $I$ wins period-2 buy order if $y_1$ was a sell (otherwise can undercut $U$)
				\item  $U$ wins period-2 buy order if $y_1$ was a buy, since $I$ knows that asset value is high
			\end{itemize}
		\end{itemize}
	\end{itemize}
\end{frame}


\begin{frame}{Post-trade transparency: Period 1}
	\begin{itemize}
		\item \structure{Period 1.} The sequential information advantage uncovered in the previous slide can make dealers bid keenly for the first order
		(Forex dealers often said to quote negative spread to large traders)
		\begin{itemize}
			\item In second period, $I$'s profit is $(1-\pi)(v^{H}-v^{L})/2$. $U$'s profit is zero
			\item Competition leads the first period half-spread to be reduced by this amount, to $(2\pi-1)(v^{H}-v^{L})/2$ (dealers undercut each other to obtain information contained in first order)
			\item The uninformed's aggregate trading cost is $\pi(v^{H}-v^{L})$ - double the cost under transparency. Why is this?
		\end{itemize}
		\item Would dealers commit to transparency?
		\begin{itemize}
			\item No, there is always an incentive to hide your orders (section 8.4.2)
			\item May explain the rise of less transparent trading venues
		\end{itemize}
	\end{itemize}
\end{frame}


\begin{frame}{Ch 8, ex 2}
	\begin{itemize}
		\item \structure{Transparency}, $t=1$: 
		\begin{align*}
			a_1^T &= \mu + \pi \sigma
			&
			b_1^T &= \mu - \pi \sigma
		\end{align*}
		\begin{align*}
			E\left[ (p_{1}^{T}-v)^{2}\right] =& \frac{1}{2} \left[ \pi(a_1^T-v^H)^2 + (1-\pi)\frac{1}{2}(a_1^T-v^H)^2 + (1-\pi)\frac{1}{2}(b_1^T-v^H)^2 \right] +
			\\
			&+ \frac{1}{2} \left[ \pi(b_1^T-v^L)^2 + (1-\pi)\frac{1}{2}(b_1^T-v^L)^2 + (1-\pi)\frac{1}{2}(a_1^T-v^L)^2 \right]
			\\
			=& (1-\pi^2) \sigma^2
		\end{align*}
	\end{itemize}
\end{frame}


\begin{frame}{Ch 8, ex 2}
	\begin{itemize}
		\item $t=2$:
		\begin{itemize}
			\item $p_2 = v$ if informed at $t=1$,
			\item $p_2 = \mu$ if uninformed at $t=1$;
		\end{itemize}
		\[ \Rightarrow E\left[ (p_{2}^{T}-v)^{2}\right] = \pi \cdot 0 + (1-\pi) \sigma^2 \]
		\item Averaging over time:
		\[ \left[\frac{1}{2}(1-\pi^2)+\frac{1}{2}(1-\pi) \right] \sigma^2 = (1-\pi)\left(1+\frac{\pi}{2}\right) \sigma^2 \]
	\end{itemize}
\end{frame}


\begin{frame}{Ch 8, ex 2}
	\begin{itemize}
		\item \structure{Opaqueness}, $t=1$ (assuming $\pi > 1/2$ to avoid crossed quotes):
		\begin{align*}
			a_1^O &= \mu + (2\pi-1)\sigma & b_1^O &= \mu - (2\pi-1)\sigma
		\end{align*}
		\[ E\left[ (p_{1}^{O}-v)^{2}\right] = 2(1-\pi)\sigma^2 \]
		\pause
		\item $t=2$:
		\[ E\left[ (p_{2}^{O}-v)^{2}\right] = 2(1-\pi)\sigma^2 \]
		\item so the average is also $2(1-\pi)\sigma^2$
	\end{itemize}
\end{frame}


\begin{frame}{Ch 8, ex 2}
	Comparison:
	\begin{align*}
		\frac{E\left[ (p_{1}^{T}-v)^{2} \right]}{2} + \frac{E\left[ (p_{2}^{T}-v)^{2} \right]}{2} &< \frac{E\left[ (p_{1}^{O}-v)^{2} \right]}{2} + \frac{E\left[ (p_{2}^{O}-v)^{2} \right]}{2}
		\\
		(1-\pi)\left(1+\frac{\pi}{2}\right) \sigma^2 &< 2(1-\pi)\sigma^2
		\\
		1+\frac{\pi}{2} &< 2
	\end{align*}
	Transparency yields better price discovery
\end{frame}

\end{document} 