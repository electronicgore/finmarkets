%%% License: Creative Commons Attribution Share Alike 4.0 (see https://creativecommons.org/licenses/by-sa/4.0/)
%%% Slides are based heavily on earlier versions of this course taught by Jesper Rudiger.

\documentclass[english,10pt
%,handout
,aspectratio=169
]{beamer}
%%% License: Creative Commons Attribution Share Alike 4.0 (see https://creativecommons.org/licenses/by-sa/4.0/)
%%% Slides are based heavily on earlier versions of this course taught by Jesper Rudiger and Peter Norman Sorensen.

\DeclareGraphicsExtensions{.eps, .pdf,.png,.jpg,.mps,}
\usetheme{reMedian}
\usepackage{parskip}
\makeatother

\renewcommand{\baselinestretch}{1.1} 

\usepackage{amsmath, amssymb, amsfonts, amsthm}
\usepackage{enumerate}
\usepackage{hyperref}
\usepackage{url}
\usepackage{bbm}
\usepackage{color}

\usepackage{tikz}
\usepackage{tikzscale}
\newcommand*\circled[1]{\tikz[baseline=(char.base)]{
		\node[shape=circle,draw, inner sep=-20pt] (char) {#1};}}
\usetikzlibrary{automata,positioning}
\usetikzlibrary{decorations.pathreplacing}
\usepackage{pgfplots}
\usepgfplotslibrary{fillbetween}
\usepackage{graphicx}

\usepackage{setspace}
%\thinmuskip=1mu
%\medmuskip=1mu 
%\thickmuskip=1mu 


\usecolortheme{default}
\usepackage{verbatim}
\usepackage[normalem]{ulem}

\usepackage{apptools}
\AtAppendix{
	\setbeamertemplate{frame numbering}[none]
}
\usepackage{natbib}



\title{Financial Markets Microstructure \\ Lecture 5}

\subtitle{Glosten-Milgrom Model\\
	Chapter 3.3 of FPR}

\author{Egor Starkov}

\date{K{\o}benhavns Unversitet \\
	Spring 2025}



\begin{document}
	\AtBeginSection[]{
		\frame<beamer>{
			\frametitle{This lecture:}
			\tableofcontents[currentsection,currentsubsection]
	}}
	\frame[plain]{\titlepage}


\begin{frame}{What did we do last time?}
\begin{enumerate}
	\item Argued that market thinness is not the only source of illiquidity
	\item Poked holes in the Efficient Market Hypothesis
	\item Defined price efficiency in many ways
	\item Began talking about the GM model
\end{enumerate}
\end{frame}


\begin{frame}{Today}
\begin{enumerate}
	\item more Glosten-Milgrom!
\end{enumerate}
\end{frame}


\section{Glosten-Milgrom (reviewed and continued)}


\begin{frame}{GM85: Overview}
	\begin{itemize}
		\item Dynamic model, periods $t = 1,2,...$; \\
		(though we will be analyzing the stage game for a given period -- essentially static)
		\item Two players in every period:
		\begin{itemize}
			\item trader and dealer
			\item \alert{dealer} long-lived; trader new every period
			\item \alert{trader} can be informed or not
		\end{itemize}
		\item One asset with fundamental value $v$ (unknown), common belief $v \sim F(v)$
	\end{itemize}
\end{frame}



\begin{frame}{GM85: Model (1)}
	\textbf{Trader:} is either a speculator or a noise trader, can submit a market order $d_t \in \{1, -1\}$ to buy or sell one unit of the asset with fundamental value $v$ (or do nothing, $d_t=0$)
	\begin{itemize}
		\item \structure{Speculator} (probability $\pi$): has private information about $v$.
		\begin{itemize}
			\item We will usually assume speculator simply knows $v$ (not much changes if he only has a noisy private signal about it).
			\item Risk neutral, chooses his market order $d_t$ to maximize expected profit $d_t \cdot (v-p_t)$:
			%\item `Hides' behind noise traders
		\end{itemize}
		\item \structure{Noise trader} (probability $1-\pi$): no pvt info about $v$; trades for other reasons (hedging, liquidity).
		\begin{itemize}
			\item We assume he follows some fixed strategy: buys with probability $\beta_B$; sells w.p. $\beta_S$; abstains w.p. $1-\beta_B - \beta_S$
			\item \alert{Important}: this assumption is for simplicity only; this strategy can be perfectly rational! We just don't model what generates it.
			\\E.g., could say noise traders choose $d_t$ to maximize profit $\mathbb{E}[d_t(v+y_t-p_t)]$, where $y$ is $t$-trader's idiosyncratic valuation (due to risk, time, liquidity preferences...)
		\end{itemize}
	\end{itemize}
\end{frame}


\begin{frame}{GM85: Model (2)}
	\textbf{Dealer (market maker)}
	\begin{itemize}
		\item Risk neutral
		\item Willing to trade \alert{exactly one unit} (buy/sell/no trade) each period
		\item Sets \alert{bid and ask prices} (for a single unit)
		\item Quote price before seing trade (limit order)
		\item Does not know whether trader is speculator or noise trader (but knows $\pi$)
		\item Expected profit from trade is $\mathbb{E}[-d_t(v-p_t)]$
		\item \structure{Competitive}: prices=expected asset value conditional on information
		\item Trading is sequential: market orders served one by one
	\end{itemize}
\end{frame}


\begin{frame}{GM85: Model (3)}
	\begin{itemize}
		\item \textbf{Equilibrium}:
		\begin{itemize}
			\item An equilibrium consists of \structure{bid and ask prices} and \structure{speculator's strategy}
			\item They must be such that: (i) prices are competitive (zero profit for MM), (ii) speculator best-responds to prices (maximizes expected gain).
		\end{itemize}
	\end{itemize}
\end{frame}


\begin{frame}{Analysis. A: Market making}
	\begin{itemize}
		\item Dealer quotes bid and ask prices on \textit{one unit}
		\begin{itemize}
			\item Can revise prices between each incoming trade
		\end{itemize}
		\item Quoted ask price $a_t$ only relevant if next incoming trader decides to buy
		\begin{itemize}
			\item Dealer's payoff in this case is given by $\mathbb{E}[a_t-v|\Omega_{t-1}, Buy] = a_t - \mathbb{E}[v|\Omega_{t-1}, Buy]$
			\item Same for bid $b_t$; payoff: $\mathbb{E}[v|\Omega_{t-1}, Sell] - b_t$
			\item (Note payoffs above rely on risk-neutrality)
		\end{itemize}
		\item Perfect competition among dealers implies zero expected profit from either trade type $\Rightarrow$ \structure{ask price} and \structure{bid price} are
		\begin{align*}
			a_t & = \mathbb{E}[v|\Omega_{t-1}, Buy]; \\
			b_t &= \mathbb{E}[v|\Omega_{t-1},  Sell].
		\end{align*}
		\item Notice that both sides of the equality depend on prices
	\end{itemize}
\end{frame}


\begin{frame}{Analysis. B: Informed trading}
	\begin{itemize}
		\item Speculator knows $v$. Given prices $a_t$ and $b_t$, the expected profits $\Pi$ are:
		\begin{equation*}
			\Pi(v,a_t,b_t,d_t)= \left\{
			\begin{aligned}
				&v - a_t  	&& \text{ if } d_t=1; \quad && (Buy)\\
				&0			&&\text{ if } d_t=0; \quad && (Abstain)\\
				&b_t - v 	&& \text{ if } d_t=-1. \quad && (Sell)
			\end{aligned}
			\right.
		\end{equation*}
		\pause
		\item Speculator's best response to $(a_t,b_t)$ is: (assume $a_t \geq b_t$)
		\begin{itemize}
			\item Buy when $v > a_t$, i.e. when $v$ is large enough
			\item Sell when $v<b_t$, i.e. when $v$ is small enough
			\item Abstain if $a_t > v > b_t$
		\end{itemize}
	\end{itemize}
\end{frame}


\begin{frame}{Analysis. C: Equilibrium definition}
	Dealer must make zero profit ({competition}), traders must trade optimally.
	This gives us two \structure{equilibrium conditions}.
	\begin{itemize}
		\item Let $\sigma_t$ denote the speculator's strategy, where $\sigma_t(d_t|v)$ is the probability that the speculator places order $d_t$ if value is $v$
		\item \alert{An equilibrium} consists of \structure{prices $(a_t,b_t)$} and \structure{strategy $\sigma_t$} such that:
		\begin{enumerate}
			\item the ask and bid prices  solve 
			\begin{align*}
				a_t & = \mathbb{E}[v| \Omega_{t-1}, Buy]; \\
				b_t & = \mathbb{E}[v| \Omega_{t-1}, Sell],
			\end{align*}
			given $\sigma_t$
			\item for each $v$, $\sigma_t$ solves 
			\[
			\max_{\sigma_t } \, \{\sigma_t(1|v) [v-a_t] + \sigma_t(-1|v)[b_t-v] \},
			\]
			given $(a_t,b_t)$.
		\end{enumerate} 
	\end{itemize}
\end{frame}


\begin{frame}{Back to homework}
	\begin{exampleblock}{GM Example}
		\begin{itemize}
			\item \textbf{Single period}: Suppose only one period (drop $t$ subscript, drop $\Omega$)
			\item \textbf{Binary outcome}: $v \in \{ 0, 1\}$, equally likely ex ante: $\mathbb{P}(v=1) = 0.5$.
			\item Suppose $0 < b < a < 1$ and noise trader's order obeys $\beta_B=\beta_S=0.5$. 
		\end{itemize}
		Questions:
		\begin{enumerate}
			\item What is the speculator's trading strategy?
			\item Can you derive dealer's \alert{prices $a$ and $b$}, as a function of $\pi$?
			\begin{itemize}
				\item If not, refresh your knowledge of conditional expectations and try again.
				\item If you already read the solution in the book, try to replicate it without looking back at the book.
			\end{itemize}
			\item Are the resulting prices efficient? (Check all three forms)
		\end{enumerate}
	\end{exampleblock}
\end{frame}


\begin{frame}{Analysis. D: Solving for equilibrium (1)}
	\begin{itemize}
		\item The orders reveal information about $v$. E.g., a \alert{buy order} is submitted:
		\begin{itemize}
			\item either by a noise trader (probability $(1-\pi)\beta_B$) -- no new information, $\mu_t = \mu_{t-1} = \mathbb{E}[v|\Omega_{t-1}]$;
			\item or by a speculator (probability $\pi \mathbb{P}(v \geq a_t | \Omega_{t-1})$) -- then learn that $v \geq a_t$, so $\mu_t = \mathbb{E} [v | \Omega_{t-1}, v \geq a_t]$.
		\end{itemize}
		\item Then $a_t$ is given by (using Bayes' rule and law of total probability; N=Noise, I=Informed):
		\visible<handout:0|2->{
		\begin{align*}
			a_t &= \mathbb{E}[v|\Omega_{t-1}, Buy]
			\\
			= \mathbb{P}(N|\Omega_{t-1},Buy) \cdot \mathbb{E}[v|\Omega_{t-1},Buy,N] &+ \mathbb{P}(I|\Omega_{t-1},Buy) \cdot \mathbb{E}[v|\Omega_{t-1},Buy,I]
			\\
			= \frac{\mathbb{P}(Buy,N|\Omega_{t-1})}{\mathbb{P}(Buy|\Omega_{t-1})} \cdot \mathbb{E}[v|\Omega_{t-1}] &+ \frac{\mathbb{P}(Buy,I|\Omega_{t-1})}{\mathbb{P}(Buy|\Omega_{t-1})} \cdot \mathbb{E} [v | \Omega_{t-1}, v \geq a_t]
			\\
			= \frac{(1-\pi)\beta_B}{(1-\pi) \beta_B + \pi \mathbb{P}(v \geq a_t)} \cdot \mu_{t-1} &+ \frac{\pi \mathbb{P}(v \geq a_t | \Omega_{t-1})}{(1-\pi) \beta_B + \pi \mathbb{P}(v \geq a_t)} \cdot \mathbb{E} [v | \Omega_{t-1}, v \geq a_t],
		\end{align*}
		}
		meaning that in the end, \structure{$a_t \geq \mu_{t-1}$}.
	\end{itemize}
\end{frame}


\begin{frame}{Analysis. D: Solving for equilibrium (2)}
	\begin{itemize}
		\item Similarly for \alert{sell orders}:
		\begin{itemize}
			\item sell order from a noise trader arrives with [unconditional] probability $(1-\pi)\beta_S$ -- no new information, $\mu_t = \mu_{t-1}$;
			\item sell order from a speculator arrives with probability $\pi \mathbb{P}(v \leq b_t | \Omega_{t-1})$ -- then learn that $v \leq b_t$, so $\mu_t = \mathbb{E} [v | \Omega_{t-1}, v \leq b_t]$.
		\end{itemize}
		\item Then $b_t$ is given by:
		\visible<handout:0|2->{
		$$ b_t = \mathbb{E}[v|\Omega_{t-1}, Sell] $$
		$$ = \frac{(1-\pi)\beta_S}{\mathbb{P}(Sell|\Omega_{t-1}, v)} \cdot \mathbb{E}[v|\Omega_{t-1}] + \frac{\pi \mathbb{P}(v \leq b_t | \Omega_{t-1})}{\mathbb{P}(Sell|\Omega_{t-1}, v)} \cdot \mathbb{E} [v | \Omega_{t-1}, v \leq b_t] $$}
		\begin{itemize}
			\visible<handout:0|2->{
			\item where $\mathbb{P}(Sell|\Omega_{t-1}, v) = (1-\pi) \beta_S + \pi \mathbb{P}(v\leq b_t)$,
			}
			\item so \structure{$b_t \leq \mu_{t-1}$}, and we have confirmed that indeed $a_t \geq b_t$.
		\end{itemize}
		\pause[3]
		\item These prices $a_t, b_t$ together with the speculator's trading strategy constitute the equilibrium.
	\end{itemize}
\end{frame}


\begin{frame}{Analysis. E: Profits}
	\begin{itemize}
		\item \structure{Informed traders} earn positive profit (since know $v$ and have an option of doing nothing)
		\item Dealers assumed competitive, hence zero profit
		\item \alert{Uninformed traders} incur a loss
		\begin{itemize}
			\item Although this is because we modeled this as a zero-sum game and explicitly ignored the uninformed traders' potential trading motives.
		\end{itemize}
	\end{itemize}
\end{frame}


\begin{frame}{Analysis. F: Price efficiency}
	\begin{itemize}
		\item $\mu_t \equiv \mathbb{E}[v | \Omega_t]$ is the expectation of $v$ \emph{after} the time-$t$ trade order is observed (by the dealer and outside observers). Note that in our model:
		$$\mu_t = \begin{cases}
			a_t > \mu_{t-1} & \text{ if buy order at } t;
			\\
			b_t < \mu_{t-1} & \text{ if sell order at } t.
		\end{cases} $$
		\item Meaning market price is efficient: $p_t = \mu_t$
		\begin{itemize}
			\item in \alert{semi-strong} form (prices anticipate and incorporate all information conveyed through trades),
			\item not in the \alert{strong} form (which would be equivalent to $p_t = v$)
		\end{itemize}
		\pause 
		\item This is because dealers are \structure{competitive} -- dealers' market power would ruin efficiency
		%\item let $s^a_t = a_t - p_t$ and $s^b_t = b_t - p_t$ denote the `half-spreads'.
	\end{itemize}
\end{frame}



%\iffalse 

\section{GM example}

\begin{frame}{Example (as in the book)}
\begin{itemize}
	\item Let's do an example similar to your homework; calculate the prices more explicitly.
	\bigskip 
	\item \textbf{Single period}: Suppose only one period (drop $t$ subscript, drop $\Omega$)
	\item \textbf{Binary outcome}: $v \in \{ v^H, v^L\}$, with prior $\theta=\mathbb{P}(v^H) $ 
	\item \textbf{Prior value}: What is the prior value of the asset before trading?
	\[
	\mu=\theta v^H+(1-\theta) v^L.
	\]
	\hyperlink{overview}{\beamerbutton{Skip example}}
\end{itemize}
\end{frame}


\begin{frame}{Example (2)}
	\begin{itemize}
		\item How do we solve the model? Look for equilibrium with trade. 
		\item Suppose $v^L < b < a < v^H$.
		\item Then speculator buys if $v=v^H$, sells if $v=v^L$. 
		\begin{itemize}
			\item That is, $\sigma(1|v^H)=1$ and $\sigma(-1|v^L)=1$
		\end{itemize}
		\item The procedure is then the following
		\begin{enumerate}
			\item Use the equilibrium conditions from before to calculate prices given the above speculator strategy
			\item Check that these prices satisfy $v^L < b < a < v^H$
		\end{enumerate}
	\end{itemize}
\end{frame}


\begin{frame}{Example (3)}
\begin{itemize}
	\item Let's solve for the ask price. First:
	\begin{align*}
		\mathbb{P}(Buy| v^H) & = (1-\pi) \beta_B + \pi \\
		\mathbb{P}(Buy| v^L) & = (1-\pi) \beta_B
	\end{align*}
	Then by Bayes' Rule
	\begin{align*}
		\mathbb{P}(v^H|Buy) & = \frac{\mathbb{P}(v^H) \mathbb{P}(Buy| v^H)}{\mathbb{P}(Buy)}  \\
		& = \frac{\theta [(1-\pi)\beta_B+\pi]}{(1-\pi)\beta_B+ \pi\theta}  \\
		& =  \theta + \frac{\theta(1-\theta) \pi}{(1-\pi)\beta_B+\pi\theta}
	\end{align*}
\end{itemize}
\end{frame}


\begin{frame}{Example (4)}
\begin{itemize}
	\item The ask price is the expected value, given a buy order:
	\begin{align*}
		a 
		& = \mathbb{P}(v^H| Buy) v^H + [1-\mathbb{P}(v^H| Buy)] v^L \\
		& = \left[ \theta + \frac{\theta(1-\theta) \pi}{(1-\pi)\beta_B+\pi\theta} \right] v^H + \left[ 1 - \left( \theta + \frac{\theta(1-\theta) \pi}{(1-\pi)\beta_B+\pi\theta} \right) \right] v^L\\
		& = \mu + \frac{\theta(1-\theta) \pi}{(1-\pi)\beta_B+\pi\theta} (v^H -v^L).
	\end{align*}
	\item Doing a similar exercise for $b$ we find
	\begin{align*}
		b 		&= \mu - \frac{\theta(1-\theta) \pi}{(1-\pi)\beta_S+\pi(1-\theta)}(v^H-v^L)
	\end{align*}
	\item Finally, we must check that our assumption holds: easy to check that $v^H > a > b > v^L$. Hence, \alert{this is an equilibrium}
\end{itemize}
\end{frame}


\begin{frame}{Example: Lessons}
	\begin{align*}
	a - \mu 		&= \frac{\theta(1-\theta) \pi}{(1-\pi)\beta_B+\pi\theta}(v^H-v^L) \\
	\mu - b 		&= \frac{\theta(1-\theta) \pi}{(1-\pi)\beta_S+\pi(1-\theta)}(v^H-v^L)
	\end{align*}
	\begin{itemize}
		\item Add the two expressions to get bid-ask \structure{spread} $S = a-b$
		\begin{itemize}
			\item $S$ \alert{increases in $\pi$}: more informed trading exacerbates adverse selection. Opposite for $(\beta_B+\beta_S)$.
			%NOTE: monotonicity in pi verified, difficult to compute
			\item If $\beta_B=\beta_S=1/2$, $S$ is \alert{increasing in $\theta(1-\theta)$}, i.e. spread higher when dealer faces greater initial uncertainty about $v$. Same for $(v^H-v^L)$.
		\end{itemize}
	\end{itemize}
\end{frame}


%\begin{frame}[label=insights]{Insights}
%	\begin{itemize}
%		\item Are prices efficient here? What kind of efficiency?
%		\hyperlink{efficiency}{\beamerbutton{Efficiency}}
%		\item What is the effect of speculators (insider traders)?
%		\hyperlink{insiders}{\beamerbutton{Insiders}}
%		\item The effect of price volatility ($v^H - v^L$)
%		\hyperlink{volatility}{\beamerbutton{Volatility}}
%		\item What's the effect of trading frequency? (How can we think about it here?)
%		\hyperlink{frequency}{\beamerbutton{Frequency}}
%		\item What is the dealer's profits? What about the speculators?
%		\hyperlink{profits}{\beamerbutton{Profits}}
%	\end{itemize}
%\end{frame}


\begin{frame}[label=example]{Example: Price discovery}
\begin{itemize}
	\item Return to the multiperiod setting. One unit traded every period, $v$ persistent.
	\item Trade flow is \alert{informative} -- trades have long-lasting effect on prices
	\item Each order conveys information, dealers learn, and 
	$$p_t \rightarrow v \quad
	\hyperlink{dynamics}{\beamerbutton{Dynamics}} $$
	prices \alert{strong-form efficient} in the long run
	\pause
	\item Speed of price discovery increasing in $\pi$
	\begin{itemize}
		\item Trade-off between \structure{price discovery} and \structure{liquidity}
	\end{itemize}
\end{itemize}
\end{frame}


\begin{frame}{Example: Simulation}
	Dealer beliefs: Each curve shows the evolution of dealer's beliefs in each run (10 runs of 100 orders)
	\quad
	\center
	\includegraphics[width=0.95\linewidth]{pics/DealerBeliefs_Image.pdf}
\end{frame}


\section{GM: conclusions}


\begin{frame}{GM: Summary of the findings}
	What did we learn from the Glosten and Milgrom model?
	\begin{enumerate}
		\item Information, prices and the spread
		\begin{itemize}
			\item Prices will reflect the information revealed by trades
			\item The spread is increasing in informational asymmetry (adverse selection) and in uncertainty about asset value
		\end{itemize}
		\item Informational efficiency
		\begin{itemize}
			\item Prices are always semi-strong efficient, in the long run also strong-form efficient
		\end{itemize}
		\item Noise trading
		\begin{itemize}
			\item Noise trading keeps the market liquid and improves spreads
			\item Informed speculation increases spreads, but improves price discovery - dilemma for regulators
		\end{itemize}
	\end{enumerate}
\end{frame}


\begin{frame}[label=overview]{GM: Discussion}
\structure{(anti?)Features}
\begin{itemize}
	\item \textbf{Dealer model}: Prices are set each period, discriminative, normally competitive (zero profits)
	\item \textbf{Non-market clearing}: Only one unit traded  - not market clearing (traders may wish to buy/sell more)
	\item Only \textbf{fundamental value} matters,  no speculation/resale
\end{itemize}
\structure{Discussion}
\begin{itemize}
	\item \textbf{Insights}: Adverse selection as a driver of the spread
	\item \textbf{Shortcomings}: Trade fixed amount, trade once, no resale
	\item \textbf{Advantages}: (Relatively) simple analysis, flexible %, trader is not price-taker (realizes price-impact)
\end{itemize}
\end{frame}


\begin{frame}{Homework}
	\begin{itemize}
		%\item See a text from The Securities and Exchange Commission on insider trading, from its homepage \url{http://www.sec.gov/answers/insider.htm}. Discuss potential restrictions in the definition.
		\item Reading:
		\begin{itemize}
			\item Read two articles on absalon on how ESMA restricted trading and binary options and SEC restricted trading in certain stocks.
			\item What is the difference between the underlying assets in the two cases?
			\item Explain ESMA's decision using GM model.
		\end{itemize}
		\item Solving:
		\begin{itemize}
			\item FPR chapter 3, exercise 3 (GM model where speculators are not perfectly informed, but instead receive a signal about the value of the asset)
			\item GM example with $v \sim U[0,1]$ (rest as in the problem assigned before today; goal: derive the equilibrium bid and ask prices)
		\end{itemize}
	\end{itemize}
\end{frame}


\section{extras}

\begin{frame}[label=dynamics]{Dynamics}
	Suppose we are in the simple binary model with the following parameters
	\begin{itemize}
		\item Probability of informed speculators: $\pi = 0.3$
		\item Probability (ex ante) of high value: $\theta = 0.5$
		\item $v^H=150$ and $v_L=100$
	\end{itemize}
	Consider 12 periods, with the following sequence of buys (b) and sells (s)
	\[
	ssbssssssssss
	\]
\end{frame}


\begin{frame}<handout:0>{Dynamics}
	First period: sell
	\center
	\includegraphics[width=0.9\linewidth]{pics/P1_Image.pdf}
\end{frame}


\begin{frame}<handout:0>{Dynamics}
	Second period: sell
	\center
	\includegraphics[width=0.9\linewidth]{pics/P2_Image.pdf}
\end{frame}


\begin{frame}<handout:0>{Dynamics}
	Third period: buy
	\center
	\includegraphics[width=0.9\linewidth]{pics/P3_Image.pdf}
\end{frame}


\begin{frame}<handout:0>{Dynamics}
	Fourth period: sell
	\center
	\includegraphics[width=0.9\linewidth]{pics/P4_Image.pdf}
\end{frame}


\begin{frame}<handout:0>{Dynamics}
	Fifth period: sell
	\center
	\includegraphics[width=0.9\linewidth]{pics/P5_Image.pdf}
\end{frame}


\begin{frame}<handout:0>{Dynamics}
	Sixth period: sell
	\center
	\includegraphics[width=0.9\linewidth]{pics/P6_Image.pdf}
\end{frame}


\begin{frame}<handout:0>{Dynamics}
	Seventh period: sell
	\center
	\includegraphics[width=0.9\linewidth]{pics/P7_Image.pdf}
\end{frame}


\begin{frame}<handout:0>{Dynamics}
	Eigth period: sell
	\center
	\includegraphics[width=0.9\linewidth]{pics/P8_Image.pdf}
\end{frame}


\begin{frame}<handout:0>{Dynamics}
	Ninth period: sell
	\center
	\includegraphics[width=0.9\linewidth]{pics/P9_Image.pdf}
\end{frame}


\begin{frame}<handout:0>{Dynamics}
	Tenth period: sell
	\center
	\includegraphics[width=0.9\linewidth]{pics/P10_Image.pdf}
\end{frame}


\begin{frame}<handout:0>{Dynamics}
	Eleventh period: sell
	\center
	\includegraphics[width=0.9\linewidth]{pics/P11_Image.pdf}
\end{frame}


\begin{frame}{Dynamics}
	\only<handout:0>{Twelfth period: sell}
	\center
	\includegraphics[width=0.9\linewidth]{pics/P12_Image.pdf}
\end{frame}


%\begin{frame}{Dynamics}
%	Let's consider an alternative case:
%	\[
%	bbbsssssssss
%	\]
%	Here we have a spell of buying in the beginning, before a long series of selling
%\end{frame}
%
%
%\begin{frame}{Dynamics}
%	\includegraphics[width=1\linewidth]{pics/Bubble_Image.pdf}
%	\hyperlink{example}{\beamerbutton{back}}
%\end{frame}


\end{document} 