%%% License: Creative Commons Attribution Share Alike 4.0 (see https://creativecommons.org/licenses/by-sa/4.0/)
%%% Slides are based heavily on earlier versions of this course taught by Jesper Rudiger.

\documentclass[english,10pt
%,handout
,aspectratio=169
]{beamer}
%%% License: Creative Commons Attribution Share Alike 4.0 (see https://creativecommons.org/licenses/by-sa/4.0/)
%%% Slides are based heavily on earlier versions of this course taught by Jesper Rudiger and Peter Norman Sorensen.

\DeclareGraphicsExtensions{.eps, .pdf,.png,.jpg,.mps,}
\usetheme{reMedian}
\usepackage{parskip}
\makeatother

\renewcommand{\baselinestretch}{1.1} 

\usepackage{amsmath, amssymb, amsfonts, amsthm}
\usepackage{enumerate}
\usepackage{hyperref}
\usepackage{url}
\usepackage{bbm}
\usepackage{color}

\usepackage{tikz}
\usepackage{tikzscale}
\newcommand*\circled[1]{\tikz[baseline=(char.base)]{
		\node[shape=circle,draw, inner sep=-20pt] (char) {#1};}}
\usetikzlibrary{automata,positioning}
\usetikzlibrary{decorations.pathreplacing}
\usepackage{pgfplots}
\usepgfplotslibrary{fillbetween}
\usepackage{graphicx}

\usepackage{setspace}
%\thinmuskip=1mu
%\medmuskip=1mu 
%\thickmuskip=1mu 


\usecolortheme{default}
\usepackage{verbatim}
\usepackage[normalem]{ulem}

\usepackage{apptools}
\AtAppendix{
	\setbeamertemplate{frame numbering}[none]
}
\usepackage{natbib}




\title{Financial Markets Microstructure \\ Lecture 11}

\subtitle{Limit order book (part 3)\\
	Chapter 6.3-6.4 of FPR}

\author{Egor Starkov}

\date{K{\o}benhavns Unversitet \\
	Spring 2025}




\begin{document}
\AtBeginSection[]{
\frame<beamer>{
\frametitle{This lecture:}
\tableofcontents[currentsection,currentsubsection]
}}
\frame[plain]{\titlepage}

%\section{Revision and readings}

\begin{frame}{Last time}
	\begin{itemize}
		\item First look into LOB markets using \cite{glosten_is_1994} (continuous and discrete versions).
		\item Limit traders act in the same capacity as the dealer did before
		\begin{itemize}
			\item but face different incentives
			\item so act differently
			\item which leads to different market outcomes.
		\end{itemize}
		\item First dive into market design: the effects of tick sizes, priority rules, and dealer interventions
	\end{itemize}
\end{frame}


\begin{frame}{Today}
	\begin{itemize}
		\item Dynamic analysis of LOB markets:
		\begin{itemize}
			\item How do traders choose between limit and market orders?
		\end{itemize}
	\end{itemize}
\end{frame}







\section{Market design}

\begin{frame}{Tick size and time priority}
	\begin{exampleblock}{}
		Suppose that there is no tick; quotes can  be placed in a continuous price space. Suppose that there is price priority. What is then the role of time priority, so that first-come quotes at identical prices are served first?
	\end{exampleblock}
	
	\only<handout:0>{
		\centering\pause
		\emph{``...The size of the trading increment is a
		powerful level that is underappreciated by
		regulators. The \alert{finer the trading increment},
		\structure{the more important price priority becomes
		relative to time priority}. In other words, if the
		market was designed to trade at continuous
		(vs discrete) non‐penny increments you could
		always win a trade by quoting the best price
		and the \structure{speed game would be non existent}''}
		
		\begin{flushright}
			--Narang, Manoj ``HFT 101 with Tradeworx'',
			May 20, 2014
		\end{flushright}
	}
\end{frame}


\begin{frame}<handout:0>{Tick size and time priority}
	\begin{itemize}
		\item The role of the tick is essentially \textit{to balance price priority and time priority}
		\item Larger tick gives more time priority, smaller tick gives more price priority
		\item When there is no tick, there is essentially no time priority...
		\item However, there is some evidence that ticks may arise \textit{endogenously}. In that case, time priority takes importance again
	\end{itemize}
\end{frame}





\section{Dynamic analysis: take or make liquidity}

\begin{frame}{Dynamic analysis}
	\begin{columns}
		\begin{column}{0.7\linewidth}
			\begin{itemize}
				\item \cite{glosten_is_1994} shows how price schedule is formed under adverse selection
				\begin{itemize}
					\item But this analysis is \structure<1>{static} -- traders have no choice between limit/market orders = making/taking liquidity
				\end{itemize}
				\item Proper \structure<1>{dynamic} analysis is quite difficult
				\pause
				\item Choice between \alert{MO/LO} depends on:
				\begin{enumerate}
					\item \structure{Price differential} (=bid-ask spread)
					\pause
					\item \structure{Execution probability} of a limit order
					\begin{itemize}
						\item ...which depends on the current state of LOB
						\item But also on next trader's choice between MO/LO
						\item ...which depends on execution probability ...
					\end{itemize}
					\pause
					\item \structure{Adverse selection} faced by limit orders
					\begin{itemize}
						\item ...which depends on \emph{who} will trade against my limit order
						\item ...which depends on next trader's choice between market/limit order depending on their pvt info ...
					\end{itemize}
				\end{enumerate}
			\end{itemize}
		\end{column}
		\begin{column}{0.3\linewidth}
			\pause[2]
			\visible<handout:0>{
				\includegraphics[scale=0.15]{pics/thonk}
				\vspace{7em}
			}
		\end{column}
	\end{columns}
\end{frame}


\begin{frame}{Literature}
	\begin{itemize}
		\item We will look at a very simple version of the \structure{\cite{foucault_order_1999}} model
		\begin{itemize}
			\item fixed ask/bid prices, endogenous choice between MO and LO
			\item The book for some reason attributes it to \cite{parlour_price_1998}, which is a quite different model of LOB markets. Old class materials may also call this ``the Parlour model''
		\end{itemize}
		\item Other models:
		\begin{itemize}
			%\item \cite{foucault_order_1999}: choice of both quotes and order type; adds public info between periods (generating winner's curse for LOs: if your LO executes, it's because there was news that is unfavorable for you)
			\item \cite{parlour_price_1998} and \citet*{foucault_limit_2005}: models with heterogeneous discount factors; patient traders choose LO, impatient choose MO
			\item \cite{bhattacharya_limit_2020}: combine all of the above in a single tractable model
			\item Worth noting: limit orders are effectively free call/put options, so could be priced as such (\cite{copeland_information_1983})
		\end{itemize}
	\end{itemize}
\end{frame}


\begin{frame}{Foucault Model: setup}
	\begin{itemize}
		\item \textbf{Exogenous prices}. \alert{Exogenous} bid and ask prices $A>v>B$
		\item \textbf{Strategic traders}. One trader arrives per period, chooses b/w limit or market order (one unit)
		\begin{itemize}
			\item LO valid for one period, then automatically cancelled. Choice depends on prob. of LO being executed, i.e. `hit' by a MO from the next trader
			\item Valuation: $v+\alert{y}$, where $y \sim U[-Y,Y]$ is idiosyncratic valuation, private and independent across traders. $v$ is known and common to all.
			Note: book assumes a binary distr-n $y \in \{-Y,Y\}$.
		\end{itemize}
		\item \textbf{Noise traders}. Numerous and patient. Always submit limit orders at $A$ (LSell) and/or $B$ (LBuy). Prioritised \emph{after} the strategic trader.
		\item \textbf{Goal}: derive the traders' optimal strategy
	\end{itemize}
\end{frame}


\begin{frame}{Profits}
	\begin{itemize}
		%if I want to sell, but there already is a LS in the book, then I still submit LS, because the other one will get cancelled before the next period
		\item Let $P^B_M (P^S_M)$ be the (endogenous) probability that the next trader submits a market buy (sell) order 
		\begin{align*}
			\text{Market sell: } & B-(v+y) \\
			\text{Limit sell: } & (A-v-y)P^B_M \\
			\text{Limit buy: } &(v+y-B)P^S_M \\
			\text{Market buy: } &v+y-A
		\end{align*}
		\item Limit order more attractive when more likely to execute: IRL (not in this model), this leads to automatic tendency for the limit order book to be replenished (resiliency)
	\end{itemize}
\end{frame}


\begin{frame}{Analysis}
	\begin{itemize}
		\item Look for stationary eqm where  $A$, $B$, $P^B_M$, $P^S_M$ are constant 
		\item From the trader profits, we see that there must exist $\overline{y}$,  $\underline{y}$ and $\hat{y}$ such that the best response (optimal order) of the trader is
		\begin{equation*}
		BR=\left\{ \begin{aligned}
		&MSell		&& 	\text{ if } 	&&	-Y\le y \le \underline{y} \\
		&LSell 		&&	\text{ if } 	&&	\underline{y} \le y \le \hat{y} \\
		&LBuy		&&	\text{ if } 	&&	\hat{y} \le y \le \overline{y} \\
		&MBuy	&&	\text{ if } 	&&	\overline{y} \le y \le Y
		\end{aligned}
		\right.
		\end{equation*}
		Traders with greater urgency/need to trade (extreme $y$) use market orders; 
		\\
		``patient'' traders use limit orders.
	\end{itemize}
\end{frame}


\begin{frame}{Equilibrium}
	\begin{itemize}
		\item From trader's BR:
		\begin{align*}
		P^S_M 	&=\mathbb{P}(-Y\le y \le \underline{y})=\frac{\underline{y}+Y}{2Y} \\
		P^B_M 	&=\mathbb{P}(\overline{y} \le y \le Y)=\frac{Y-\overline{y}}{2Y}
		\end{align*}
		\pause
		\item At $\underline{y}$, indifferent btw $MSell$ and $LSell$. At $\hat{y}$, indifferent btw $LSell$ and $LBuy$. At $\overline{y}$, indifferent btw $LBuy$ and $MBuy$. Thus:
		\begin{align}
		B-(v+\underline{y}) 					& = (A-v-\underline{y})\frac{Y-\overline{y}}{2Y} \label{eq1}\\
		(A-v-\hat{y}) \frac{Y-\overline{y}}{2Y} 			& = (v+\hat{y}-B)\frac{\underline{y}+Y}{2Y} \label{eq2}\\
		(v+\overline{y}-B) \frac{\underline{y}+Y}{2Y}  	& = v+\overline{y}-A \label{eq3}
		\end{align}
	\end{itemize}
\end{frame}


\begin{frame}{Equilibrium (2)}
	\begin{itemize}
		\item Solving \eqref{eq1}-\eqref{eq3} we obtain the thresholds. 
		\\
		If we set $v=0$, $Y=2$, $A=1$ and $B=-1$ they become
		\begin{align*}
		\underline{y} 	& = \frac{1}{2}(3-\sqrt{33}) \simeq 1.4 \\
		\hat{y} 	& = 0 \\
		\overline{y} 		& = \frac{1}{2}(\sqrt{33}-3) \simeq -1.4 
		\end{align*}
		\item Thus, in equilibrium the probability of a market buy/sell order next period is $\frac{2-1.4}{2\cdot 2}=\frac{-1.4+2}{2 \cdot 2}=0.15$
		\item Given a large spread ($A-B=2$), limit order traders are willing to accept a low execution probability ($15\%$) in order to obtain better price
	\end{itemize}
\end{frame}


\begin{frame}[label=parlourmain]{Dynamic analysis: Conclusion}
\begin{itemize}
	\item Trade occurs when seller and buyer meet and agree on terms
	\begin{itemize}
		\item Exchange brings sellers and buyers together
		\item Impatient traders search more actively (use market orders)
		\item Patient traders are more passive, offer liquidity (use limit orders)
	\end{itemize}
	\item Ordinary traders can use limit orders:
	\begin{enumerate}
		\item to compete with market makers (and profit from providing liquidity as a self-contained activity)
		\item or to reduce their trading costs by accepting some non-execution risk. E.g., \citet*{frazzini_trading_2018}: avg trading cost of a large institutional trader is $\sim 0.015\%$ (a couple of orders of magnitude smaller than quoted spreads), arguably due to using LOs to minimize price impact.
	\end{enumerate}
	\item Limit order book is dynamic: influence the choice of order
	\item In reality, many features to think about when placing orders
\end{itemize}
\end{frame}



\section{Dynamics and information}

\begin{frame}{Dynamics with adverse selection}
	\begin{itemize}
		\item Our analysis above focuses on non-execution risk, ignoring \structure{adverse selection}
		\begin{itemize}
			\item which can arise due to privately informed investors picking off the limit orders
			\item or even \alert{public news} arriving over time -- then all future traders are more informed!
			\item see ch. 6.4.3 \& 6.6 of the textbook or \hyperlink{adverse}{\beamerbutton{the extra slides}}
		\end{itemize}
		\item There is not too much interaction between the two, but there's an interesting dynamics of spread w.r.t. volatility of $\mu_t$ ($=v_t$ in the book)
		\item Read this at home.
	\end{itemize}
\end{frame}


\begin{frame}{Public information: issues and solutions}
	\begin{itemize}[<+->]
		\item \textbf{Monitoring costs}: if limit traders do not monitor the news constantly, their orders will get picked off. Monitoring reduces adverse selection, but is costly.
		\item \textbf{Pegged limit orders}: tied to another security or another market, automatically repriced when that price changes. Another insurance device, available on selected exchanges.
		\item \textbf{Algorithmic trading}: allows to automatically monitor relevant variables and reprice limit orders as needed. But at the same time increases the speed at which orders can be picked off, so cuts both ways.
	\end{itemize}
\end{frame}


\begin{frame}{Informed limit trading?}
	\begin{itemize}
		\item But of course, the best way to avoid being picked off is to be already informed.
		\item Even the full version of Foucault model only deals with public info -- there are no privately informed traders.
		\item But data shows that LOs are informative (\citet*{biais_empirical_1995,beber_order_2005,cao_information_2009})
		\item Even more: informed traders might be using more LOs than MOs (\cite{kaniel_so_2006})
		\item \cite{bhattacharya_limit_2020} have a LOB model in which informed traders use both LOs and MOs, depending on how full the book is.
	\end{itemize}
\end{frame}


\begin{frame}{LOB: conclusion}
	\begin{itemize}
		\item A lot of stuff going on in LOBs, difficult to analyze, but we tried.
		\item See \cite{parlour_limit_2008} for a slightly more detailed review of literature on LOBs.
	\end{itemize}
\end{frame}




\begin{frame}{Exercise for next week}
	\begin{itemize}
		\item Solve exercise 5 after chapter 6 (page 235) in the textbook on the effect of fees charged for limit orders and market orders in Foucault model
	\end{itemize}
\end{frame}





\appendix
\begin{frame}<handout:0>[allowframebreaks]{References}
\bibliography{../teaching}
\bibliographystyle{abbrvnat}
\end{frame}


\begin{frame}[label=adverse]{Dynamic model with adverse selection}
	\begin{itemize}
		%\item Now, let us try to derive the time-$t$ prices $A_t$ and $B_t$ endogenously. 
		\item Consider now the Foucault model with \alert{public news}:
		the market valuation evolves as $\mu_t = \mu_{t-1}+\epsilon_t$ where $\epsilon_t \in \{-\sigma, \sigma \}$ with equal probabilities.
		\item  $\mu_t$ is observed in period $t$ (but LOs from $t-1$ cannot be cancelled or repriced). The period-$t$ trader has valuation $\mu_t+y_t$
		\item Again, we look for a stationary equilibrium. In particular:
		\begin{itemize}
			\item Suppose that $A_t-\mu_{t-1}=\mu_{t-1}-B_t=S$ is constant.
			\item Suppose there exist $\underline{y}(\epsilon_t),\overline{y}(\epsilon_t),\hat{y}(\epsilon_t)$ that describe the trader's strategy in \textit{each} period  as a function of the value innovation $\epsilon_t$ 
			\item Thus there exist time-invariant $P^S_M$ and $P^B_M$ that characterize the probability of a next-period market sell/buy order 
		\end{itemize}
	\end{itemize}
\end{frame}


\begin{frame}{Dynamic model with adverse selection (2)}
	\begin{itemize}
		\item Intuitively, $\underline{y}(\epsilon_t),\overline{y}(\epsilon_t),\hat{y}(\epsilon_t)$ are decreasing in $\epsilon_t$:
		\begin{itemize}
			\item higher $\epsilon_t$ means more favorable quotes for MB ($A_{t-1}$ is relatively low), 
			\item less favorable for MS ($B_{t-1}$ is relatively high)
		\end{itemize}
		\item The interesting trade-off here is generated by $\sigma$: when $\sigma$ increases,
		\begin{itemize}
			\item the worse is the risk of being picked-off (so limit orders are less appealing),
			\item but the lower is the execution risk (so limit orders are more appealing)
		\end{itemize}
	\end{itemize}
\end{frame}


\begin{frame}{Dynamic model with adverse selection (3)}
	\begin{itemize}
		\item To solve the model: the indifference conditions at $\underline{y},\hat{y},\bar{y}$ are
		\begin{align*}
		B_t-(\mu_t+\underline{y}(\epsilon_t)) 					& = (A_{t+1}-\mathbb{E}_t[\mu_{t+1}|MB]-\underline{y}(\epsilon_t) 	)P^B_M \\
		(A_{t+1}-\mathbb{E}_t[\mu_{t+1}|MB]-\hat{y}(\epsilon_t))P^B_M 			& = (\mathbb{E}_t[\mu_{t+1}|MS]+\hat{y}(\epsilon_t)-B_{t+1})P^S_M  \\
		(\mathbb{E}_t[\mu_{t+1}|MS]+\overline{y}(\epsilon_t)-B_{t+1}) P^S_M  	& = \mu_t+\overline{y}(\epsilon_t)-A_t 
		\end{align*}
		\item Focus on the first and last equation. Use the definition of $S$:
		\begin{align*}
		-S-\epsilon_t-\underline{y}(\epsilon_t) 					& = (S-\mathbb{E}_t[\epsilon_{t+1}|MB]-\underline{y}(\epsilon_t) 	)P^B_M \\
		(S+\mathbb{E}[\epsilon_{t+1}|MS]+\overline{y}(\epsilon_t)) P^S_M  	& =-S+\epsilon_t+\overline{y}(\epsilon_t)
		\end{align*}
		\item Solve for $\underline{y}(\epsilon_t)$ and $\overline{y}(\epsilon_t)$. Determine $S$ from $A_t=\mathbb{E}_t[\mu_t|MB]$ and $B_t= \mathbb{E}_t[\mu_t|MS]$. Notice adv. sel. only affects limit orders. \hyperlink{parlourmain}{\beamerbutton{Back}}
	\end{itemize}
\end{frame}


\end{document} 