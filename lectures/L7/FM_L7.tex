%%% License: Creative Commons Attribution Share Alike 4.0 (see https://creativecommons.org/licenses/by-sa/4.0/)
%%% Slides are based heavily on earlier versions of this course taught by Jesper Rudiger.

\documentclass[english,10pt
%,handout
,aspectratio=169
]{beamer}
%%% License: Creative Commons Attribution Share Alike 4.0 (see https://creativecommons.org/licenses/by-sa/4.0/)
%%% Slides are based heavily on earlier versions of this course taught by Jesper Rudiger and Peter Norman Sorensen.

\DeclareGraphicsExtensions{.eps, .pdf,.png,.jpg,.mps,}
\usetheme{reMedian}
\usepackage{parskip}
\makeatother

\renewcommand{\baselinestretch}{1.1} 

\usepackage{amsmath, amssymb, amsfonts, amsthm}
\usepackage{enumerate}
\usepackage{hyperref}
\usepackage{url}
\usepackage{bbm}
\usepackage{color}

\usepackage{tikz}
\usepackage{tikzscale}
\newcommand*\circled[1]{\tikz[baseline=(char.base)]{
		\node[shape=circle,draw, inner sep=-20pt] (char) {#1};}}
\usetikzlibrary{automata,positioning}
\usetikzlibrary{decorations.pathreplacing}
\usepackage{pgfplots}
\usepgfplotslibrary{fillbetween}
\usepackage{graphicx}

\usepackage{setspace}
%\thinmuskip=1mu
%\medmuskip=1mu 
%\thickmuskip=1mu 


\usecolortheme{default}
\usepackage{verbatim}
\usepackage[normalem]{ulem}

\usepackage{apptools}
\AtAppendix{
	\setbeamertemplate{frame numbering}[none]
}
\usepackage{natbib}




\title{Financial Markets Microstructure \\ Lecture 10}

\subtitle{Limit order book (part 2)\\
	Chapter 6.3-6.5 of FPR}

\author{Egor Starkov}

\date{K{\o}benhavns Unversitet \\
	Spring 2021}




\begin{document}
\AtBeginSection[]{
\frame<beamer>{
\frametitle{This lecture:}
\tableofcontents[currentsection,currentsubsection]
}}
\frame[plain]{\titlepage}

%\section{Revision and readings}

\begin{frame}{Last time}
	\begin{itemize}
		\item First look into LOB markets using \cite{glosten_is_1994}.
		\item Limit traders act in the same capacity as the dealer did before
		\begin{itemize}
			\item but face different incentives
			\item so act differently
			\item which leads to different market outcomes
		\end{itemize}
	\end{itemize}
\end{frame}


\begin{frame}{Today}
	\begin{itemize}
		\item Market design:
		\begin{itemize}
			\item We have discussed how dealer markets differ from LOB markets for different market participants
			\item What if we put a dealer in a LOB market?
			\item What about other dimensions of market design? Tick size, priority rules? How does this affect market outcomes?
		\end{itemize}
		\item Dynamic analysis of LOB markets:
		\begin{itemize}
			\item How do traders choose between limit and market orders?
		\end{itemize}
	\end{itemize}
\end{frame}







\section{Market design}

\begin{frame}{Market design}
	\begin{itemize}
		\item There are many dimensions in which legislation or exhange rules can regulate trade
		\item Today's phrase of the day: ``\structure{unintended consequences}''
		\begin{itemize}
			\item Attempts to mitigate a particular inefficiency may have far-fetching consequences
			\item We will look at a few examples
		\end{itemize}
	\end{itemize}
\end{frame}


\begin{frame}{Tick size}
	\begin{itemize}
		\item Assume \structure{time priority} is the second order after price priority
		\begin{itemize}
			\item I.e., first limit order posted at tick executes first
		\end{itemize}
		\item Profit of the limit trader at price $A_k$ is:
		\begin{itemize}
			\item Zero for the marginal (last) limit order at $A_k$
			\item Strictly positive for inframarginal orders (if $C>0$) because order executes with higher probability
		\end{itemize}
		\item Q: what happens if we \alert{change the tick size}?
		\visible<handout:0|1->{
			\item This profit is reduced with \alert{smaller tick sizes}
			\begin{itemize}
				\item Hence decreasing tick size drives away limit traders and reduces depth
				\item But it will also reduce spread (by design) and reduce trading costs for the opposite side of the market (liquidity demanders)
			\end{itemize}
		}
	\end{itemize}
\end{frame}


\begin{frame}{Tick size}
	\begin{itemize}
		\item Driving away limit traders intuitively also has dynamic repercussions
		\visible<handout:0|>{
			\begin{itemize}
				\item LOB is replenished more slowly after trades -- so market orders traded more frequently against non-competitive prices
			\end{itemize}
		}
		\item \cite{goldstein_eighths_2000} explored the NYSE 1997 case (tick size from $\$1/8$ to $\$1/16$)
		\begin{itemize}
			\item Trading costs decreased for small orders
			\item Unclear for large orders
			\item Aligns with our predictions (smaller spread, smaller depth)
		\end{itemize}
	\end{itemize}
\end{frame}


\begin{frame}{Tick size and time priority}
	\begin{exampleblock}{}
		Suppose that there is no tick; quotes can  be placed in a continuous price space. Suppose that there is price priority. What is then the role of time priority, so that first-come quotes at identical prices are served first?
	\end{exampleblock}
	
	\centering
	\emph{``...The size of the trading increment is a
	powerful level that is underappreciated by
	regulators. The \alert{finer the trading increment},
	\structure{the more important price priority becomes
	relative to time priority}. In other words, if the
	market was designed to trade at continuous
	(vs discrete) non‐penny increments you could
	always win a trade by quoting the best price
	and the \structure{speed game would be non existent}''}
	
	\begin{flushright}
		--Narang, Manoj ``HFT 101 with Tradeworx'',
		May 20, 2014
	\end{flushright}
\end{frame}


\begin{frame}{Tick size and time priority}
	\begin{itemize}
		\item The role of the tick is essentially \textit{to balance price priority and time priority}
		\item Larger tick gives more time priority, smaller tick gives more price priority
		\item When there is no tick, there is essentially no time priority...
		\item However, there is some evidence that ticks may arise \textit{endogenously}. In that case, time priority takes importance again
	\end{itemize}
\end{frame}


\begin{frame}{Priority rules}
	\begin{itemize}
		\item With \structure{pro-rata} allocation (limit orders at given tick executed proportionally to their size), as opposed to \alert{time priority}:
		\begin{itemize}
			\item The expected profit of all orders at price $A_k$ must be zero (as opposed to strictly positive)
			\item So execution probabilities must be lower for all orders
			\item Meaning more orders are posted under pro-rata.
			% not sure about this last part
		\end{itemize}
		\item Lower profits lead to same dynamic consequences as with reducing tick size
		\item Pro-rata allocation rule used in the electronic futures markets for the leading short-term interest rate and for the two-year U.S. Treasuries.
	\end{itemize}
\end{frame}


\begin{frame}{Hybrid market}
	Hybrid market
	\begin{itemize}
		\item Suppose a dealer can compete with the limit order book, as follows
		\item The dealer may observe the size $q$ before serving the order (last-mover advantage)
		\item Can profit by pricing at $\mathbb{E}[v|q=x]$ rather than an average of $\mathbb{E}[v|q \geq y]$ for $y \leq x$ which is used by the LOB
		\item Especially profitable on small trades
		\item But the existence of such a dealer invalidates our analysis of the LOB
		\item Profitable limit orders are being picked off
	\end{itemize}
\end{frame}


\begin{frame}{Example with hybrid market}
	\begin{itemize}
		\item Example 1 previous lecture continued. Assume an uninformed dealer receives order $q$. Can either send order to LOB or execute himself (at better price) 
		\item Focus on ask side. Let $A^H_k$ be the hybrid ask price. When dealer observes $q=q_S$, he knows trader is noise trader and thus $\mathbb{E}[v|\structure{N}]=\mu$. 
		\begin{itemize}
			\item Can execute order at just below $A^H_1$ and earn profit $A^H_1-\mu$.
		\end{itemize}
		\item Hence, only large orders $q=q_L$ are sent to LOB. LOB traders will expect this, and will price as if any order arriving to the LOB is large:
		\begin{align*}
		\mathbb{E}[v|q \ge q_S]=\mathbb{E}[v|q \ge q_L]=\mu+\frac{2\pi}{1+\pi} \sigma,
		\end{align*}
		and thus $A^H_1=A^H_2=\mu+\frac{2\pi}{1+\pi} \sigma$.
		\item $A^H_1>A_1$ and $A^H_2=A_2$: hybrid market less liquid than normal market
	\end{itemize}
\end{frame}


\begin{frame}{Hybrid market}
	\begin{itemize}
		\item To be fair:
		\item Addind a dealer to LOB market decreases liquidity...
		\begin{itemize}
			\item ...in ``good times'', when there would've been a thick LOB
		\end{itemize}
		\item But this can help in bad times
		\begin{itemize}
			\item if LOB is empty then adding a dealer has no adverse effects and will actually increase liquidity
		\end{itemize}
		\item So in the end, adding a dealer is like a \structure{liquidity insurance} for the market
	\end{itemize}
\end{frame}


\begin{frame}{Market Design: conclusion}
	\begin{itemize}
		\item Regulation aimed at improving market liquidity can backfire by distorting agents' incentives
		\item Shameless plug:
		\begin{itemize}
			\item If you want to know more about market design, I teach a course on \structure{Mechanism Design} in Fall
			\item Explore how to design environments in which agents operate (i.e., distort their incentives) in order to generate desired behavior
		\end{itemize}
	\end{itemize}
\end{frame}



\section{Dynamic analysis: take or make liquidity}

\begin{frame}{Dynamic analysis}
	\begin{itemize}
		\item \cite{glosten_is_1994} models the static price schedule under adverse selection (with quote display costs)
		\item Traders must choose: place limit/market orders = make/take liquidity
		\begin{itemize}
			\item Limit order can yield price improvement compared to market order
			\item ...but at the cost of (best case) delay or (worst case) non-execution
			\item Plus there's always adverse selection (winner's curse)
		\end{itemize}
		\item Models of such choice:
		\begin{itemize}
			\item \cite{parlour_price_1998}: time priority of limit orders means it's more attractive to add limit orders when there's currently few of them
			\item \cite{foucault_order_1999}: limit order placed by traders whose private valuation makes current prices unattractive
			\item \citet*{foucault_limit_2005}: patient traders place limit orders inside the spread
		\end{itemize}
	\end{itemize}
\end{frame}


\begin{frame}{Take or make}
	\begin{itemize}
		\item Proper dynamic analysis is quite difficult
		\item E.g., my choice between market/limit order today depends on:
		\begin{itemize}
			\item Execution probability of a limit order
			\begin{itemize}
				\item ...which depends on the current state of LOB
				\item But also depends on next trader's choice between market/limit order
				\item ...which depends on execution probability etc
			\end{itemize}
			\item Adverse selection I face by submitting limit order
			\begin{itemize}
				\item ...which depends on \emph{who} will trade against my limit order
				\item ...which depends on next trader's choice between market/limit order depending on their pvt info ...
			\end{itemize}
		\end{itemize}
		\item So we will look at a very simple model, loosely based on \cite{parlour_price_1998}
	\end{itemize}
\end{frame}


\begin{frame}{Model based on Parlour (1998)}
	\begin{itemize}
		\item \textbf{Exogenous prices}. Bid and ask prices exogenously given as $A>v>B$
		\item \textbf{Traders}. Arriving trader chooses btw limit or market order (one unit)
		\begin{itemize}
			\item Limit order valid one period. Choice depends on prob. of limit order being executed, i.e. `hit' by a market order from the next trader
			\item Valuation: $v+y$. $y$ is uniformly distributed on $(-Y,Y)$, unobserved and independent across traders. $v$ is known and common to all.
		\end{itemize}
		\item \textbf{Profits}. Let $P^B_M (P^S_M)$ be prop. of next-period market buy (sell) order 
		\begin{align*}
		\text{Market sell: } & B-(v+y) \\
		\text{Limit sell: } & (A-v-y)P^B_M \\
		\text{Limit buy: } &(v+y-B)P^S_M \\
		\text{Market buy: } &v+y-A
		\end{align*}
	\end{itemize}
\end{frame}


\begin{frame}{Analysis}
	\begin{itemize}
		\item Look for stationary eq. where  $A$, $B$, $P^B_M$, $P^S_M$ are constant 
		\item From the trader profits, we see that there must exist $\overline{y}$,  $\underline{y}$ and $\hat{y}$ such that the best response (optimal order) of the trader is
		\begin{equation*}
		BR=\left\{ \begin{aligned}
		&MSell		&& 	\text{ if } 	&&	-Y\le y \le \underline{y} \\
		&LSell 		&&	\text{ if } 	&&	\underline{y} \le y \le \hat{y} \\
		&LBuy		&&	\text{ if } 	&&	\hat{y} \le y \le \overline{y} \\
		&MBuy	&&	\text{ if } 	&&	\overline{y} \le y \le Y
		\end{aligned}
		\right.
		\end{equation*}
		\item Traders with greater urgency to trade (extreme $y$) use market orders
		\item Limit order more attractive when more likely to execute:  automatic tendency for the limit order book to be replenished (resiliency)
	\end{itemize}
\end{frame}


\begin{frame}{Equilibrium}
	\begin{itemize}
		\item From trader's BR:
		\begin{align*}
		P^S_M 	&=\mathbb{P}(-Y\le y \le \underline{y})=\frac{\underline{y}+Y}{2Y} \\
		P^B_M 	&=\mathbb{P}(\overline{y} \le y \le Y)=\frac{Y-\overline{y}}{2Y}
		\end{align*}
		\item At $\underline{y}$, indifferent btw $MSell$ and $LSell$. At $\hat{y}$, indifferent btw $LSell$ and $LBuy$. At $\overline{y}$, indifferent btw $LBuy$ and $MBuy$. Thus:
		\begin{align}
		B-(v+\underline{y}) 					& = (A-v-\underline{y})\frac{Y-\overline{y}}{2Y} \label{eq1}\\
		(A-v-\hat{y}) \frac{Y-\overline{y}}{2Y} 			& = (v+\hat{y}-B)\frac{\underline{y}+Y}{2Y} \label{eq2}\\
		(v+\overline{y}-B) \frac{\underline{y}+Y}{2Y}  	& = v+\overline{y}-A \label{eq3}
		\end{align}
	\end{itemize}
\end{frame}


\begin{frame}{Equilibrium (2)}
	\begin{itemize}
		\item Solving \eqref{eq1}-\eqref{eq3} we obtain the thresholds. If we set $v=0$, $Y=2$, $A=1$ and $B=-1$ they become
		\begin{align*}
		\underline{y} 	& = \frac{1}{2}(3-\sqrt{33}) \simeq 1.4 \\
		\hat{y} 	& = 0 \\
		\overline{y} 		& = \frac{1}{2}(\sqrt{33}-3) \simeq -1.4 
		\end{align*}
		\item Thus, in equilibrium the probability of a market buy/sell order next period is $\frac{2-1.4}{2\cdot 2}=\frac{-1.4+2}{2 \cdot 2}=0.15$
		\item Given  large spread ($A-B=2$), limit order traders are willing to accept a low execution probability ($15\%$) in order to obtain better price
	\end{itemize}
\end{frame}


\begin{frame}{Dynamics with adverse selection}
	\begin{itemize}
		\item Our analysis above focuses on non-execution risk, ignoring adverse selection.
		\item The two can be combined in one model -- see \cite{foucault_order_1999} (also ch. 6.5 of the textbook). Also: \hyperlink{adverse}{\beamerbutton{Want to know more?}}
		\item There is no real interaction between the two though, so we will not get any new insights compared to Glosten and Parlour models.
	\end{itemize}
\end{frame}


\begin{frame}[label=parlourmain]{Conclusion}
	\begin{itemize}
		\item Trade occurs when seller and buyer meet and agree on terms
		\begin{itemize}
			\item Exchange brings sellers and buyers together
			\item Impatient traders search more actively (use market orders)
			\item Patient traders are more passive, offer liquidity (use limit orders)
		\end{itemize}
		\item Ordinary traders can use limit orders to compete with market makers
		\item Limit order book is dynamic: influence the choice of order
		\item In reality, many features to think about when placing orders
	\end{itemize}
\end{frame}


\begin{frame}{Exercise for next week}
	\begin{itemize}
		\item Solve exercise 5 on page 235 in the textbook on the effect of fees charged for limit orders and market orders
	\end{itemize}
\end{frame}





\appendix
\begin{frame}[allowframebreaks]{References}
\bibliography{../teaching}
\bibliographystyle{abbrvnat}
\end{frame}


\begin{frame}[label=adverse]{Dynamic model with adverse selection}
	\begin{itemize}
		\item Now, let us try to derive the time-$t$ prices $A_t$ and $B_t$ endogenously. 
		\item Suppose $v_t=v_{t-1}+\epsilon_t$ where $\epsilon_t \in \{-\sigma, \sigma \}$ with equal probability. $v_t$ is observed in period $t$. The period-$t$ trader has valuation $v_t+y_t$. 
		\item Again, we look for a stationary equilibrium. In particular:
		\begin{itemize}
			\item Suppose that $A_t-v_{t-1}=v_{t-1}-B_t=S$ is constant.
			\item Suppose there exist $\underline{y}(\epsilon_t),\overline{y}(\epsilon_t),\hat{y}(\epsilon_t)$  that describe the trader's strategy in \textit{each} period  as a function of the value innovation $\epsilon_t$ 
			\item Thus there exist time-invariant $P^S_M$ and $P^B_M$ that characterize the probability of a next-period market sell/buy order 
		\end{itemize}
	\end{itemize}
\end{frame}


\begin{frame}{Dynamic model with adverse selection (2)}
	\begin{itemize}
		\item We can write the equilibrium conditions as
		\begin{align*}
		B_t-(v_t+\underline{y}(\epsilon_t)) 					& = (A_{t+1}-\mathbb{E}_t[v_{t+1}|buy]-\underline{y}(\epsilon_t) 	)P^B_M \\
		(A_{t+1}-\mathbb{E}_t[v_{t+1}|buy]-\hat{y}(\epsilon_t))P^B_M 			& = (\mathbb{E}_t[v_{t+1}|sell]+\hat{y}(\epsilon_t)-B_{t+1})P^S_M  \\
		(\mathbb{E}_t[v_{t+1}|sell]+\overline{y}(\epsilon_t)-B_{t+1}) P^S_M  	& = v_t+\overline{y}(\epsilon_t)-A_t 
		\end{align*}
		\item Focus on the first and last equation. Use definition of $S$:
		\begin{align*}
		-S-\epsilon_t-\underline{y}(\epsilon_t) 					& = (S-\mathbb{E}_t[\epsilon_{t+1}|buy]-\underline{y}(\epsilon_t) 	)P^B_M \\
		(S+\mathbb{E}[\epsilon_{t+1}|sell]+\overline{y}(\epsilon_t)) P^S_M  	& =-S+\epsilon_t+\overline{y}(\epsilon_t)
		\end{align*}
		\item Solve for $\underline{y}(\epsilon_t)$ and $\overline{y}(\epsilon_t)$. Determine $S$ from $A_t=\mathbb{E}_t[v_t|buy]$ and $B_t= \mathbb{E}_t[v_t|sell]$. Notice adv. sel. affects only limit orders. \hyperlink{parlourmain}{\beamerbutton{Back}}
	\end{itemize}
\end{frame}


\end{document} 