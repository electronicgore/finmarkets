%%% License: Creative Commons Attribution Share Alike 4.0 (see https://creativecommons.org/licenses/by-sa/4.0/)
%%% Slides are based heavily on earlier versions of this course taught by Jesper Rudiger.

\documentclass[english,10pt
%,handout
,aspectratio=169
]{beamer}
%%% License: Creative Commons Attribution Share Alike 4.0 (see https://creativecommons.org/licenses/by-sa/4.0/)
%%% Slides are based heavily on earlier versions of this course taught by Jesper Rudiger and Peter Norman Sorensen.

\DeclareGraphicsExtensions{.eps, .pdf,.png,.jpg,.mps,}
\usetheme{reMedian}
\usepackage{parskip}
\makeatother

\renewcommand{\baselinestretch}{1.1} 

\usepackage{amsmath, amssymb, amsfonts, amsthm}
\usepackage{enumerate}
\usepackage{hyperref}
\usepackage{url}
\usepackage{bbm}
\usepackage{color}

\usepackage{tikz}
\usepackage{tikzscale}
\newcommand*\circled[1]{\tikz[baseline=(char.base)]{
		\node[shape=circle,draw, inner sep=-20pt] (char) {#1};}}
\usetikzlibrary{automata,positioning}
\usetikzlibrary{decorations.pathreplacing}
\usepackage{pgfplots}
\usepgfplotslibrary{fillbetween}
\usepackage{graphicx}

\usepackage{setspace}
%\thinmuskip=1mu
%\medmuskip=1mu 
%\thickmuskip=1mu 


\usecolortheme{default}
\usepackage{verbatim}
\usepackage[normalem]{ulem}

\usepackage{apptools}
\AtAppendix{
	\setbeamertemplate{frame numbering}[none]
}
\usepackage{natbib}




\title{Financial Markets Microstructure \\ Exercise class 5}

\author{Egor Starkov}

\date{K{\o}benhavns Unversitet \\
	Spring 2020}



\begin{document}
	\AtBeginSection[]{
		\frame<beamer>{
			\frametitle{This class:}
			\tableofcontents[currentsection,currentsubsection]
	}}
	
\frame[plain]{\titlepage}
\addtocounter{framenumber}{-1}


%\begin{frame}{Plan}
%	\begin{itemize}
%		\item L9: Ch 8, ex 2 (p 303)
%		\item L10: Ch 9, ex 1 (p 347)
%		\item PS2: ex 2
%	\end{itemize}
%\end{frame}



\section{L9: Ch 8, ex 2}

\begin{frame}{L9: Ch 8, ex 2}
	\textbf{Price discovery and transparency. }Consider the model of
	\structure{post-trade transparency} described in section 8.2. Consider the time-averaged
	expected squared deviation between the transaction price and the true value
	of the security, that is
	
	\begin{equation*}
	\frac{E\left[ (p_{1}^{k}-v)^{2}\right] }{2}+\frac{E\left[ (p_{2}^{k}-v)^{2}%
		\right] }{2},
	\end{equation*}%
	where $p_{t}^{k}$ is the transaction price in period $t=1,2$ in regime $%
	k=T,O $ (transparent, opaque). Show that price discovery is more efficient
	in the transparent market. You may limit your analysis to the case $\pi >%
	\frac{1}{2} $ in which the equilibrium first-period spread is positive.
\end{frame}


\begin{frame}{Post-trade transparency}
	\begin{itemize}
		\item If orders arrive sequentially, how does information about past orders affect
		\item \textbf{Value}: high $v^H$ or low $v^L$ with equal probability
		\begin{itemize}
			\item Mean: $\mu=(v^H+v^L)/2$
			\item \alert{Denote} $\sigma = (v^H-v^L)/2$
		\end{itemize}
		\item \textbf{Dealers}: set quotes, competitive, risk neutral
		\item \textbf{Traders}: two traders arrive, submit unit market orders
		\begin{itemize}
			\item With prob. $\pi$: both are informed
			\item With prob. $(1-\pi)/2$: both liquidity traders;  first  seller, then buyer
			\item With prob. $(1-\pi)/2$: both liquidity traders;  first  buyer, then seller
		\end{itemize}
		\item \textbf{Transparent market}: All dealers observe the first order $y_1$
		\begin{itemize}
			\item Set $a_1=\mu+\pi(v^{H}-\mu)$ and $a_{2y_1}=\mathbb{E}[v|y_1,buy]$
		\end{itemize}
	\end{itemize}
\end{frame}


\begin{frame}{Post-trade transparency: Period 2}
	\textbf{Opaque market}: First dealer gains informational advantage. Focus on \alert{ask side}
	\begin{itemize}
		\item \structure{Period 2}. Denote the dealer who observed period-1 trade by \alert{$I$}, and the other dealer by \alert{$U$}.
		\begin{itemize}
			\item For technical reasons, suppose $I$ sets price after observing $U$'s quote
			\item \structure{Dealer $U$}: How to quote if you didn't see the first trade and second trade is buy? 
			\begin{itemize}
				\item If you set ask $a^U_2<v^{H}$ you will be undercut by $I$  if first order was a sell
				\item You only get to trade if first order was buy: lose $v^{H}-a^U_2$
				\item Thus, uninformed dealers need to quote $a^U_2=v^{H}$
			\end{itemize}
			\item \structure{Dealer $I$}: Suppose you saw the first trade, and second trade is a buy:
			\begin{itemize}
				\item Set price at $a^I_{2s}=v^{H}$ if first trade was a sell, and $a^I_{2b} = v^H - \epsilon$ if buy
				\item  $I$ wins period-2 buy order if $y_1$ was a sell (otherwise can undercut $U$)
				\item  $U$ wins period-2 buy order if $y_1$ was a buy, since $I$ knows that asset value is high
			\end{itemize}
		\end{itemize}
	\end{itemize}
\end{frame}


\begin{frame}{Post-trade transparency: Period 1}
	\begin{itemize}
		\item \structure{Period 1.} The sequential information advantage uncovered in the previous slide can make dealers bid keenly for the first order
		(Forex dealers often said to quote negative spread to large traders)
		\begin{itemize}
			\item In second period, $I$'s profit is $(1-\pi)(v^{H}-v^{L})/2$. $U$'s profit is zero
			\item Competition leads the first period half-spread to be reduced by this amount, to $(2\pi-1)(v^{H}-v^{L})/2$ (dealers undercut each other to obtain information contained in first order)
			\item The uninformed's aggregate trading cost is $\pi(v^{H}-v^{L})$ - double the cost under transparency. Why is this?
		\end{itemize}
		\item Would dealers commit to transparency?
		\begin{itemize}
			\item No, there is always an incentive to hide your orders (section 8.4.2)
			\item May explain the rise of less transparent trading venues
		\end{itemize}
	\end{itemize}
\end{frame}


\begin{frame}{L9: Ch 8, ex 2}
	\begin{itemize}
		\item \structure{Transparency}, $t=1$: 
		\begin{align*}
			a_1^T &= \mu + \pi \sigma
			&
			b_1^T &= \mu - \pi \sigma
		\end{align*}
		\begin{align*}
			E\left[ (p_{1}^{T}-v)^{2}\right] =& \frac{1}{2} \left[ \pi(a_1^T-v^H)^2 + (1-\pi)\frac{1}{2}(a_1^T-v^H)^2 + (1-\pi)\frac{1}{2}(b_1^T-v^H)^2 \right] +
			\\
			&+ \frac{1}{2} \left[ \pi(b_1^T-v^L)^2 + (1-\pi)\frac{1}{2}(b_1^T-v^L)^2 + (1-\pi)\frac{1}{2}(a_1^T-v^L)^2 \right]
			\\
			=& (1-\pi^2) \sigma^2
		\end{align*}
	\end{itemize}
\end{frame}


\begin{frame}{L9: Ch 8, ex 2}
	\begin{itemize}
		\item $t=2$:
		\begin{itemize}
			\item $p_2 = v$ if informed at $t=1$,
			\item $p_2 = \mu$ if uninformed at $t=1$;
		\end{itemize}
		\[ \Rightarrow E\left[ (p_{2}^{T}-v)^{2}\right] = \pi \cdot 0 + (1-\pi) \sigma^2 \]
		\item Averaging over time:
		\[ \left[\frac{1}{2}(1-\pi^2)+\frac{1}{2}(1-\pi) \right] \sigma^2 = (1-\pi)\left(1+\frac{\pi}{2}\right) \sigma^2 \]
	\end{itemize}
\end{frame}


\begin{frame}{L9: Ch 8, ex 2}
	\begin{itemize}
		\item \structure{Opaqueness}, $t=1$ (assuming $\pi > 1/2$ to avoid crossed quotes):
		\begin{align*}
			a_1^O &= \mu + (2\pi-1)\sigma & b_1^O &= \mu - (2\pi-1)\sigma
		\end{align*}
		\[ E\left[ (p_{1}^{O}-v)^{2}\right] = 2(1-\pi)\sigma^2 \]
		\pause
		\item $t=2$:
		\[ E\left[ (p_{2}^{O}-v)^{2}\right] = 2(1-\pi)\sigma^2 \]
		\item so the average is also $2(1-\pi)\sigma^2$
	\end{itemize}
\end{frame}


\begin{frame}{L9: Ch 8, ex 2}
	Comparison:
	\begin{align*}
		\frac{E\left[ (p_{1}^{T}-v)^{2} \right]}{2} + \frac{E\left[ (p_{2}^{T}-v)^{2} \right]}{2} &< \frac{E\left[ (p_{1}^{O}-v)^{2} \right]}{2} + \frac{E\left[ (p_{2}^{O}-v)^{2} \right]}{2}
		\\
		(1-\pi)\left(1+\frac{\pi}{2}\right) \sigma^2 &< 2(1-\pi)\sigma^2
		\\
		1+\frac{\pi}{2} &< 2
	\end{align*}
	Transparency yields better price discovery
\end{frame}


\section{L10: Ch 9, ex 1}

\begin{frame}{L10: Ch 9, ex 1}
	\textbf{Liquidity premium in the presence of dividend income.}
	Consider a stock with a dividend yield $d$ per period and with fundamental
	value $\mu _{t}$ at date $t$ (equal to its midprice $m_{t}$ at that date).
	Investors hold the stock for one period and can trade it at a constant
	percentage bid-ask spread $s$ in each period. Their required rate of return
	on the stock is given, equal to $r$.
\end{frame}


\begin{frame}{L10: Ch 9, ex 1, part a}
	\begin{exampleblock}{}
		(a) Define the gross-of-transaction-cost return $1+R$ in terms of $\mu _{t}$, $\mu _{t+1}$, and $d$.
	\end{exampleblock}
	
	\pause
	
	\[ 1+R = \frac{\mu_{t+1}}{\mu_t} + d \]
	
	Note that ``gross-of-transaction-cost return'' interpretation of $R$ differs from how we interpreted $R$ in lectures (which was ``rate of price growth'').
\end{frame}


\begin{frame}{L10: Ch 9, ex 1, part b}
	\begin{exampleblock}{}
		(b) Determine the equilibrium gross return $1+R$ as a function of $r$, $s$, and $d$ alone.
	\end{exampleblock}
	
	\pause
	
	\begin{itemize}
		\item Asset bought at $t$ at price $a_t = (1 + s/2) \mu_t$,
		\item sold at $t+1$ at $b_{t+1} = (1 - s/2) \mu_{t+1}$,
		\item yields $d$ in the interim.
		\item Required return is $1 + r = \frac{b_{t+1} + d \mu_t}{a_t}$.
		\item Plugging in and rearranging:
		\[ 1+R = \frac{1}{1-\frac{s}{2}} \left[ (1+r)\left(1+\frac{s}{2}\right) - \frac{s}{2} d \right] \]
	\end{itemize}
\end{frame}


\begin{frame}{L10: Ch 9, ex 1, part c}
	\begin{exampleblock}{}
		(c) How does the liquidity premium respond to an increase in the
		dividend yield $d$? What is the intuitive reason for this result?
	\end{exampleblock}
	
	\pause
	
	\begin{itemize}
		\item Liquidity premium is 
		\[ R-r = \left(1+r-\frac{d}{2}\right) \frac{s}{1-\frac{s}{2}} \]
		\item Decreasing in $d$
		\begin{itemize}
			\item dividends do not suffer from stock illiquidity
			\item if larger share of the return is generated by the dividend, investors lose less from illiquidity and require smaller liquidity premium.
		\end{itemize}
	\end{itemize}
\end{frame}






\section{PS2: ex 2}

\begin{frame}{PS2: ex 2 [Ch.9, ex.6]}
	In Lecture 10 we saw a model of over-the-counter trading (DGP). Spread in that model is given by
	\[
	S=a-b=\frac{(1+z)c}{2(r+2\psi)+(1-2\psi)\phi(1-z)}
	\]
\end{frame}


\begin{frame}{PS2: ex 2 part a}
	\begin{exampleblock}{}
		(a) Assuming $z<1$, how does the bid-ask spread respond to changes in the probability $\phi$ of finding a dealer? How does the answer depend on the value of the probability $\psi$ that the investor's valuation will change in the future?
	\end{exampleblock}
	
	\pause
	\begin{align*}
		\frac{\partial S}{\partial \phi} & = -\frac{(1+z)c(1-z)(1-2\psi)}{[2(r+2\psi)+(1-2\psi)\phi(1-z)]^2}
	\end{align*}
	This fraction is positive if $\psi>1/2$, negative if $\psi<1/2$.	
\end{frame}


\begin{frame}{PS2: ex 2 part b}
	\begin{exampleblock}{}
		(b) What is the intuitive explanation for this result?
	\end{exampleblock}
	
	\pause
	
	The model has no adverse selection, so the spread is driven by the \textit{bargaining} between dealers and traders. Increasing $\phi$ implies that traders trade with higher probability in the future. There are two effects of this:
	\begin{itemize}
		\item It increases traders' \alert{willingness to pay} (higher $S$) since they have less risk of getting `stuck' with the asset if their preferences change
		\item It increases traders' \structure{bargaining power} (lower $S$) since they have higher probability of finding a dealer next period if they do not trade today
	\end{itemize}
	When $\psi$ is high, first effect dominates, since trader is worried about future trades. When $\psi$ is low, second effect dominates, since trader is more worried about current trade
\end{frame}




\end{document} 