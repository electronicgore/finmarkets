%%% License: Creative Commons Attribution Share Alike 4.0 (see https://creativecommons.org/licenses/by-sa/4.0/)
%%% Slides are based heavily on earlier versions of this course taught by Jesper Rudiger.

\documentclass[english,10pt
%,handout
,aspectratio=169
]{beamer}
%%% License: Creative Commons Attribution Share Alike 4.0 (see https://creativecommons.org/licenses/by-sa/4.0/)
%%% Slides are based heavily on earlier versions of this course taught by Jesper Rudiger and Peter Norman Sorensen.

\DeclareGraphicsExtensions{.eps, .pdf,.png,.jpg,.mps,}
\usetheme{reMedian}
\usepackage{parskip}
\makeatother

\renewcommand{\baselinestretch}{1.1} 

\usepackage{amsmath, amssymb, amsfonts, amsthm}
\usepackage{enumerate}
\usepackage{hyperref}
\usepackage{url}
\usepackage{bbm}
\usepackage{color}

\usepackage{tikz}
\usepackage{tikzscale}
\newcommand*\circled[1]{\tikz[baseline=(char.base)]{
		\node[shape=circle,draw, inner sep=-20pt] (char) {#1};}}
\usetikzlibrary{automata,positioning}
\usetikzlibrary{decorations.pathreplacing}
\usepackage{pgfplots}
\usepgfplotslibrary{fillbetween}
\usepackage{graphicx}

\usepackage{setspace}
%\thinmuskip=1mu
%\medmuskip=1mu 
%\thickmuskip=1mu 


\usecolortheme{default}
\usepackage{verbatim}
\usepackage[normalem]{ulem}

\usepackage{apptools}
\AtAppendix{
	\setbeamertemplate{frame numbering}[none]
}
\usepackage{natbib}



\title{Financial Markets Microstructure \\ Lecture 8}

\subtitle{Market Fragmentation\\
	Chapter 7 of FPR}

\author{Egor Starkov}

\date{K{\o}benhavns Unversitet \\
	Spring 2020}


\begin{document}
\AtBeginSection[]{
\frame<beamer>{
\frametitle{This lecture:}
\tableofcontents[currentsection,currentsubsection]
}}
\frame[plain]{\titlepage}

%\section{Revision and problems}

\begin{frame}{Last week(s)}
	\begin{itemize}
		\item Models of order-driven markets
		\item (Glosten's model): limit traders provide liquidity in a way similar but not same to dealers
		\item (Parlour's model): LOB is resilient / self-replenishing
		\item Market design: measures directed at improving liquidity can backfire by distorting incentives
	\end{itemize}
\end{frame}


\begin{frame}{Today}
	\begin{itemize}
		\item Analyze fragmented markets: i.e. multiple markets selling the same asset
		\item Look at fragmentation costs and benefits
		\item Compare these across batch markets and LOB markets
		\item This will give us an opportunity to revisit some of the models we have looked at previously
		\item Finally, we will look at regulation
	\end{itemize}
\end{frame}



\section{Fragmentation intro}

\begin{frame}{History lesson}
	\begin{itemize}
		\item A listing in the US used to confer near-exclusive trading rights
		\begin{itemize}
			\item For instance, if a stock were NYSE-listed, almost all of the trading would occur on the NYSE
			\item Until 1999, most stock options were only traded on the exchange where they were listed
		\end{itemize}
		\item Many European countries used to require that stocks of their companies are only traded on local exchanges
		\item All of that has changed in the past 20 years
		\begin{itemize}
			\item Today, many (at least high-cap) stocks are \structure{cross-listed} on many exchanges
			\item Even if a stock is not listed on a given exchange, it can be \structure{admitted for trading}
		\end{itemize}
	\end{itemize}
\end{frame}


\begin{frame}{Fragmentation and consolidation}
	\begin{itemize}
		\item Refer to markets with multiple venues trading the same stock as \textit{fragmentated} (as opposed to consolidated)
		\item Regulation often tries to create `virtual consolidation': make fragmented markets act \textit{as if} they were consolidated
		\begin{itemize}
			\item A central concept in the US is the \textit{National Best Bid and Offer} (NBBO): the highest bid and lowest ask price in (any exchange within) the market at a given time
			\item US regulation requires that trade always takes place at the NBBO
		\end{itemize}
		\item EU's ban of countries' ``concentration rules'' aims for a different kind of consolidation -- one where all stocks can be traded on same exchange rather than spread around local exchanges.
		\begin{itemize}
			\item We're not going to talk much about this dimension.
		\end{itemize}
	\end{itemize}
\end{frame}


\begin{frame}{Priority violations}
	\begin{itemize}
		\item  LOB has different kinds of priority: price, visibility,  time
		\item In a fragmented market, no overall coordination mechanism, so the following may occur:
		\begin{itemize}
			\item Violation of \textbf{visibility priority}. An undisplayed order at a price of 100 might be executed on exchange $A$ even though there are quantities visible at 100 on exchange $B$
			\item Violation of \textbf{time priority}. A limit order to buy at a price of 100 that was entered at 10:00 AM on exchange $A$ might be filled before an order to buy at 100 that was entered at 9:30 AM on Exchange $B$
			\item Violation of \textbf{price priority}. A limit order to buy at a price 100 on exchange $A$ might be executed even though there is, at the same time, a limit order to buy at a price of 101 on exchange $B$. Known as a \textit{trade through}
		\end{itemize}
	\end{itemize}
\end{frame}


\begin{frame}{Other consequences}
	Apart from violating priority rules, fragmentation can lead to:
	\begin{itemize}
		\item \alert{Higher trading costs} (due to search costs or not finding best price)
		\item \alert{Worse price discovery}
		\item \alert{Less total liquidity} (network effects)
	\end{itemize}
	\pause
	but also to:
	\begin{itemize}
		\item \structure{Lower trading costs} (due to competition among exchanges)
		\item \structure{Better price discovery} (due to aggregating many different signals instead of one)
		\item \structure{More total liquidity} (liq providers face less competition, so more willing to participate)
	\end{itemize}
	Will look at some of these and others in greater detail today
\end{frame}



\section{Fragmentation costs}

\begin{frame}{Costs: A. Adverse selection in standard Kyle}
	First, let us look at the effect of fragmentation on adverse selection
	\begin{itemize}
		\item Section 7.2.1 analysis draws on Kyle's model
		\item Recall the baseline model from our lecture:
		\begin{itemize}
			\item Risky asset value $v \sim N(\mu,\sigma^{2}_{v})$
			\item Insider observes $v$
			\begin{itemize}
				\item Places market order $x$; maximizes $\mathbb{E}[x(v-p)|x]$
			\end{itemize}
			\item Noise trader demand $u \sim N(0, \sigma^{2}_{u})$
			\item Market maker observes aggregate order flow $q = x + u$
			\begin{itemize}
				\item Competitively prices asset at $p = \mathbb{E}[v|q]$
				(Recall the difference here to what we did in the LOB model)
			\end{itemize}
		\end{itemize}
	\end{itemize}
\end{frame}


\begin{frame}{Adverse selection in standard Kyle (2)}
	\begin{itemize}
		\item The insider uses a linear strategy $x=\beta(v-\mu)$
		\item Market makers observe $q=x+u=\beta(v-\mu) + u$. Then
		\[
		p=\mathbb{E}[v|q]=\mu + \lambda q
		\]
		\item Solving the MM's problem we get price-impact parameter
		\[
		\lambda = \frac{\beta \sigma^{2}_{v}}{\beta^{2} \sigma^{2}_{v}+\sigma^{2}_{u}}
		\]
		\item The insider takes for granted the pricing rule $p=\mu+\lambda q$
		\begin{itemize}
			\item Expected gain is $x(v-\mu-\lambda x)$ since $\mathbb{E}[u]=0$
		\end{itemize}
		\item Solving trader's problem gives $x=\beta(v-\mu)$ where $\beta=1/(2\lambda)$
	\end{itemize}
\end{frame}


\begin{frame}{Adverse selection in standard Kyle (3)}
	\begin{itemize}
		\item `Match the coefficients' $\frac{1}{2\beta} = \lambda = \frac{\beta \sigma^{2}_{v}}{\beta^{2} \sigma^{2}_{v}+\sigma^{2}_{u}}$ to get the equilibrium
		\[
		\frac{1}{\lambda} = 2 \beta = 2 \frac{\sigma_{u}}{\sigma_{v}}
		\]
		\item Insider equilibrium trading:
		\[
		x = \frac{\sigma_{u}}{\sigma_{v}} (v-\mu)
		\]
		\item MM makes zero profit:  insider's gain is noise traders' loss
		\[
		\underbrace{\mathbb{E}[x(p-v)]}_{\textcolor{blue}{\text{Insider gain}}} = \underbrace{\mathbb{E}[u(\lambda u)]}_{\textcolor{blue}{\text{Noise loss}}} = \lambda \sigma^{2}_{u} = \frac{\sigma_{v}\sigma_{u}}{2}
		\]
	\end{itemize}
\end{frame}


\begin{frame}{Adverse selection in a fragmented market}
	Now fragment the market in this Kyle model. Suppose:
	\begin{enumerate}
		\item Two noise trader groups: $u=u_{1}+u_{2}$, $u_i$ independent and $u_{i} \sim N(0, \sigma^{2}_{ui})$
		\item Noise traders trade in separate markets; MMs and informed traders present in both markets
		\item Agents cannot choose their market (does not matter)
	\end{enumerate}
\end{frame}


\begin{frame}{Adverse selection in a fragmented market (2)}
	\begin{itemize}
		\item \textbf{Prices}.  Since on each market $i=1,2$ we must have $p_i=\mathbb{E}[v|q]$,
			\begin{align*}
				p_i &= \mu + \lambda_i q = \mu + \lambda (x_i + u_i)
				\\
				&= \mu + \lambda_i (\beta_i(v-\mu) + u_i)
				\\
				&= \mu + \frac{\sigma_v}{2 \sigma_{ui}} \left( \frac{\sigma_{ui}}{\sigma_v} (v-\mu) + u \right)
				\\
				&= \mu + \frac{1}{2}(v-\mu)+\frac{\sigma_v}{2\sigma_{ui}}u_i
			\end{align*}
		\structure{On average, price is the same} in both markets. But in very short run prices may differ across markets
	\end{itemize}
\end{frame}


\begin{frame}{Adverse selection in a fragmented market (3)}
	\begin{itemize}
		\item \textbf{Trading}. Informed trading is given by $x_i= \beta_i (v-\mu)$ with $\beta_i= \frac{\sigma_{ui}}{\sigma_{v}}$:
		\begin{center}
			$
			x_1+x_2=(v-\mu)\frac{\sigma_{u1}+\sigma_{u2}}{\sigma_v}>x
			$
		\end{center}
		\textcolor{red}{Agg. informed trading greater than in consolidated case}
		
		\item To see this, note: 
		\begin{align*}
		\mathbb{V}(u) = \sigma_u^2 &= \sigma_{u1}^2 + \sigma_{u2}^2 = \mathbb{V}(u_1+u_2)
		\\
		\Rightarrow
		\sigma_u^2 &< (\sigma_{u1} + \sigma_{u2})^2
		\\
		\Rightarrow
		\sigma_u &< \sigma_{u1} + \sigma_{u2}
		\end{align*}
	\end{itemize}
\end{frame}


\begin{frame}{Adverse selection in a fragmented market (4)}
	\begin{itemize}
		\item \textbf{Adverse selection costs}. Measure by loss of noise traders:
		\begin{itemize}
			\item The expected loss of group $i$ is $\sigma_{v} \sigma_{ui}/2$: Total loss is 
			\[
			\frac{\sigma_{v}(\sigma_{u1}+\sigma_{u2})}{2} > \frac{\sigma_{v} \sigma_{u}}{2}
			\]
			\item If two separate markets, \textit{greater adverse selection loss}
		\end{itemize}
		\textcolor{red}{More informed trading gives greater adverse selection loss}
	\end{itemize}
\end{frame}


\begin{frame}{Adverse selection in a fragmented market (4)}
	\begin{itemize}
		\item \textbf{Market depth}. The textbook says depth decreases in aggregate from fragmentation:
		\begin{align*}
		\lambda = \frac{\sigma_v}{2 \sigma_u} < \min \{ \lambda_1, \lambda_2 \}
		\end{align*}
		%This is \alert{wrong}.
	\end{itemize}
\end{frame}


\begin{frame}{Conclusion: Adverse selection in a fragmented market}
	\textbf{Adverse selection costs}
	\begin{itemize}
		\item Presence of other noise traders reduce adverse selection costs: fragmentation is bad for noise traders (welfare)
		\item If noise traders coordinate, it can be stable that they gather around a less efficient platform
		\item Such a coordination failure can be a barrier to entry for new trading platforms; Section 7.3
	\end{itemize}
	\textbf{Aside}
	\begin{itemize}
		\item Above observations may explain why trade volume is often concentrated at specific times of day (early and late)
		\item Also an argument for batch trading versus continuous trading
	\end{itemize}
\end{frame}



\begin{frame}{B. Less risk sharing}
We now turn to the effect fragmentation on market makers' risk sharing
\begin{itemize}
\item Section 7.2.2: Analysis drawn on Stoll's model
\item Risky asset value $v \sim N(\mu, \sigma^{2}_{v})$
\item Risk averse dealer with mean-variance preference
\begin{itemize}
\item Initial asset position $z$; risk aversion $\rho$
\item Asset supply is
\[
y = z + \frac{p-\mu}{\rho \sigma^{2}_{v}}
\]
\item Invert to get pricing schedule $p=m+\lambda q$ where $m=\mu-\lambda z$ and $\lambda = \rho \sigma^{2}_{v}$
\end{itemize}
\item \textbf{Trading cost}. Trading cost of order of size $q$ is $\lambda q^{2}$
\end{itemize}
\end{frame}


\begin{frame}{}
\begin{itemize}
\item Suppose there are two dealers: risk aversions $\rho_{1}$ and $\rho_{2}$
\begin{itemize}
\item If consolidated trading, dealers' aggregate supply curve is
\[
y_{1}+y_{2} = z_{1}+z_{2}+\frac{p-\mu}{\rho_{1}\sigma^{2}_{v}}+\frac{p-\mu}{\rho_{2} \sigma^{2}_{v}} =  z_{1}+z_{2}+\left(\frac{1}{\rho_1}+\frac{1}{\rho_2}\right)\frac{p-\mu}{\sigma^{2}_{v}}
\]
\item This is as if trading with one dealer with risk aversion $\bar{\rho}$:
\[
\frac{1}{\bar{\rho}}= \frac{1}{\rho_{1}} + \frac{1}{\rho_{2}}
\]
\end{itemize}
\end{itemize}
\begin{block}{}
\begin{itemize}
\item Risk averse dealers share risks in the consolidated market: $\bar{\rho}<\rho_{i}$
\item Trading cost $\bar{\rho}\sigma^{2}_{v}q^{2}$ is lower than either of $\rho_{i}\sigma^{2}_{v}q^{2}$
\end{itemize}
\end{block}
\end{frame}

%TODO
% all of the above assumes noone can choose which market to participate in.
% If a trader can split order across mkts in Kyle model, the AGGREGATE depth of the fragmented mkt is larger than that of a consolidated mkt
% If same in Stoll model, fragmentation has no effect (trader will equalize marginal prices = share risks among traders)
% If noise traders can flow between mkts in Kyle model, they will flow towards the deeper one (liqty begets liqty)


\begin{frame}{C. Less competition}
Finally, let's think about the effect of fragmentation on competition
\begin{itemize}
\item Section 7.2.3
\item Previously: imperfect competition tends to give lower depth in batch markets, as dealers strategically withhold supply of inframarginal units
\item Fragmentation tends to reduce order flow competition, and may therefore induce lower depth in markets
\item Just think of the `Cournot' model of competition presented in the book: fewer dealers imply lower market depth \textcolor{blue}{within a platform}
\end{itemize}
\end{frame}



\section{Fragmentation benefits}

\begin{frame}{Fragmentation benefits}
\begin{itemize}
\item Competition among trading platforms may reduce fees (section 7.4.1)
\item Also encourages technological innovation --  may benefit traders
\item Section 7.4.2: Competing limit order books may provide better aggregate liquidity
\begin{itemize}
\item To show this, use LOB model of section 6.2 with display cost $C>0$, tick size $\Delta >0$, no adverse selection (asset value is $\mu$)
\item Recall: incoming order $q$, limit sell orders posted at cumulative quantity $Y$ and price $A$ satisfy the zero-profit condition
\[
0 = \mathbb{P}(q \geq Y)[A-\mu]-C,
\]
solved by 
\begin{equation} \tag{6.7}
A= \mu + \frac{C}{\mathbb{P}(q \geq Y)}.
\end{equation}
\end{itemize}
\end{itemize}
\end{frame}


\begin{frame}{Fragmentation benefits (2)}
Now we make the following assumptions:
\begin{itemize}
\item \textbf{Fragmentation}. Limit orders are supplied in two markets, $I$ and $E$
\begin{itemize}
\item At ask price $A$, denote cumulative limit sell orders by $Y^I$ and $Y^E$
\end{itemize}
\item \textbf{Market orders}. 
\begin{itemize}
\item Market order is $Buy$ with chance 1/2, and of size $q \sim F(q)$
\item With probability $1-\gamma$, the entire order goes to $I$
\item With probability $\gamma$, the incoming order is split:
\begin{itemize}
\item With prob. 1/2, order first goes to $I$, whatever remains goes to $E$
\item With prob. 1/2, order first goes to $E$, whatever remains goes to $I$
\end{itemize}
\end{itemize}
\item  \textbf{Trade probability}. We need to model $\mathbb{P}(q_i \geq Y^i)$ for each market
\end{itemize}
\end{frame}


\begin{frame}{Fragmentation benefits: trading probabilities}
\begin{itemize}
\item This results in the following execution probabilities
\begin{align}
\mathbb{P}(q_I \geq Y^I) & = \frac{1}{2} \left[\underbrace{\left(1-\gamma+\frac{\gamma}{2} \right) (1-F(Y^I))}_{\textcolor{blue}{\text{$I$ is executed first}}} + \underbrace{\frac{\gamma}{2}(1-F(\bar{Y}))}_{\textcolor{blue}{\text{$I$ is executed second}}}\right] \tag{7.11} \\
\mathbb{P}(q_E \geq Y^E) & = \frac{1}{2} \left[\underbrace{\frac{\gamma}{2} (1-F(Y^E))}_{\textcolor{blue}{\text{$E$ is executed first}}} + \underbrace{\frac{\gamma}{2}(1-F(\bar{Y}))}_{\textcolor{blue}{\text{$E$ is executed second}}}\right] \tag{7.12}
\end{align}
where $\bar{Y}=Y^I+Y^E$
\end{itemize}
\end{frame}


\begin{frame}{Fragmentation benefits: equilibrium}
\begin{itemize}
\item If both markets active ($Y^I>0$ and $Y^E>0$) then (6.7) holds for both
\begin{itemize}
\item Then the probabilities in (7.11) and (7.12) must be identical, implying
\[
\left(1-\gamma + \frac{\gamma}{2}\right)(1-F(Y^I)) = \frac{\gamma}{2}(1-F(Y^E))
\]
\item When $\gamma >0$, this implies $F(Y^I) > F(Y^E)$, so $Y^I > Y^E$
\item More orders and greater execution probability in market $I$
\item Why? 
\pause
\textcolor{blue}{Market $I$ has an `routing advantage', meaning it is more attractive $\rightarrow$ more orders}
\end{itemize}
\end{itemize}
\end{frame}


\begin{frame}{Fragmentation benefits: comparing volumes}
We can relate $Y^I, Y^E$ and $Y$ as follows. (Recall $\bar{Y}=Y^I+Y^E$).
\begin{itemize}
\item  Equation (6.7) also holds with single platform. This implies $\mathbb{P}(q_I \geq Y^I) = \mathbb{P}(q \geq Y)$ 
\item By (7.11): $2\times \mathbb{P}(q_I \geq Y^I) =$ weighted avg. of $1-F(Y^I)$ and $1-F(\bar{Y})$
\item But with a single platform: $2 \times \mathbb{P}(q \geq Y)=1-F(Y)$
\item Hence:
\[
Y^I<Y<Y^I+Y^E = \bar{Y}.
\]
\begin{block}{}
\begin{center}
Fragmentation leads to greater aggregate volume $\bar{Y}$
\end{center}
\end{block}
\item Fragmented market is deeper than consolidated market
\end{itemize}
\end{frame}


\begin{frame}{Fragmentation benefits: conclusion}
\begin{itemize}
\item Previous slide: fragmentation has positive effect on market depth
\item Why? \textcolor{blue}{Fragmentation allows for more competition.} Fragmentation essentially allows limit orders to \textit{partially} sidestep time priority:
\begin{itemize}
\item When market is consolidated, first limit orders always get executed first
\item But in fragmented market, you can post limit order on another market at same price, and (maybe) get executed first
\end{itemize}
\item As argued when analyzing the pro-rata rule, removing time priority may lead to more orders, but these orders may appear more slowly
\item Result relies on $\Delta>0$ -- otherwise at $E$ you can undercut price at $I$ 
\item Section 7.4.3: there is a critical value of $\gamma$, below which $Y^E=0$
\begin{itemize}
\item Once again, a barrier to entry
\end{itemize}
\end{itemize}
\end{frame}



\section{Regulation}

\begin{frame}{Regulation: Introduction}
\begin{itemize}
\item Section 7.5 discusses two main regulatory regimes which seek to strike a balance among the costs and benefits of fragmentation
\item The two regimes are:
\begin{itemize}
\item \textbf{NMS}: `National Market System' for US equities
\item \textbf{MiFID}: `Market in Financial Instruments Directive' for EU equities
\end{itemize}
\item We focus here on NMS
\end{itemize}
\end{frame}


\begin{frame} \label{regulation}
\frametitle{Regulation: NMS}
 National Market System (NMS) for U.S. equities
\begin{itemize}
\item Overseen by the Securities and Exchage Comission (SEC) according to the regulation \textit{RegNMS}
\item Main elements:
\begin{itemize}
\item \textbf{Order protection}: Intermarket price priority to prevent trade-throughs (see next slides for definition)
\item \textbf{Access rule}: Access to data such as price quotations
\item \textbf{Sub-Penny Rule}: Minimum price increments of \$0.01 ($>\$1$) and \$0.0001($<\$1$)
\end{itemize}
\item Protects the best orders in each \textit{market center} from `trade-throughs' -- but definitions are subtle: Large orders need not get the best price
\end{itemize}
\end{frame}


\begin{frame}[label=protection]
\frametitle{NMS: Computing NBBO}
\begin{itemize}	
\item First, let's try to compute the National Best Bid and Offer (NBBO)
\includegraphics[trim= 0 150 0 170, width=0.7\paperwidth]{pics/NBBO3}
\item The  left-hand side records a series of quotes. On the right-hand side, we have updated the current quote 
\pause
\item At 9.35: best national bid is 70.00 and best national offer (ask) is 70.10
\end{itemize}
\end{frame}


\begin{frame}
\frametitle{NMS: Trade-through}
\begin{itemize}	
\item Recall NBBO:
\includegraphics[trim= 0 150 0 170, width=0.7\paperwidth]{pics/NBBO3}
\item If at 9.35 somebody executed a market sell order at exchange C at 69.90, this would constitute a \textit{trade through}
\pause
\item They could have gotten a better price at exchanges A or B
\end{itemize}
\end{frame}


\begin{frame}
\frametitle{NMS: Order protection}
\begin{itemize}	
\item RegNMS prevents trade-throughs of protected orders
\item \textcolor{blue}{Protected orders} must satisfy these criteria:
\begin{itemize}
\item Visible 
\item Best bid/offer of \textit{market center} (not necessarily NBBO)
\item Accessible (available for automatic execution)
\end{itemize}
\item Consider the following \textbf{bid books} 
\includegraphics[width=0.65\paperwidth]{pics/TwoMarket}
\item National best bid: 50.39 (bid of 50.40 not visible)
\item Both Alan, Beth and Ben's offers are protected
\end{itemize}
\end{frame}


\begin{frame}
\frametitle{NMS: Order protection (2)}
\begin{itemize}	
\item Consider the following \textbf{bid books} 
\includegraphics[width=0.7\paperwidth]{pics/TwoMarket}
\item Thus, market B cannot execute a market sell order of 100 shares against Beth's order - it would trade through Alan's protected order
\item In this case, market B might eigher (i) cancel the market order, (ii) enter it as a limit order, or (iii) redirect it to market A
\end{itemize}
\end{frame}


\begin{frame}
\frametitle{NMS: Sweep orders}
\begin{itemize}
\item Consider another example (protected orders in \textbf{bold})
\quad
\includegraphics[width=0.7\paperwidth]{pics/TwoMarket2}
\item RegNMS would allow the following order (a so-called \textit{sweep} order)
\begin{itemize}
\item Sell 400 shares to exchange A
\item Sell 100 shares to exchange B
\end{itemize}
\item This order respects all protected bids. But...
\end{itemize}
\end{frame}


\begin{frame}{Regulation: MiFID}
Market in Financial Instruments Directive (MiFID) for EU securities
\begin{itemize}
\item Broke national monopolies on trading
\item Main elements:
\begin{itemize}
\item \textbf{Best execution}: similar to order protection, but considers not only price, but also size, costs, speed/chance of execution
\item \textbf{Pre-trade transparency}: make price information (e.g. quotes) available
\item \textbf{Post-trade transparency}: publish price, volume and time of trades
\end{itemize}
\item Focus on `maximum harmonization'
\item $[$Read Fidessa on fragmentation in Europe$]$
\end{itemize}
\end{frame}


\begin{frame}{Conclusion}
\begin{itemize}
\item Fragmentation is ubiquitous 
\item Costs may include more adverse selection, less risk sharing and less competition
\item But there may also be benefits: depends on setting and trading format
\item Regulation targets price priority in particular, by trying to prevent trade throughs
\item But this is most effective for small order sizes
\end{itemize}
\end{frame}


\begin{frame}{Exercise for next week}
\begin{itemize}
\item In Absalon I have uploaded an article from the Economist December 7, 2013, on the Paris Bourse. Consider the passage ``Politicians in France routinely hit the roof when symbols of national virility risk falling into foreign hands, and Euronext is no exception.'' What could be the valid concerns of the politicians? Can it connect to chapter 7 in our textbook?
\item Solve exercise 3 on page 276 on brokers receiving payments for order flow
\end{itemize}
\end{frame}





\end{document} 