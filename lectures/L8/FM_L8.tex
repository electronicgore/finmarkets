%%% License: Creative Commons Attribution Share Alike 4.0 (see https://creativecommons.org/licenses/by-sa/4.0/)
%%% Slides are based heavily on earlier versions of this course taught by Jesper Rudiger.

\documentclass[english,10pt
%,handout
,aspectratio=169
]{beamer}
%%% License: Creative Commons Attribution Share Alike 4.0 (see https://creativecommons.org/licenses/by-sa/4.0/)
%%% Slides are based heavily on earlier versions of this course taught by Jesper Rudiger and Peter Norman Sorensen.

\DeclareGraphicsExtensions{.eps, .pdf,.png,.jpg,.mps,}
\usetheme{reMedian}
\usepackage{parskip}
\makeatother

\renewcommand{\baselinestretch}{1.1} 

\usepackage{amsmath, amssymb, amsfonts, amsthm}
\usepackage{enumerate}
\usepackage{hyperref}
\usepackage{url}
\usepackage{bbm}
\usepackage{color}

\usepackage{tikz}
\usepackage{tikzscale}
\newcommand*\circled[1]{\tikz[baseline=(char.base)]{
		\node[shape=circle,draw, inner sep=-20pt] (char) {#1};}}
\usetikzlibrary{automata,positioning}
\usetikzlibrary{decorations.pathreplacing}
\usepackage{pgfplots}
\usepgfplotslibrary{fillbetween}
\usepackage{graphicx}

\usepackage{setspace}
%\thinmuskip=1mu
%\medmuskip=1mu 
%\thickmuskip=1mu 


\usecolortheme{default}
\usepackage{verbatim}
\usepackage[normalem]{ulem}

\usepackage{apptools}
\AtAppendix{
	\setbeamertemplate{frame numbering}[none]
}
\usepackage{natbib}



\title{Financial Markets Microstructure \\ Lecture 8}

\subtitle{Market Fragmentation\\
	Chapter 7 of FPR}

\author{Egor Starkov}

\date{K{\o}benhavns Unversitet \\
	Spring 2020}


\begin{document}
\AtBeginSection[]{
\frame<beamer>{
\frametitle{This lecture:}
\tableofcontents[currentsection,currentsubsection]
}}
\frame[plain]{\titlepage}

%\section{Revision and problems}

\begin{frame}{Last week(s)}
	\begin{itemize}
		\item Models of order-driven markets
		\item (Glosten's model): limit traders provide liquidity in a way similar but not same to dealers
		\item (Parlour's model): LOB is resilient / self-replenishing
		\item Market design: measures directed at improving liquidity can backfire by distorting incentives
	\end{itemize}
\end{frame}


\begin{frame}{Today}
	\begin{itemize}
		\item Analyze fragmented markets: i.e. multiple markets selling the same asset
		\item Look at fragmentation costs and benefits
		\item Compare these across batch markets and LOB markets
		\item This will give us an opportunity to revisit some of the models we have looked at previously
		\item Finally, we will look at regulation
	\end{itemize}
\end{frame}



\section{Fragmentation intro}

\begin{frame}{History lesson}
	\begin{itemize}
		\item A listing in the US used to confer near-exclusive trading rights
		\begin{itemize}
			\item For instance, if a stock were NYSE-listed, almost all of the trading would occur on the NYSE
			\item Until 1999, most stock options were only traded on the exchange where they were listed
		\end{itemize}
		\item Many European countries used to require that stocks of their companies are only traded on local exchanges
		\item All of that has changed in the past 20 years
		\begin{itemize}
			\item Today, many (at least high-cap) stocks are \structure{cross-listed} on many exchanges
			\item Even if a stock is not listed on a given exchange, it can be \structure{admitted for trading}
		\end{itemize}
	\end{itemize}
\end{frame}


\begin{frame}{Fragmentation and consolidation}
	\begin{itemize}
		\item Refer to markets with multiple venues trading the same stock as \textit{fragmentated} (as opposed to consolidated)
		\item Regulation often tries to create `virtual consolidation': make fragmented markets act \textit{as if} they were consolidated
		\begin{itemize}
			\item A central concept in the US is the \textit{National Best Bid and Offer} (NBBO): the highest bid and lowest ask price in (any exchange within) the market at a given time
			\item US regulation requires that trade always takes place at the NBBO
		\end{itemize}
		\item EU's ban of countries' ``concentration rules'' aims for a different kind of consolidation -- one where all stocks can be traded on same exchange rather than spread around local exchanges.
		\begin{itemize}
			\item We're not going to talk much about this dimension.
		\end{itemize}
	\end{itemize}
\end{frame}


\begin{frame}{Priority violations}
	\begin{itemize}
		\item  LOB has different kinds of priority: price, visibility,  time
		\item In a fragmented market, no overall coordination mechanism, so the following may occur:
		\begin{itemize}
			\item Violation of \textbf{visibility priority}. An undisplayed order at a price of 100 might be executed on exchange $A$ even though there are quantities visible at 100 on exchange $B$
			\item Violation of \textbf{time priority}. A limit order to buy at a price of 100 that was entered at 10:00 AM on exchange $A$ might be filled before an order to buy at 100 that was entered at 9:30 AM on Exchange $B$
			\item Violation of \textbf{price priority}. A limit order to buy at a price 100 on exchange $A$ might be executed even though there is, at the same time, a limit order to buy at a price of 101 on exchange $B$. Known as a \textit{trade through}
		\end{itemize}
	\end{itemize}
\end{frame}


\begin{frame}{Other consequences}
	Apart from violating priority rules, fragmentation can lead to:
	\begin{itemize}
		\item \alert{Higher trading costs} (due to search costs or not finding best price)
		\item \alert{Worse price discovery} (information is more dispersed)
		\item \alert{Less total liquidity} (network effects)
	\end{itemize}
	\pause
	but also to:
	\begin{itemize}
		\item \structure{Lower trading costs} (due to competition among exchanges)
		\item \structure{Better price discovery} (due to aggregating many different signals instead of one)
		\item \structure{More total liquidity} (liq providers face less competition, so more willing to participate)
	\end{itemize}
	Will look at some of these and others in greater detail today
\end{frame}



\section{Fragmentation and trading costs}

\begin{frame}{Platform competition reduces trading costs}
	\begin{itemize}
		\item Competition among exchanges is a significant driver of trading fees and commissions
		\item Sometimes just the threat of entry is enough
		\item E.g. case of Dutch stocks:
		\begin{exampleblock}{}
			\begin{itemize}
				\item before 2003, most of Dutch stocks were traded on Euronext's NSC
				\item in May 2003, Deutsche B{\"o}rse and LSE separately announced they wanted to enter this market
				\item Euronext's \alert{order entry fee} was $0.3$ EUR in Jan 2004...
				\item ...halved to $0.15$ for limit orders on Apr 4, 2004...
				\item ...suspended for market orders on May 24, 2004 (LSE's EuroSETS launch day)...
				\item ...and finally waived completely for market and limit orders on Jan 31, 2005 together with reducing execution fees.
				\item (EuroSETS never really gained much traction)
			\end{itemize}
		\end{exampleblock}
	\end{itemize}
\end{frame}


\begin{frame}{Brokers' agency}
	\begin{itemize}
		\item Traders rarely place their orders directly -- they usually go to a broker (bank)
		\item When market is fragmented, brokers should ideally search all markets for the best price
		\begin{itemize}
			\item (Not a trivial problem, given the existence of dark pools, crossing networks of unknown depth, and hidden orders on standard exchanges)
		\end{itemize}
		\item But \alert{cost} of this search is incurred by \alert{broker}, while \structure{gains} are transmitted to \structure{trader}.
		\begin{itemize}
			\item So there is an agency problem
			\item Exacerbated by payments brokers can receive from exchanges for routing all or some orders through this exchange
		\end{itemize}
		\item Regulation (US) aims to solve this via
		\begin{itemize}
			\item ``best execution rules'' imposed on brokers
			\item no-trade-through/order-protection rules which say order must be routed automatically to trade at NBBO
		\end{itemize}
	\end{itemize}
\end{frame}



\section{Fragmentation in Kyle model}

\begin{frame}{Kyle model: refresher}
	To talk about liquidity and price impact, let us look at the effects of fragmentation within \alert{Kyle model}.
	
	Recall the baseline model from past lectures:
	\begin{itemize}
		\item Risky asset value $v \sim N(\mu,\sigma^{2}_{v})$
		\item \structure{Insider} observes $v$
		\begin{itemize}
			\item Places market order $x$; maximizes $\mathbb{E}[x(v-p)|x]$
		\end{itemize}
		\item \structure{Noise trader} demand $u \sim N(0, \sigma^{2}_{u})$
		\item \structure{Market maker} observes aggregate order flow $q = x + u$
		\begin{itemize}
			\item Competitively prices asset at $p = \mathbb{E}[v|q]$
			(Recall the difference here to what we did in the LOB model)
		\end{itemize}
	\end{itemize}
\end{frame}


\begin{frame}{Kyle model: refresher (2)}
	\begin{itemize}
		\item The insider uses a linear strategy $x=\beta(v-\mu)$
		\item Market makers observe $q=x+u=\beta(v-\mu) + u$. Then
		\[
		p=\mathbb{E}[v|q]=\mu + \lambda q
		\]
		\item Solving the MM's problem we get price-impact parameter
		\[
		\lambda = \frac{\beta \sigma^{2}_{v}}{\beta^{2} \sigma^{2}_{v}+\sigma^{2}_{u}}
		\]
		\item The insider takes for granted the pricing rule $p=\mu+\lambda q$
		\begin{itemize}
			\item Expected gain is $x(v-\mu-\lambda x)$ since $\mathbb{E}[u]=0$
		\end{itemize}
		\item Solving trader's problem gives $x=\beta(v-\mu)$ where $\beta=1/(2\lambda)$
	\end{itemize}
\end{frame}


\begin{frame}{Kyle model: refresher (3)}
	\begin{itemize}
		\item `Match the coefficients' $\frac{1}{2\beta} = \lambda = \frac{\beta \sigma^{2}_{v}}{\beta^{2} \sigma^{2}_{v}+\sigma^{2}_{u}}$ to get the equilibrium
		\[
		\frac{1}{\lambda} = 2 \beta = 2 \frac{\sigma_{u}}{\sigma_{v}}
		\]
		\item Insider equilibrium trading:
		\[
		x = \frac{\sigma_{u}}{\sigma_{v}} (v-\mu)
		\]
		\item MM makes zero profit:  insider's gain is noise traders' loss
		\[
		\underbrace{\mathbb{E}[x(p-v)]}_{\structure{\text{Insider gain}}} = \underbrace{\mathbb{E}[u(\lambda u)]}_{\structure{\text{Noise loss}}} = \lambda \sigma^{2}_{u} = \frac{\sigma_{v}\sigma_{u}}{2}
		\]
	\end{itemize}
\end{frame}


\begin{frame}{Kyle model with a fragmented market}
	Now fragment the market in this Kyle model. Suppose:
	\begin{enumerate}
		\item Two noise trader groups: $u=u_{1}+u_{2}$, $u_i$ independent and $u_{i} \sim N(0, \sigma^{2}_{ui})$
		\item Noise traders trade in separate markets; MMs and informed traders present in both markets (MM competitive; speculators one per market)
		\item Agents cannot choose their market
	\end{enumerate}
\end{frame}


\begin{frame}{Fragmented Kyle model: prices}
	\begin{itemize}
		\item \textbf{Prices}.  Since on each market $i=1,2$ we must have $p_i=\mathbb{E}[v|q]$,
			\begin{align*}
				p_i &= \mu + \lambda_i q_i = \mu + \lambda_i (x_i + u_i)
				\\
				&= \mu + \lambda_i (\beta_i(v-\mu) + u_i)
				\\
				&= \mu + \frac{\sigma_v}{2 \sigma_{ui}} \left( \frac{\sigma_{ui}}{\sigma_v} (v-\mu) + u_i \right)
				\\
				&= \mu + \frac{1}{2}(v-\mu)+\frac{\sigma_v}{2\sigma_{ui}}u_i
			\end{align*}
		
		\item \structure{On average, price is the same} in both markets. But in very short run prices may differ across markets
	\end{itemize}
\end{frame}


\begin{frame}{Fragmented Kyle model: volumes}
	\begin{itemize}
		\item \textbf{Trading}. Informed trading is given by $x_i= \beta_i (v-\mu)$ with $\beta_i= \frac{\sigma_{ui}}{\sigma_{v}}$:
		\[
			x_1+x_2=(v-\mu)\frac{\sigma_{u1}+\sigma_{u2}}{\sigma_v}>x
		\]
		
		\item \alert{Agg. informed trading greater than in consolidated case}
		
		\item To see this, note: 
		\begin{align*}
			\mathbb{V}(u) = \sigma_u^2 &= \sigma_{u1}^2 + \sigma_{u2}^2 = \mathbb{V}(u_1+u_2)
			\\
			\Rightarrow
			\sigma_u^2 &< (\sigma_{u1} + \sigma_{u2})^2
			\\
			\Rightarrow
			\sigma_u &< \sigma_{u1} + \sigma_{u2}
		\end{align*}
	\end{itemize}
\end{frame}


\begin{frame}{Fragmented Kyle model: profits}
	\begin{itemize}
		\item \textbf{Adverse selection costs}. Measure by loss of noise traders:
		\begin{itemize}
			\item The expected loss of group $i$ is $\sigma_{v} \sigma_{ui}/2$: Total loss is 
			\[
			\frac{\sigma_{v}(\sigma_{u1}+\sigma_{u2})}{2} > \frac{\sigma_{v} \sigma_{u}}{2}
			\]
			\item If two separate markets, \textit{greater adverse selection loss}
		\end{itemize}
		
		\item \alert{More informed trading gives greater adverse selection loss}
		
		\item Of course, greater profits for informed traders
		
		\item In the above, we assume one speculator per market. If two speculators would compete in the consolidated mkt, the effect is even stronger.
	\end{itemize}
\end{frame}


\begin{frame}{Fragmented Kyle model: depth}
	\begin{itemize}
		\item \textbf{Market depth}. \alert{Depth in each market lower than in consolidated market}:
		\[
		\lambda = \frac{\sigma_v}{2 \sigma_u} < \min \{ \lambda_1, \lambda_2 \}
		\]
		
		\pause
		\item What about \structure{aggregate}? Say you want to trade $X$ total.
		\begin{itemize}
			\item When splitting optimally across two markets $X=x_1+x_2$, will equalize marginal prices: $p_1 = \mu + \lambda_1 x_1 = \mu + \lambda_2 x_2 = p_2$.
			\item So $x_1 = \frac{\lambda_2}{\lambda_1+\lambda_2} X$ and $x_2 = \frac{\lambda_1}{\lambda_1+\lambda_2} X$ and trading moves prices in each market to $p_i = \mu + \frac{\lambda_1 \lambda_2}{\lambda_1 + \lambda_2} X$ compared to $p = \mu + \lambda X$ in consolidated market. Can verify that $\frac{\lambda_1 \lambda_2}{\lambda_1 + \lambda_2} < \lambda$
			\item \structure{Fragmented market is deeper in aggregate}
		\end{itemize}
	\end{itemize}
\end{frame}


\begin{frame}{Fragmented Kyle model: price discovery}
	\begin{itemize}
		\item \textbf{Price discovery}. We used the fact that everything is normal and $\mathbb{E}[q]=0$ to obtain (see L5)
		$$
			\mathbb{E}[v|q] = \mu +\frac{\mathbb{C}(v,q)}{\mathbb{V}(q)} q
		$$
		\begin{itemize}
			\item This still works in vector form with $\vec{q}=(q_1,q_2)$
			\item ...or with $\vec{p}=(p_1,p_2)$, since $p_i = \mu + \lambda_i q_i$
		\end{itemize}
		\pause
		\item Doing the magic, get $\mathbb{E}[v|\vec{p}] = \mu + \frac{2}{3}(p_1 - \mu) + \frac{2}{3}(p_2 - \mu)$. With more magic can find $\mathbb{V}(v|\vec{p}) = \frac{\sigma^2_v}{3}$, while in consolidated market $\mathbb{V}(v|p) = \frac{\sigma^2_v}{2}$
		\item \structure{Better price discovery in fragmented market}
		\item Because have more signals from which to learn about $v$ 
		\pause 
		\begin{itemize}
			\item although comparing ``two speculators in two mkts'' vs ``one speculator in one mkt''. Two competing insiders in one market would yield \alert{same price discovery}
		\end{itemize}
	\end{itemize}
\end{frame}


\begin{frame}
	More details for the previous slide:
	\begin{align*}
		\mathbb{E}[v|p] &= \mathbb{E}[v] + (\mathbb{C}(v,p_1), \mathbb{C}(v,p_2)) \cdot \mathbb{V}^{-1}(p) \cdot \left(\vec{p} - \mathbb{E}[\vec{p}] \right)^T;
		\\
		\mathbb{V}(v|p) &= \mathbb{V}(v) - \mathbb{C}(v,p) \cdot \mathbb{V}^{-1}(p) \cdot \mathbb{C}^T(v,p)
		\\
		p_i &= \mu + \lambda_i \beta_i (v-\mu) + \lambda_i u_i; \qquad \beta_i = \frac{1}{2\lambda_i} = \frac{\sigma_{ui}}{\sigma_v};
		\\
		\mathbb{C}(v,p_i) &= \lambda_i \beta_i \sigma_v^2 = \frac{\sigma_v^2}{2}
		\\
		\mathbb{V}(p) &= \left(
			\begin{array}{c c}
				\mathbb{V}(p_1)		& \mathbb{C}(p_1,p_2)
				\\
				\mathbb{C}(p_1,p_2)	& \mathbb{V}(p_2)	
			\end{array}
		\right) = \left(
			\begin{array}{c c}
				\lambda_1^2 (\beta_1^2\sigma_v^2 + \sigma_{u1}^2)	& \lambda_1 \lambda_2 \beta_1 \beta_2 \sigma_v^2
				\\
				\lambda_1 \lambda_2 \beta_1 \beta_2 \sigma_v^2	& \lambda_2^2 (\beta_2^2\sigma_v^2 + \sigma_{u2}^2)
			\end{array}
		\right)
		\\
		&= \frac{\sigma_v^2}{4} \left(
			\begin{array}{c c}
				2	& 1
				\\
				1	& 2
			\end{array}
		\right)
		\\
		\Rightarrow \mathbb{V}^{-1}(p) &= \frac{4}{3 \sigma_v^2} \left(
			\begin{array}{c c}
				2	& -1
				\\
				-1	& 2
			\end{array}
		\right)
	\end{align*}
\end{frame}


\begin{frame}{Fragmented Kyle model: liquidity provision}
	\begin{itemize}
		\item In the analysis above, traders were chain-bound to their markets
		\item If speculators could choose, they would split across different markets to avoid competing with each other.
		\begin{itemize}
			\item If cannot commit to stay away from each other's market then fragmentation has little effect.
		\end{itemize}
		\item What if noise traders can choose?
		\begin{itemize}
			\item They would go to a deeper market. Depth $\frac{1}{\lambda_i} = \frac{2 \sigma_{ui}}{\sigma_v}$.
			\item Larger markets would get larger. \structure{Liquidity begets liquidity.}
			\item This is a natural barrier to entry for new trading platforms and source of ``payments for order flow'' we mentioned
		\end{itemize}
	\end{itemize}
\end{frame}


\begin{frame}{Conclusion: fragmented Kyle model}
	\textbf{Adverse selection costs}
	\begin{itemize}
		\item Fragmentation is bad for noise traders (welfare)
		\item If noise traders coordinate, it can be stable that they gather around a less efficient platform
		\item Fragmentation may create extra depth
	\end{itemize}
	\textbf{Aside}
	\begin{itemize}
		\item Note that ``fragmentation'' may also mean time
		\item Above observations may explain why trade volume is often concentrated at specific times of day (early and late)
		\item Also an argument for batch trading versus continuous trading
	\end{itemize}
\end{frame}



\section{Fragmentation in other models}

\begin{frame}{Stoll model and risk sharing}
	We now turn to the effect fragmentation on market makers' risk sharing
	\begin{itemize}
		\item Section 7.2.2: Analysis drawn on Stoll's model
		\item Risky asset value $v \sim N(\mu, \sigma^{2}_{v})$
		\item Risk averse dealer with mean-variance preference
		\begin{itemize}
			\item Initial asset position $z$; risk aversion $\rho$
			\item Asset supply is
			\[
			y = z + \frac{p-\mu}{\rho \sigma^{2}_{v}}
			\]
			\item Invert to get pricing schedule $p=m+\lambda q$ where $m=\mu-\lambda z$ and $\lambda = \rho \sigma^{2}_{v}$
		\end{itemize}
		\item \textbf{Trading cost}. Trading cost of order of size $q$ is $\lambda q^{2}$
	\end{itemize}
\end{frame}


\begin{frame}{Stoll model and risk sharing (2)}
	\begin{itemize}
		\item Suppose there are two dealers: risk aversions $\rho_{1}$ and $\rho_{2}$
		\begin{itemize}
			\item If consolidated trading, dealers' aggregate supply curve is
			\[
			y_{1}+y_{2} = z_{1}+z_{2}+\frac{p-\mu}{\rho_{1}\sigma^{2}_{v}}+\frac{p-\mu}{\rho_{2} \sigma^{2}_{v}} =  z_{1}+z_{2}+\left(\frac{1}{\rho_1}+\frac{1}{\rho_2}\right)\frac{p-\mu}{\sigma^{2}_{v}}
			\]
			\item This is as if trading with one dealer with risk aversion $\bar{\rho}$:
			\[
			\frac{1}{\bar{\rho}}= \frac{1}{\rho_{1}} + \frac{1}{\rho_{2}}
			\]
		\end{itemize}
	\end{itemize}
	\begin{block}{}
		\begin{itemize}
			\item Risk averse dealers share risks in the consolidated market: $\bar{\rho}<\rho_{i}$
			\item Trading cost $\bar{\rho}\sigma^{2}_{v}q^{2}$ is lower than either of $\rho_{i}\sigma^{2}_{v}q^{2}$
		\end{itemize}
	\end{block}
\end{frame}


\begin{frame}{Stoll model and risk sharing (3)}
	\begin{itemize}
		\item The above assumes again that both dealers and traders are constrained to one market.
		\item Otherwise perfect insurance is possible across dealers and fragmentation has no effect.
	\end{itemize}
\end{frame}


\begin{frame}{Glosten model}
	Competing limit order books may provide better aggregate depth
	\begin{itemize}
		\item To show this, use LOB model of section 6.2 with display cost $C>0$, tick size $\Delta >0$, no adverse selection (asset value is $\mu$)
		\item Recall: incoming order $q$, limit sell orders posted at cumulative quantity $Y$ and price $A$ satisfy the zero-profit condition
		\[
		0 = \mathbb{P}(q \geq Y)[A-\mu]-C,
		\]
		solved by 
		\begin{equation} \tag{6.7}
		A= \mu + \frac{C}{\mathbb{P}(q \geq Y)}.
		\end{equation}
	\end{itemize}
\end{frame}


\begin{frame}{Glosten model (2)}
	Now we make the following assumptions:
	\begin{itemize}
		\item \textbf{Fragmentation}. Limit orders are supplied in two markets, $I$ and $E$
		\begin{itemize}
			\item At single available ask price $A$, denote cumulative limit sell orders by $Y^I$ and $Y^E$
		\end{itemize}
		\item \textbf{Market orders}. 
		\begin{itemize}
			\item Market order is $Buy$ with chance 1/2, and of size $q \sim F(q)$
			\item With probability $1-\gamma$, the entire order goes to $I$
			\item With probability $\gamma$, the incoming order is split:
			\begin{itemize}
				\item With prob. 1/2, order first goes to $I$, whatever remains goes to $E$
				\item With prob. 1/2, order first goes to $E$, whatever remains goes to $I$
			\end{itemize}
		\end{itemize}
		\item  \textbf{Trade probability}. We need to model $\mathbb{P}(q_i \geq Y^i)$ for each market
	\end{itemize}
\end{frame}


\begin{frame}{Glosten model: trading probabilities}
	\begin{itemize}
		\item This results in the following execution probabilities
		\begin{align}
		\mathbb{P}(q_I \geq Y^I) & = \frac{1}{2} \left[\underbrace{\left(1-\gamma+\frac{\gamma}{2} \right) (1-F(Y^I))}_{\structure{\text{$I$ is executed first}}} + \underbrace{\frac{\gamma}{2}(1-F(\bar{Y}))}_{\structure{\text{$I$ is executed second}}}\right] \tag{7.11} \\
		\mathbb{P}(q_E \geq Y^E) & = \frac{1}{2} \left[\underbrace{\frac{\gamma}{2} (1-F(Y^E))}_{\structure{\text{$E$ is executed first}}} + \underbrace{\frac{\gamma}{2}(1-F(\bar{Y}))}_{\structure{\text{$E$ is executed second}}}\right] \tag{7.12}
		\end{align}
		where $\bar{Y}=Y^I+Y^E$
	\end{itemize}
\end{frame}


\begin{frame}{Glosten model: equilibrium}
	\begin{itemize}
		\item If both markets active ($Y^I>0$ and $Y^E>0$) then (6.7) holds for both
		\item Then the probabilities in (7.11) and (7.12) must be identical, implying
		\[
		\left(1-\gamma + \frac{\gamma}{2}\right)(1-F(Y^I)) = \frac{\gamma}{2}(1-F(Y^E))
		\]
		\item When $\gamma < 1$, this implies $F(Y^I) > F(Y^E)$, so $Y^I > Y^E$
		\item More orders and greater execution probability in market $I$
		\item Why? 
		\pause
		\structure{Market $I$ has a `routing advantage', meaning it is more attractive $\rightarrow$ more orders}
	\end{itemize}
\end{frame}


\begin{frame}{Glosten model: comparing volumes}
	We can relate $Y^I, Y^E$ and $Y$ as follows. (Recall $\bar{Y}=Y^I+Y^E$).
	\begin{itemize}
		\item  Equation (6.7) also holds with single platform. This implies $\mathbb{P}(q_I \geq Y^I) = \mathbb{P}(q \geq Y)$ 
		\item By (7.11): $2\times \mathbb{P}(q_I \geq Y^I) =$ weighted avg. of $1-F(Y^I)$ and $1-F(\bar{Y})$
		\item But with a single platform: $2 \times \mathbb{P}(q \geq Y)=1-F(Y)$
		\item Hence:
		\[
		Y^I<Y<Y^I+Y^E = \bar{Y}.
		\]
		\begin{block}{}
			\begin{center}
				Fragmentation leads to greater aggregate volume $\bar{Y}$
			\end{center}
		\end{block}
		\item Fragmented market is deeper than consolidated market
	\end{itemize}
\end{frame}


\begin{frame}{Glosten model: conclusion}
	\begin{itemize}
		\item Previous slide: fragmentation has positive effect on market depth
		\item Why? \structure{Fragmentation allows for more competition.} Fragmentation essentially allows limit orders to \textit{partially} sidestep time priority:
		\begin{itemize}
			\item When market is consolidated, first limit orders always get executed first
			\item But in fragmented market, you can post limit order on another market at same price, and (maybe) get executed first
		\end{itemize}
		\item As argued when analyzing the pro-rata rule, removing time priority may lead to more orders, but these orders may appear more slowly
		\item Result relies on $\Delta>0$ -- otherwise at $E$ you can undercut price at $I$ 
		\item Section 7.4.3: there is a critical value of $\gamma$, below which $Y^E=0$
		\begin{itemize}
			\item Once again, a barrier to entry
		\end{itemize}
	\end{itemize}
\end{frame}



%\section{Regulation}
%
%\begin{frame}{Regulation: Introduction}
%\begin{itemize}
%\item Section 7.5 discusses two main regulatory regimes which seek to strike a balance among the costs and benefits of fragmentation
%\item The two regimes are:
%\begin{itemize}
%\item \textbf{NMS}: `National Market System' for US equities
%\item \textbf{MiFID}: `Market in Financial Instruments Directive' for EU equities
%\end{itemize}
%\item We focus here on NMS
%\end{itemize}
%\end{frame}
%
%
%\begin{frame} \label{regulation}
%\frametitle{Regulation: NMS}
% National Market System (NMS) for U.S. equities
%\begin{itemize}
%\item Overseen by the Securities and Exchage Comission (SEC) according to the regulation \textit{RegNMS}
%\item Main elements:
%\begin{itemize}
%\item \textbf{Order protection}: Intermarket price priority to prevent trade-throughs (see next slides for definition)
%\item \textbf{Access rule}: Access to data such as price quotations
%\item \textbf{Sub-Penny Rule}: Minimum price increments of \$0.01 ($>\$1$) and \$0.0001($<\$1$)
%\end{itemize}
%\item Protects the best orders in each \textit{market center} from `trade-throughs' -- but definitions are subtle: Large orders need not get the best price
%\end{itemize}
%\end{frame}
%
%%\item \structure{Protected orders} must satisfy these criteria:
%%\begin{itemize}
%%	\item Visible 
%%	\item Best bid/offer of \textit{market center} (not necessarily NBBO)
%%	\item Accessible (available for automatic execution)
%%\end{itemize}
%
%
%
%
%
%\begin{frame}
%\frametitle{NMS: Sweep orders}
%\begin{itemize}
%\item Consider another example (protected orders in \textbf{bold})
%\quad
%\includegraphics[width=0.7\paperwidth]{pics/TwoMarket2}
%\item RegNMS would allow the following order (a so-called \textit{sweep} order)
%\begin{itemize}
%\item Sell 400 shares to exchange A
%\item Sell 100 shares to exchange B
%\end{itemize}
%\item This order respects all protected bids. But...
%\end{itemize}
%\end{frame}
%
%
%\begin{frame}{Regulation: MiFID}
%	Market in Financial Instruments Directive (MiFID) for EU securities
%	\begin{itemize}
%		\item Broke national monopolies on trading
%		\item Main elements:
%		\begin{itemize}
%			\item \textbf{Best execution}: similar to order protection, but considers not only price, but also size, costs, speed/chance of execution
%			\item \textbf{Pre-trade transparency}: make price information (e.g. quotes) available
%			\item \textbf{Post-trade transparency}: publish price, volume and time of trades
%		\end{itemize}
%		\item Focus on `maximum harmonization'
%		\item $[$Read Fidessa on fragmentation in Europe$]$
%	\end{itemize}
%\end{frame}


\begin{frame}{Conclusion}
	\begin{itemize}
		\item Fragmentation is ubiquitous 
		\item Costs may include more adverse selection, less risk sharing and less competition among traders
		\item But there may also be benefits, depends on setting and trading format
		\item Regulation targets price priority in particular, by trying to prevent trade throughs 
		\item But this is only really effective for small order sizes
	\end{itemize}
\end{frame}


\begin{frame}{Exercises}
	\begin{itemize}
		\item Read two articles on absalon related to today's topics (Economist on algortihms and Reuters on dark markets)
		\item Solve exercise 3 on page 276 (ch.7) on brokers receiving payments for order flow
	\end{itemize}
\end{frame}





\end{document} 