%%% License: Creative Commons Attribution Share Alike 4.0 (see https://creativecommons.org/licenses/by-sa/4.0/)
%%% Slides are based heavily on earlier versions of this course taught by Jesper Rudiger.

\documentclass[english,10pt]{beamer}
%\documentclass[english,10pt,handout]{beamer}
%%% License: Creative Commons Attribution Share Alike 4.0 (see https://creativecommons.org/licenses/by-sa/4.0/)
%%% Slides are based heavily on earlier versions of this course taught by Jesper Rudiger and Peter Norman Sorensen.

\DeclareGraphicsExtensions{.eps, .pdf,.png,.jpg,.mps,}
\usetheme{reMedian}
\usepackage{parskip}
\makeatother

\renewcommand{\baselinestretch}{1.1} 

\usepackage{amsmath, amssymb, amsfonts, amsthm}
\usepackage{enumerate}
\usepackage{hyperref}
\usepackage{url}
\usepackage{bbm}
\usepackage{color}

\usepackage{tikz}
\usepackage{tikzscale}
\newcommand*\circled[1]{\tikz[baseline=(char.base)]{
		\node[shape=circle,draw, inner sep=-20pt] (char) {#1};}}
\usetikzlibrary{automata,positioning}
\usetikzlibrary{decorations.pathreplacing}
\usepackage{pgfplots}
\usepgfplotslibrary{fillbetween}
\usepackage{graphicx}

\usepackage{setspace}
%\thinmuskip=1mu
%\medmuskip=1mu 
%\thickmuskip=1mu 


\usecolortheme{default}
\usepackage{verbatim}
\usepackage[normalem]{ulem}

\usepackage{apptools}
\AtAppendix{
	\setbeamertemplate{frame numbering}[none]
}
\usepackage{natbib}




\title{Financial Markets Microstructure \\ Exercise class 2}

\author{Egor Starkov}

\date{K{\o}benhavns Unversitet \\
	Spring 2020}



\begin{document}
\frame[plain]{\titlepage}
\addtocounter{framenumber}{-1}



\section{Glosten-Milgrom model}

\begin{frame}{Lecture 3}
	\begin{itemize}
		\item FPR chapter 3 exercises (p. 124-125):
		\begin{itemize}
			\item exercise 2 -- solve GM model with numbers instead of letters
			\item exercise 3 about the GM model where speculators are not perfectly informed, but instead receive a signal about the value of the asset
		\end{itemize}
		\item Bonus problem: GM model with uniform $v$
	\end{itemize}
\end{frame}




\subsection{ex 2}

\begin{frame}{Exercise 2}
	A small risky company's stock is worth either \$10 ($v_L$) or \$20 ($v_H$) with probability 1/2 each ($\theta=1-\theta=1/2$). 
	\begin{enumerate}[a]
		\small 
		\item Compute  bid and ask prices set by risk neural competitive market makers in the absence of informed trading.
		\item Compute bid and ask prices set by risk neutral competitive market makers when they expect 1/10 of trade initiators to be informed (know the stock's true value) and to trade as profit maximizers, while the other 9/10 are uninformed and buy or sell with \textit{equal probability}. Assume  all transactions are the same size.
		\item Compute the average trading cost to an uninformed trader and the average gain to an informed one, assuming a unit trade size in both cases.
		\item Do you agree with the following statement? ``Insider trading does not harm most market participants: it harms only those who are unlucky enough to trade with an insider.''
	\end{enumerate}
\end{frame}


\begin{frame}{Exercise 2.a}
	\begin{exampleblock}{}
		(a) Compute  bid and ask prices set by risk neural competitive market makers in the absence of informed trading.
	\end{exampleblock}
	
	\begin{align*}
		a &= \mathbb{E}[v | \Omega, Buy]
		\\
		\visible<2->{
			&= \frac{\frac{1}{2} \beta_B 20 + \frac{1}{2} \beta_B 10}{\frac{1}{2} \beta_B + \frac{1}{2} \beta_B} = 15
		}
	\end{align*}
	\begin{align*}
		b &= \mathbb{E}[v | \Omega, Sell]
		\\
		\visible<2->{
			&= \frac{\frac{1}{2} \beta_B 20 + \frac{1}{2} \beta_B 10}{\frac{1}{2} \beta_B + \frac{1}{2} \beta_B} = 15
		}
	\end{align*}
\end{frame}


\begin{frame}{Exercise 2.b}
	\begin{exampleblock}{}
		(b) Compute bid and ask prices set by risk neutral competitive market makers when they expect 1/10 of trade initiators to be informed (know the stock's true value) and to trade as profit maximizers, while the other 9/10 are uninformed and buy or sell with equal probability. Assume  all transactions are the same size.
	\end{exampleblock}
	\vspace{-1em}
	\begin{align*}
		a &= \mathbb{E}[v | \Omega, Buy]
		\\
		\visible<2->{
			&= \frac{\frac{1}{2} \left(0.1 + 0.9\cdot\frac{1}{2}\right) 20 + \frac{1}{2} \left(0.9 \cdot \frac{1}{2}\right) 10}{\frac{1}{2} \left(0.1 + 0.9\cdot\frac{1}{2}\right) + \frac{1}{2} \left(0.9 \cdot \frac{1}{2}\right)} = 15.5
		}
	\end{align*}
	\begin{align*}
		b &= \mathbb{E}[v | \Omega, Sell]
		\\
		\visible<2->{
			&= \frac{\frac{1}{2} \left(0.9 \cdot \frac{1}{2}\right) 20 + \frac{1}{2} \left(0.9\cdot\frac{1}{2} + 0.1\right) 10}{\frac{1}{2} \left(0.9 \cdot \frac{1}{2}\right) + \frac{1}{2} \left(0.9\cdot\frac{1}{2} + 0.1\right)} = 14.5
		}
	\end{align*}
\end{frame}


\begin{frame}{Exercise 2.c}
	\begin{exampleblock}{}
		(c) Compute the average trading cost to an uninformed trader and the average gain to an informed one, assuming a unit trade size in both cases.
	\end{exampleblock}
	\pause
	\begin{align*}
		\mathbb{E} \Pi_U = \frac{1}{2} (15-15.5) + \frac{1}{2} (14.5-15) = -0.5
	\end{align*}
	\begin{align*}
		\mathbb{E} \Pi_I = \frac{1}{2} (20 - 15.5) + \frac{1}{2} (14.5 - 10) = 4.5
	\end{align*}
	(note that $\frac{1}{2}-\frac{1}{2}$ probabilities relate to different events in the two expressions)
\end{frame}


\begin{frame}{Exercise 2.d}
	\begin{exampleblock}{}
		(d) Do you agree with the following statement? 
		\begin{quotation}
			\smallskip
			``Insider trading does not harm most market participants: it harms only those who are unlucky enough to trade with an insider.''
		\end{quotation}
	\end{exampleblock}
	\pause
	No. Uninformed traders pay the cost arising from informed trading, even if they do not trade directly with the informed traders.
\end{frame}




\subsection{ex 3}

\begin{frame}{Exercise 3}
	\begin{itemize}
		\item $v \in \{v^H,v^L\}$ w.p. $\frac{1}{2}$;
		\item dealer competitive, risk-neutral, does not know $v$;
		\item trader uninformed w.p. $1-\pi$, then buys and sells w.p. $\frac{1}{2}$;
		\item trader informed w.p. $\pi$, then observes \structure{signal about state}
		\begin{itemize}
			\item signal accurate w.p. $\rho \in \left(\frac{1}{2},1\right]$.
		\end{itemize}
	\end{itemize}
\end{frame}


\begin{frame}{Exercise 3.a}
	\begin{exampleblock}{}
		(a) Write dealer's expected profits upon receiving buy/sell order, assuming informed trader follows his signal.
	\end{exampleblock}
	\begin{align*}
		\mathbb{E} [\Pi_D | Buy] &= 
		\visible<2->{
			\frac{ \pi \frac{1}{2} \rho (a-v^H) + \pi \frac{1}{2} (1-\rho) (a-v^L) + (1-\pi) \frac{1}{2} (a-\mu)}{\pi \frac{1}{2} \rho + \pi \frac{1}{2} (1-\rho) + (1-\pi) \frac{1}{2} }
			\\
			&= \pi \rho (a - v^H) + \pi (1-\rho) (a-v^L) + (1-\pi) (a - \mu)
		}
	\end{align*}
	\begin{align*}
		\mathbb{E} [\Pi_D | Sell] &= 
		\visible<2->{
			\frac{ \pi \frac{1}{2} \rho (v^L-b) + \pi \frac{1}{2} (1-\rho) (v^H-b) + (1-\pi) \frac{1}{2} (\mu-b)}{\pi \frac{1}{2} \rho + \pi \frac{1}{2} (1-\rho) + (1-\pi) \frac{1}{2} }
			\\
			&= \pi \rho (v^L-b) + \pi (1-\rho) (v^H-b) + (1-\pi) (\mu - b)
		}
	\end{align*}
	\visible<2->{where $\mu = \frac{v^H+v^L}{2}$.}
\end{frame}


\begin{frame}{Exercise 3.b}
	\begin{exampleblock}{}
		(b) Compute the bid and ask prices set by a risk-neutral competitive dealer.
	\end{exampleblock}
	\begin{align*}
		a &= 
		\visible<2->{\pi \rho v^H + \pi (1-\rho) v^L + (1-\pi) \mu
		}
	\end{align*}
	\begin{align*}
		b &= 
		\visible<2->{\pi \rho v^L + \pi (1-\rho) v^H + (1-\pi) \mu
		}
	\end{align*}
	\visible<2->{from $\mathbb{E} [\Pi_D | Buy] = \mathbb{E} [\Pi_D | Sell] = 0$.}
\end{frame}


\begin{frame}{Exercise 3.c}
	\begin{exampleblock}{}
		(c) Derive bid-ask spread as a function of signal's informativeness $\rho$. When is the market more or less liquid? Why?
	\end{exampleblock}
	\begin{align*}
		S &= a - b
		\\
		\visible<2->{
			&= \pi (2\rho - 1) (v^H - v^L)
		}
	\end{align*}
	\visible<2->{Higher $\rho$ $\Leftrightarrow$ larger $S$ $\Leftrightarrow$ less liquid market because of larger adverse selection costs (same as larger $\pi$).}
\end{frame}


\begin{frame}{Exercise 3.d}
	\begin{exampleblock}{}
		(d) Verify that the speculator's strategy (buy after signal $H$, sell after signal $L$) is optimal.
	\end{exampleblock}
	Consider signal $H$:
	\begin{align*}
		\mathbb{E} [\Pi_I | H, Buy] &= 
		\visible<2->{
			\rho (v^H - a) + (1-\rho) (v^L - a)
			\\
			&= \left( \rho v^H + (1-\rho) v^L - \mu \right) (1-\pi) > 0
		}
	\end{align*}
	\begin{align*}
		\mathbb{E} [\Pi_I | H, Sell] &= 
		\visible<2->{
			\rho (b - v^H) + (1-\rho) (b - v^L)
			\\
			&= (1-\pi) \left[ \mu - (\rho v^H + (1-\rho) v^L) \right] - \pi (2\rho - 1) (v^H - v^L) < 0
		}
	\end{align*}
	Same for signal $L$.
\end{frame}




\subsection{GM with Uniform value}

\begin{frame}{GM with Uniform value}
	\begin{itemize}
		\item \textbf{Uniform outcome}: Suppose $v$ is uniformly distributed on [0,1]
		\item \textbf{Prior value}: Prior density $f(v) = 1$, and $\mu = \mathbb{E}[v] = \int_0^1 v f(v) dv= 1/2$
		\item Look for an ask price $a < 1$:
		\begin{itemize}
			\item Speculator buys if  $v>a$ and sells if $v<b$ ($v=a$ has zero prob.). Thus:
			\begin{equation*}
			\mathbb{P}(Buy|v) = 
			\left\{
			\begin{aligned}
			(1-\pi) \beta_B + \pi 	&\text{ if } v > a; \\
			(1-\pi) \beta_B 		&\text{ if } v<a.
			\end{aligned}
			\right.
			\end{equation*}
			\item Recall Bayes' Rule: $f(v|  Buy) = \frac{f(v) \mathbb{P}(Buy| v)} {\mathbb{P}(Buy)}$. Then,
			\begin{equation*}
			f(v|  Buy)=\left\{
			\begin{aligned}
			&\frac{(1-\pi)\beta_B + \pi} {(1-\pi)\beta_B + (1-a)\pi}	&& \text{ if } v>a; \\
			&\frac{(1-\pi)\beta_B} {(1-\pi)\beta_B + (1-a)\pi}		&& \text{ if } v<a.
			\end{aligned}
			\right.
			\end{equation*}
		\end{itemize}
	\end{itemize}
\end{frame}


\begin{frame}{GM with Uniform value (2)}
	Now we explicity have $a$ on both sides of $a=\mathbb{E}[v|Buy]$. Must solve:
	\begin{align*}
	a 
	& = \mathbb{E}[v|Buy] \\
	& = \int^{a}_0 \frac{(1-\pi) \beta_B \cdot \structure{v}}{(1-\pi)\beta_B + (1-a) \pi} \, dv + \int^{1}_{a} \frac{[(1-\pi) \beta_B + \pi] \cdot \alert{v}}{(1-\pi)\beta_B + (1-a) \pi} \, dv \\
	& =  \frac{(1-\pi) \beta_B \cdot \structure{a^2/2}}{(1-\pi)\beta_B + (1-a) \pi}+  \frac{[(1-\pi) \beta_B + \pi] \cdot \alert{(1-a^2)/2}}{(1-\pi)\beta_B + (1-a) \pi} \\
	& =  \frac{(1-\pi) \beta_B + \pi -\pi a^2}{2(1-\pi)\beta_B + 2(1-a) \pi} = \frac{1}{2} + \frac{\pi a (1 - a)}{2(1-\pi)\beta_B + 2(1-a) \pi}
	\end{align*}
	This is a quadratic equation in $a$. For $\pi=\beta_B=1/2$, solution is $a=\frac{3}{2} \pm \frac{\sqrt{3}}{2}$. Since we assumed $a<1$, equilibrium in this case  is $a=\frac{3}{2} - \frac{\sqrt{3}}{2}$.
\end{frame}




\section{Trading costs and Inventory risk}
\subsection{ex 9}

\begin{frame}{Lecture 4}
	\begin{itemize}
		\item If you are bored, solve \hyperlink{ex9}{Exercise 9} on page 128 about the bid-ask spread in the mean-SD model
		\item If you feel like a challenge, solve \hyperlink{ex10}{Exercise 10} on page 128
	\end{itemize}
\end{frame}


\begin{frame}[label=ex9]{Exercise 9}
	\begin{itemize}
		\item Two-period model with inventory risk;
		\item dealer's utility: $U = \mathbb{E}_t [w_{t+1}] - \rho \sqrt{\mathbb{V}(w_{t+1})}$;
		\item dealer can receive a buy order $y_t > 0$ or sell order $-y_t < 0$ (here $y_t$ is some fixed amount);
		\item dealer has initial endowment $z_t$.
	\end{itemize}
\end{frame}


\begin{frame}{Exercise 9.a}
	\begin{exampleblock}{}
		(a) Compute the bid-ask spread as a function of $y_t$ when the dealer has a long initial position ($z_t > 0$)
	\end{exampleblock}
	\pause
	\begin{itemize}
		\item From lecture: the dealer's optimal supply is
		\begin{itemize}
			\item Supply any $y_t \leq z_t$ if $p_t = \mu_t - \rho\sigma_{\epsilon}$,
			\item Supply any $y_t \geq z_t$ if $p_t = \mu_t + \rho\sigma_{\epsilon}$
		\end{itemize}
		\item I.e., the ask price is
		\begin{align*}
			a_t = 
			\begin{cases}
				\mu_t - \rho\sigma_{\epsilon} & \text{ if } y_t \leq z_t;
				\\
				\mu_t + \rho\sigma_{\epsilon} & \text{ if } y_t \geq z_t;
			\end{cases}
		\end{align*}
		and the bid price is $b_t = \mu_t - \rho\sigma_{\epsilon}$ for any $y_t$.
	\end{itemize}
\end{frame}


\begin{frame}{Exercise 9.a}
	\begin{exampleblock}{}
		(a) Compute the bid-ask spread as a function of $y_t$ when the dealer has a long initial position ($z_t > 0$)
	\end{exampleblock}
	\begin{itemize}
		\item So spread is
		\begin{align*}
			S_t = 
			\begin{cases}
				0 & \text{ if } y_t \leq z_t;
				\\
				2\rho\sigma_{\epsilon} & \text{ if } y_t \geq z_t;
			\end{cases}
		\end{align*}
	\end{itemize}
\end{frame}


\begin{frame}{Exercise 9.b}
\begin{exampleblock}{}
	(b) Compute the bid-ask spread as a function of $y_t$ when the dealer has a short initial position ($z_t < 0$)
\end{exampleblock}
\begin{itemize}
	\item Similar to above:
	\begin{align*}
		S_t = 
		\begin{cases}
		0 & \text{ if } y_t \leq |z_t|;
		\\
		2\rho\sigma_{\epsilon} & \text{ if } y_t \geq |z_t|;
		\end{cases}
	\end{align*}
\end{itemize}
\end{frame}


\begin{frame}{Exercise 9.c}
	\begin{exampleblock}{}
		(c) Represent these two cases graphically
	\end{exampleblock}
	\centering
	\includegraphics[scale=0.2]{pics/ex9c_1}
	
	Case 1: $z_t > 0$
\end{frame}


\begin{frame}{Exercise 9.c}
	\begin{exampleblock}{}
		(c) Represent these two cases graphically
	\end{exampleblock}
	\centering
	\includegraphics[scale=0.2]{pics/ex9c_2}
	
	Case 2: $z_t < 0$
\end{frame}




\subsection{ex 10}

\begin{frame}[label=ex10]{Exercise 10: inventory cost and order flow risk}
	\begin{itemize}
		\item Take inventory cost model from lecture;
		\item dealer's utility: $\mathbb{E}U(w_{t+1}) = \mathbb{E}_t [w_{t+1}] - \rho \sqrt{\mathbb{V}(w_{t+1})}$;
		\item dealers competitive and short-sighted (i.e., $\mathbb{E}U(w_{t+1})=0$);
		\item dealer sets pricing schedule $p_t(y_t)$ depending on order flow $y_t$;
		\item traders submit single-unit orders ($d_t = \pm 1$):
		\begin{itemize}
			\item w.p. $1-\delta$ trader unsophisticated -- places buy/sell order w.p. $1/2$ each;
			\item w.p. $\delta$ trader price-sensitive -- buys if $p_{t-1} < \mu_{t-1}$, sells if $p_{t-1} > \mu_{t-1}$
			% idea: traders do not observe p_t, base their decisions around previous period's values
			\item (no privately informed traders)
		\end{itemize}
	\end{itemize}
\end{frame}


\begin{frame}{Exercise 10.a}
	\begin{exampleblock}{}
		(a) Assume $p_t = \mu_t - \beta(z_t - y_t) = \mu_t - \beta z_{t+1}$. 
		
		Compute $\mathbb{E}_t[p_{t+1}]$ and $\mathbb{V}_t(p_{t+1})$.
	\end{exampleblock}
	\pause
	\begin{itemize}
		\item Market clearing: $d_t = y_t$;
		\begin{itemize}
			\item so if $z_{t+1} > 0$ then $p_t < \mu_t$ then $\mathbb{E}d_{t+1} = \delta$;
			\item if $z_{t+1} < 0$ then $p_t > \mu_t$ then $\mathbb{E}d_{t+1} = -\delta$.
		\end{itemize}
		\item Price at $t+1$ is $p_{t+1} = \mu_{t+1} - \beta(z_{t+1} - d_{t+1})$, so
		% Expectation computed as of the end of period t, after trades at t -- then z_{t+1} is already known
		\begin{align*}
			\mathbb{E}_t [p_{t+1}] &= \mu_t - \beta z_{t+1} + \beta \mathbb{E} [d_{t+1}] = 
			\begin{cases}
				\mu_t - \beta z_{t+1} + \beta \delta & \text{ for } z_{t+1} > 0;
				\\
				\mu_t & \text{ for } z_{t+1} = 0;
				\\
				\mu_t - \beta z_{t+1} - \beta \delta & \text{ for } z_{t+1} < 0;
			\end{cases}
			\\
			\mathbb{V}_t (p_{t+1}) &= 
			\begin{cases}
				\beta^2 (1-\delta^2) + \sigma_\epsilon^2 & \text{ for } z_{t+1} \neq 0;
				\\
				\beta^2 + \sigma_\epsilon^2 & \text{ for } z_{t+1} = 0;
			\end{cases}
		\end{align*}
	\end{itemize}
\end{frame}


\begin{frame}{Exercise 10.b}
	\begin{exampleblock}{}
		(b) Solve for the equilibrium pricing policy.
	\end{exampleblock}

	\begin{align*}
		U(y_t) = \mathbb{E}_t [p_{t+1}] (z_t - y_t) + c_t + p_t y_t - \rho \sqrt{\mathbb{V}(p_{t+1})} \cdot | z_t - y_t |
		% first three elements: Ew_{t+1}, namely current cash ct, revenue from today's sales pt*yt, and expected liquidation cost of remaining holdings
	\end{align*}
	Dealer chooses how much to sell given some fixed price $p_t$, so schedule $p_t(y_t)$ must satisfy $\frac{\partial U}{\partial y_t} = 0$. Inverse supply function is then given by:
	\pause
	\begin{align*}
		p_t &= 
		\begin{cases}
			\mathbb{E}_t [p_{t+1}] - \rho \sqrt{\mathbb{V}(p_{t+1})} & \text{ if } z_{t+1} > 0
			\\
			\mathbb{E}_t [p_{t+1}] & \text{ if } z_{t+1} = 0
			\\
			\mathbb{E}_t [p_{t+1}] + \rho \sqrt{\mathbb{V}(p_{t+1})} & \text{ if } z_{t+1} > 0
		\end{cases}
		\\
		&= 
		\begin{cases}
		\mu_t - \beta z_{t+1} + \beta \delta - \rho \sqrt{\beta^2 (1-\delta^2) + \sigma_\epsilon^2} & \text{ if } z_{t+1} > 0
		\\
		\mu_t & \text{ if } z_{t+1} = 0
		\\
		\mu_t - \beta z_{t+1} - \beta \delta + \rho \sqrt{\beta^2 (1-\delta^2) + \sigma_\epsilon^2} & \text{ if } z_{t+1} > 0
		\end{cases}
	\end{align*}
\end{frame}


\begin{frame}{Exercise 10.c}
	\begin{exampleblock}{}
		(c) Find $\beta$.
	\end{exampleblock}
	
	\pause
	\begin{itemize}
		\item We assumed that $p_t = \mu_t - \beta z_{t+1}$.
		\item For this to coincide with our answer to (b), it must be that 
		$$ \beta \delta - \rho \sqrt{\beta^2 (1-\delta^2) + \sigma_\epsilon^2} = 0 $$
		$$ \Leftrightarrow \beta = \frac{\rho \sigma_\epsilon}{\sqrt{\delta^2 - \rho^2(1-\delta^2)}} $$
		(need $\rho < \sqrt{\frac{\delta^2}{1 - \delta^2}}$)
		\item $\beta$ increasing in $\rho$ and $\sigma_\epsilon$ -- similar to lecture
		\item (NEW!) $\beta$ decreasing in $\delta$ -- high $\delta$ means order flow is very predictable, less risk for the dealer. Also $\delta$-traders help dealer rebalance his inventory.
	\end{itemize}
\end{frame}


\begin{frame}{Exercise 10.d}
	\begin{exampleblock}{}
		(d) Compute spread.
	\end{exampleblock}
	
	\pause
	\begin{itemize}
		\item We have $p_t(z_{t+1}) = \mu_t - \beta z_{t+1}$, hence
		\begin{itemize}
			\item $a_t = p_t(z_t - 1) = \mu_t - \beta z_t + \beta$
			\item $b_t = p_t(z_t + 1) = \mu_t - \beta z_t - \beta$
			\item $S_t = a_t - b_t = 2\beta = \frac{2 \rho \sigma_\epsilon}{\sqrt{\delta^2 - \rho^2(1-\delta^2)}}$
		\end{itemize}
		\item Spread compensates dealer for two types of risk here:
		\begin{itemize}
			\item fundamental risk ($\sigma_\epsilon$)
			\item order flow risk ($1-\delta$)
			\item both affect $t+1$ value of asset holdings -- through $\mu_{t+1}$ and $p_{t+1}$ resp.
		\end{itemize}
	\end{itemize}
\end{frame}


\end{document} 