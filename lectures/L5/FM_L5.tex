%%% License: Creative Commons Attribution Share Alike 4.0 (see https://creativecommons.org/licenses/by-sa/4.0/)
%%% Slides are based heavily on earlier versions of this course taught by Jesper Rudiger.

\documentclass[english,10pt
%,handout
,aspectratio=169
]{beamer}
%%% License: Creative Commons Attribution Share Alike 4.0 (see https://creativecommons.org/licenses/by-sa/4.0/)
%%% Slides are based heavily on earlier versions of this course taught by Jesper Rudiger and Peter Norman Sorensen.

\DeclareGraphicsExtensions{.eps, .pdf,.png,.jpg,.mps,}
\usetheme{reMedian}
\usepackage{parskip}
\makeatother

\renewcommand{\baselinestretch}{1.1} 

\usepackage{amsmath, amssymb, amsfonts, amsthm}
\usepackage{enumerate}
\usepackage{hyperref}
\usepackage{url}
\usepackage{bbm}
\usepackage{color}

\usepackage{tikz}
\usepackage{tikzscale}
\newcommand*\circled[1]{\tikz[baseline=(char.base)]{
		\node[shape=circle,draw, inner sep=-20pt] (char) {#1};}}
\usetikzlibrary{automata,positioning}
\usetikzlibrary{decorations.pathreplacing}
\usepackage{pgfplots}
\usepgfplotslibrary{fillbetween}
\usepackage{graphicx}

\usepackage{setspace}
%\thinmuskip=1mu
%\medmuskip=1mu 
%\thickmuskip=1mu 


\usecolortheme{default}
\usepackage{verbatim}
\usepackage[normalem]{ulem}

\usepackage{apptools}
\AtAppendix{
	\setbeamertemplate{frame numbering}[none]
}
\usepackage{natbib}




\title{Financial Markets Microstructure \\ Lecture 7}

\subtitle{Depth determinants \\ %Empirics of illiquidity\\
	Chapter 4 of FPR}

\author{Egor Starkov}

\date{K{\o}benhavns Unversitet \\
	Spring 2022}


\begin{document}
	
\AtBeginSection[]{
	\frame<beamer>{
		\frametitle{This lecture:}
		\tableofcontents[currentsection,currentsubsection]
}}

\frame[plain]{\titlepage}

%\section{Revision and problems}

\begin{frame}{Previously on...}
	\begin{enumerate}
		\item The spread is not only driven by adverse selection: order costs and inventory risk have an effect as well
		\item However, we would expect the dynamic effect of these three different mechanisms to be different
		\begin{itemize}
			\item Order costs: only short-run effect 
			\item Inventory risk: medium-run effect, but should be zero in the long run
			\item Adverse selection: long-run effect 
		\end{itemize}
	\end{enumerate}
\end{frame}


%\begin{frame}{Exercises from last time}
%	\begin{itemize}
%		\item In Absalon, I have attached an article from the Economist, December 14 2013, on the Volcker rule. Discuss the claim that \emph{``In practice banks will probably respond by making markets for a narrow range of securities that already trade frequently, and thus might reasonably be expected to do so in the future. Meanwhile, the securities that now change hands less frequently are likely to be shunned, making them even harder to trade.''}
%		\item Two other Bloomberg articles are on Einar Aas and his exclusion from Nasdaq. Use this case to discuss how risk aversion can stem from regulation.
%	\end{itemize}
%\end{frame}


\begin{frame}{Today}
	\begin{itemize}
		\item \textbf{Trade size}
		\begin{itemize}
			\item How does trade size affect prices? 
			\item I.e., what determines market \structure{depth}?
			\item (Spoiler: mostly the same factors as with liquidity)
			\item Will look at Kyle (1985) model -- an alternative to GM that allows flexible trade size
		\end{itemize}
		%\item \textbf{Empirical estimation}
		%\begin{itemize}
		%	\item Before we looked at how to estimate the spread, but without a theory for what drives it
		%	\item Thus: talk about estimating drivers of the spread
		%	\item Furthermore:
		%	\begin{itemize}
		%		\item Look at estimating price impact/depth
		%		\item Look at estimating proportion of informed trading
		%	\end{itemize}
		%\end{itemize}
	\end{itemize}
\end{frame}



%\section{Depth determinants and Kyle model}

\begin{frame}{Prices and trade size}
	\begin{itemize}
		\item How does trade size affect prices? 
		\begin{itemize}
			\item Spread larger for large trades, price moves further from efficient level
			\item I.e., market has limited \structure{depth}
		\end{itemize}
		\item Why?
		\pause
		\begin{enumerate}[<+->]
			\item \alert{Adverse selection}\visible<handout:0>{: larger trades indicate more/stronger news}
			\item \alert{Inventory risk}\visible<handout:0>{: large positions are risky and take dealers longer to unwind, hence require larger premiums}
			\item \structure{Imperfectly competitive dealers}\visible<handout:0>{: market power allows dealers to set wider spread and steeper or flatter pricing schedules}
			\item \structure{Order processing costs}\visible<handout:0>{: may increase or decrease (per stock) in total order size}
		\end{enumerate}
	\end{itemize}
\end{frame}


\begin{frame}{Kyle model}
	\begin{itemize}
		\item We will look at Kyle (1985) model which links market depth to adverse selection
		\item It can be extended to accomodate imperfect competition among dealers (see 4.2.4) and inventory risk (4.3)
		\begin{itemize}
			\item the inventory risk version is broadly similar to the Stoll model that we skipped
			\item trading costs are very difficult to incorporate in this model
		\end{itemize}
	\end{itemize}
\end{frame}


\begin{frame}{Setup: Broad strokes}
	\begin{itemize}
		\item \structure{Speculator}: has private information
		\begin{itemize}
			\item Trades using a `large' speculative market order
			\item Strategically moderates order size to reduce price impact
			\item `Hides' behind noise traders who submit a random size order
		\end{itemize}
		\item \structure{Market makers/dealers (MM)}
		\begin{itemize}
			\item Risk neutral and competitive (zero profits)
			\item Observes aggregate order flow, but cannot distinguish speculative orders from noise orders
			\item In Kyle model orders are cleared by the MM in \textit{batches}, as opposed to one-by-one in Glosten \& Milgrom
		\end{itemize}
	\end{itemize}
\end{frame}


\begin{frame}{Setup}
\begin{itemize}
	\item \structure{Asset}: Trade in one risky asset with value  $v \sim \mathcal{N}(\mu, \sigma^2_v)$
	\item \structure{Speculator}: Observes true value $v$ (perfect information)
	\begin{itemize}
		\item Places market order $x$
		\item If the order clears at price $p$: gain is $x(v-p)$
		\item Does \alert{not know} $p$ when choosing $x$: maximizes expected gain (risk neutral) given $\mathbb{E}[p|x]$
	\end{itemize}
	\item \structure{Noise trader}: Has random demand $u \sim \mathcal{N}(0, \sigma^2_u)$
	\item \structure{MM}: Submits a supply schedule of $(q,p)$ combinations:
	\begin{itemize}
		\item Observes aggregate flow $q=x+u$, but not $x$ and $u$
		\item Competitive (zero profit): $p = \mathbb{E}[v|q]$
	\end{itemize}
	\item \structure{Assumption}: $u$ and $v$ are jointly normal and independent
	%\item \structure{Method}: Postulate linear strategy-- then check that this is equilibrium
\end{itemize}
\end{frame}


\begin{frame}{Setup: Timing}
	\begin{itemize}
		\item To be explicit, the timing is as follows:
	\end{itemize}
	\begin{enumerate}
		\item at the beginning of the period:
		\begin{itemize}
			\item speculator chooses order size $x$
			\item noise traders submit their order $u$
			\item dealer submits price schedule $(q,p)$
		\end{itemize}
		\item then market price $p(q)$ is determined given total order $q=x+u$
		\item at the end of the period payoffs are realized
	\end{enumerate}
\end{frame}


\begin{frame}{Linear equilibrium: outline}
\begin{itemize}
	\item Look for equilibrium where speculator's strategy is linear: \alert{$x=\beta(v-\mu)$}
	\begin{itemize}
		\item Note: $\beta$ is endogenously determined by the equilibrium
		\item $\beta>0$ measures speculator \textit{aggression}
	\end{itemize}
	\item MM knows the speculator's strategy:
	\begin{itemize}
		\item Observes $q=x+u=\beta(v-\mu)+u$, and wants to \alert{estimate $v$}
		\item Intuitively, \alert{$\mathbb{E}[v|q] = \mu + \lambda  q$}, where $\lambda $ is the regression coefficient $\mathbb{C}(v,q)/\mathbb{V}(q)$, and recall that $\mu=\mathbb{E}[v]$. (This is derived formally in the following slides.)
	\end{itemize}
	\item Since $p=\mathbb{E}[v|q]$, we can use the conditional expectation to get a price impact equation 
	\[
	p-\mu=\lambda q.
	\]
	As before, $1/\lambda$ is a measure of market depth
\end{itemize}
\end{frame}


\begin{frame}{Aside: Deriving the distribution of $v|q$}
	\begin{block}{Claim}
		If $q=\beta(v-\mu) + u$ and $v,u$ are joint normal then $v|q$ is normal, with
		\begin{align*}
			\mathbb{E}[v|q] &= \mathbb{E}[v] + \frac{\mathbb{C}(v,q)}{\mathbb{V}(q)} (q-\mathbb{E}[q]),
			\\
			\mathbb{V}(v|q) &= 1/(1/\sigma^2_v + \beta^2 / \sigma^2_u).
		\end{align*}
	\end{block}
	%\structure{Proof}:
	\begin{enumerate}
		\item Will show for our case ($v \sim \mathcal{N}(\mu, \sigma^2_v)$, $u \sim \mathcal{N}(0, \sigma^2_u)$, and $v \perp u$). In this case:
		\begin{align*}
			q    & \sim \mathcal{N}(0, \beta^2 \sigma^2_v+\sigma^2_u),
			\\
			q|v & \sim \mathcal{N}(\beta(v-\mu), \sigma^2_u).
		\end{align*}
		Also, note that $\mathbb{C}(v,q)= \mathbb{C}(v, \beta(v-\mu)+u)=\beta\sigma^2_v$
	\end{enumerate}
\end{frame}


\begin{frame}%{Aside: Normal information model (2)}
	\begin{enumerate}
		\setcounter{enumi}{1}
		\item Use Bayes' rule to derive the conditional density: $f(v|q) = f(v) \frac{f(q|v)}{f(q)} $.
		\\
		(remember that pdf of $x \sim \mathcal{N}(\mu,\sigma^2)$ is \structure{$f(x)=\frac{1}{\sqrt{2 \pi \sigma^2}} e^{-\frac{(x-\mu)^2}{2 \sigma^2}}$})
		\begin{align*}
			f(v|q) = f(v) \frac{f(q|v)}{f(q)} 
			&= \sqrt{\frac{2\pi (\beta^2 \sigma^2_v+\sigma^2_u)}{2 \pi \sigma^2_v \cdot 2 \pi \sigma^2_u}} e^{- \frac{(v-\mu)^2}{2\sigma^2_v} - \frac{(q-\beta(v-\mu))^2}{2\sigma^2_u } + \frac{q^2}{2(\beta^2 \sigma^2_v + \sigma^2_u)}}
			\\
			&= \frac{1}{\sqrt{2 \pi \frac{\sigma^2_v \sigma^2_u}{\beta^2 \sigma^2_v+\sigma^2_u}}} e^{-\alert{\left[ \frac{(v-\mu)^2}{2\sigma^2_v} + \frac{(q-\beta(v-\mu))^2}{2\sigma^2_u } \right]} + \frac{q^2}{2(\beta^2 \sigma^2_v + \sigma^2_u)}}
		\end{align*}
		\item $f(v|q)$ looks like a normal pdf! From the leading fraction infer that  $\mathbb{V}(v|q) = \frac{\sigma^2_v \sigma^2_u}{\beta^2 \sigma^2_v+\sigma^2_u}$
		
	\end{enumerate}
\end{frame}


\begin{frame}
	\begin{enumerate}
		\setcounter{enumi}{3}
		\item Rearrange the terms in the power to have the $2\mathbb{V}(v|q)$ in the denominator:
		\begin{align*}
			&- \frac{(v-\mu)^2}{2\sigma^2_v} - \frac{(q-\beta(v-\mu))^2}{2\sigma^2_u } + \frac{q^2}{2(\beta^2 \sigma^2_v + \sigma^2_u)} =
			\\
			&= - \frac{(v-\mu)^2 \sigma^2_u}{2\sigma^2_v \sigma^2_u} - \frac{(q-\beta(v-\mu))^2 \sigma^2_v}{2 \sigma^2_v \sigma^2_u } + \frac{\frac{q^2}{\beta^2 \sigma^2_v + \sigma^2_u} \sigma^2_v  \sigma^2_u}{2 \sigma^2_v  \sigma^2_u} \
			\\
			&= \frac{1}{\frac{2 \sigma^2_v \sigma^2_u}{\beta^2 \sigma^2_v+\sigma^2_u}} \left[ -\frac{(v-\mu)^2 \sigma^2_u}{\beta^2 \sigma^2_v+\sigma^2_u} - \frac{(q-\beta(v-\mu))^2 \sigma^2_v}{\beta^2 \sigma^2_v+\sigma^2_u} + \frac{q^2 \sigma^2_v  \sigma^2_u}{ \left(\beta^2 \sigma^2_v + \sigma^2_u \right)^2} \right]
		\end{align*}
		\item And then rewrite the square bracket as a parabola of $v$:
		\begin{align*}
			&= \frac{1}{2\mathbb{V}(v|q)} \left[ \alert{-v^2} \frac{ \sigma^2_u + \beta^2 \sigma^2_v }{\beta^2 \sigma^2_v+\sigma^2_u} \alert{+ 2v} \frac{ \mu \sigma^2_u + \beta^2 \mu \sigma^2_v + \beta q \sigma^2_v }{\beta^2 \sigma^2_v+\sigma^2_u} + ... \right]
			\\
			&= \frac{-\left( v - \frac{\mu \sigma^2_u + \beta^2 \mu \sigma^2_v + \beta q \sigma^2_v}{\beta^2 \sigma^2_v + \sigma^2_u} \right)^2 }{2\mathbb{V}(v|q)}
			= \frac{-\left( v-\alert{\mu-\frac{\beta \sigma^2_v}{\beta^2 \sigma^2_v + \sigma^2_u}q } \right)^2 }{2\mathbb{V}(v|q)}
		\end{align*} 
		(I leave it to you to confirm that the "..." part works out)
	\end{enumerate}
\end{frame}


\begin{frame}
	\begin{enumerate}
		\setcounter{enumi}{5}
		\item Putting everything together:
		\[
			f(v|q) = \frac{1}{\sqrt{2 \pi \frac{\sigma^2_v \sigma^2_u}{\beta^2 \sigma^2_v+\sigma^2_u}}} e^{-\frac{\left( v - \mu - \frac{\beta \sigma^2_v}{\beta^2 \sigma^2_v + \sigma^2_u}q \right)^2 }{2 \frac{\sigma^2_v \sigma^2_u}{\beta^2 \sigma^2_v+\sigma^2_u}}}
		\]
		This is indeed a pdf of the normal distribution with 
		\begin{align*}
			\mathbb{E}[v|q] &= \alert{\mu+\frac{\beta \sigma^2_v}{\beta^2 \sigma^2_v + \sigma^2_u}q} = \mathbb{E}[v] + \frac{\mathbb{C}(v,q)}{\mathbb{V}(q)} (q-\mathbb{E}[q]),
			\\
			\mathbb{V}[v|q] &= \structure{ \left( \frac{1}{\sigma^2_v}+\frac{\beta^2}{\sigma^2_u} \right)^{-1}},
		\end{align*}
		as claimed. 
	\end{enumerate}
\end{frame}


\begin{frame}{Dealer's strategy}
	\[
		\mathbb{E}[v|q] = \mathbb{E}[v] + \frac{\mathbb{C}(v,q)}{\mathbb{V}(q)} (q-\mathbb{E}[q]) = \mu + \frac{\mathbb{C}(v,q)}{\mathbb{V}(q)} q,
	\]
	From zero profit condition: \alert{$p = \mathbb{E}[v|q]$}, hence
	\begin{align*}
		p &= \mu + \lambda q,
		\\
		\text{where } \lambda &\equiv \frac{\mathbb{C}(v,q)}{\mathbb{V}(q)} = \frac{\beta \sigma^2_v}{\beta^2 \sigma^2_v + \sigma^2_u}
	\end{align*}
\end{frame}


\begin{frame}{Speculator's strategy}
	The speculator takes for granted the pricing rule $p=\mu+\lambda q$
	\begin{itemize}
		\item The \structure{profit} is $\Pi(x) = x(v-p) = x(v-\mu-\lambda x - \lambda u)$
		\item \structure{Expected profit} is $ \mathbb{E}[\Pi(x)] = x(v-\mu-\lambda x)$, since $\mathbb{E}[u]=0$
		\item Speculator chooses $x$ to maximize $\mathbb{E}[\Pi(x)]$. Using the first-order condition:
		\begin{align*}
			v-\mu -2\lambda x &=0
			\\
			\Rightarrow \alert{x} &\alert{=\beta(v-\mu),}
			\\
			\text{where } \alert{\beta} &\alert{=1/(2\lambda)}
		\end{align*}
		\item Confirmed that it is optimal for the speculator to use a linear strategy!
		\item Note analogy to monopoly problem: \visible<handout:0|2->{\structure{trade-off between trading more and trading at better price}}
	\end{itemize}
\end{frame}


\begin{frame}{Closing the equilibrium}
	\begin{itemize}
		\item Finally, `match' the coefficients:
		\[
			\frac{1}{2\beta}=\lambda = \frac{\beta \sigma^2_v}{\beta^2 \sigma^2_v + \sigma^2_u}
		\]
		i.e. $\beta^2 \sigma^2_v + \sigma^2_u = 2 \beta^2 \sigma^2_v$ which yields
		\[
				\alert{ \beta = \frac{\sigma_u}{\sigma_v}}\text{ and } \alert{ \lambda = \frac{\sigma_v}{2\sigma_u}}.
		\]
		\item Thus: the strategies are optimal given the prices, and the prices optimal given the strategies $\rightarrow $ \textbf{equilibrium}
	\end{itemize}
\end{frame}


\begin{frame}{Equilibrium properties}
	\[
		\alert{ \beta = \frac{\sigma_u}{\sigma_v}}\text{ and } \alert{ \lambda = \frac{\sigma_v}{2\sigma_u}}.
	\]
	\begin{itemize}
		\item Speculator is more \textbf{aggressive} ($\beta$ higher) when:
		\begin{enumerate}
			\item The informational advantage $\sigma_v$ is smaller (why?)
			\item There's more noise $\sigma_u$ to hide behind (why?)
		\end{enumerate}
		\pause
		\item \textbf{Market depth}:
		\[
			\frac{1}{\lambda} = 2\beta = 2 \frac{\sigma_u}{\sigma_v}
		\]
		The market is deeper when there is less insider trading and more noise trading
	\end{itemize}
\end{frame}


\begin{frame}{Equilibrium properties}
	\begin{itemize}
		\item Insider's a priori (before observing $v$) \textbf{expected gain}:
		\begin{align*}
		\mathbb{E}[x(v-\mu-\lambda x)] 
		& =\mathbb{E} \left[ \beta(v-\mu)\left(v-\mu-\frac{v-\mu}{2}\right) \right] \\
		&=\beta\frac{ \sigma^2_v}{2}=\frac{\sigma_v \sigma_u}{2}
		\end{align*}
		Comment: speculator expects a positive profit (could abstain). Competitive risk-neutral MM earns zero profits. Noise traders lose. Same as in GM.
		\pause
		\item Market maker's perceived posterior (after observing $q$) \textbf{variance} of $v$ is
		\[
		\mathbb{V}(v|q) = \frac{1}{1/\sigma^2_v + \beta^2/\sigma^2_u} = \frac{\sigma^2_v}{2}
		\]
		Exactly half the prior variance: \structure{Insider reveals half his information}
	\end{itemize}
\end{frame}


\begin{frame}{Kyle's model: summary}
	\begin{itemize}
		\item \textbf{Dealer/market maker model}: Competitive, risk-neutral (zero profit) dealer chooses a supply schedule
		\item \textbf{Informed trader}: Observes signal about asset value and places market order
		\item \textbf{Market clearing}: \structure{Auction}, dealer observes only total demand (informed + noise), total demand clears
		\item \textbf{Insights}: informed trading is a factor generating limited market depth, insider always reveals half his information
		\item \textbf{Advantage}: Richer trading opportunities, trader not price-taker
		\item \textbf{Shortcomings}: Still no resale
	\end{itemize}
\end{frame}


\begin{frame}{Kyle with inventory risk I}
\begin{itemize}
	\item Now let's look at how market maker's \alert{inventory risk} can lead to limited depth.
	\item Assume \structure{no informed trading}: $x = 0$.
	\item Asset value $v \sim \mathcal{N}(\mu, \sigma^2_v)$
	\item Market maker has mean-variance preferences over their next-period wealth:
	\[
	U(w_{t+1})=\mathbb{E}[w_{t+1}] - \frac{\rho}{2} \mathbb{V}(w_{t+1}),
	\]
	where $w$ is composed of cash and asset holdings: $w_{t+1} = (z_t - q_t) v + q_t p_t$
	\item MM's initial asset position is $z_t$ (initial cash is irrelevant, ignore it).
\end{itemize}
\end{frame}


\begin{frame}{Kyle with inventory risk II}
\begin{itemize}
	\item To get the pricing schedule, follow the competitive logic:
	\begin{itemize}
		\item The market-maker takes some market price $p$ as given, chooses how much $q(p)$ to sell at this price:
		\begin{align*}
			\max_q \left\{ \underbrace{(z_t-q)\mathbb{E}[v] + qp}_{\mathbb{E}[w_{t+1}]} - \underbrace{\frac{\rho}{2} (z_t-q)^2 \mathbb{V}(v) }_{ \mathbb{V}(w_{t+1})} \right\}
		\end{align*}
		\item FOC: $p - \mu + \rho (z_t-q) \sigma_v^2 = 0 \iff \structure{q(p) =} z_t + \frac{p - \mu}{\rho \sigma_v^2}$
	\end{itemize}
	\item For market to clear, need $q(p) = u = q$ (dealer's supply = total traders' market order), so inverting the pricing schedule we get: 
	$$p(q) = \mu + \rho \sigma_v^2 (q - z_t).$$
\end{itemize}
\end{frame}


\begin{frame}{Kyle with inventory risk III}
$$p(q) = \mu + \rho \sigma_v^2 (q - z_t)$$
Takeaways:
\begin{enumerate}
	\item Depth (dictated by the dealer's willingness to trade at a given price) is limited 
	\item This is despite traders still being completely price-insensitive in this model!
	\item Price impact depends on asset riskiness $\sigma_v^2$ and MM's risk aversion $\rho$.
	\item Midquote depends on $z_t$
\end{enumerate}
So really, all the same stuff as in GM with inventory risk.
\\ The book also looks at versions with many MMs with heterogeneous $\rho$s, and many imperfectly competitive MMs.
\end{frame}


\begin{frame}{Extensions}
	Other extensions are possible:
	\begin{enumerate}
		\item \textbf{Dynamics}
		\begin{itemize}
			\item In a fully dynamic model, the insider reveals less than half of the information in each period. Why?
			\pause \visible<handout:0>{\structure{In order to better benefit from informational advantage}}
			\pause
			\item In the limit where trade is continuous over $[0,T]$, then $\mathbb{V}(v|q_0, ..., q_t) \simeq (T-t)\frac{\sigma^2_v}{T}$: variance decreases linearly in time. 
			\pause \structure{Model of how to split a large trade over time}
		\end{itemize}
		\pause
		\item \textbf{More insiders}
		\begin{itemize}
			\item More insiders are more competitive; more aggressive
			\item The market is more liquid and more information revealed
			\item In dynamic model with several insiders: rush to trade on common information from the beginning
		\end{itemize}
	\end{enumerate}
\end{frame}


\begin{frame}{}
	\begin{enumerate}
		\setcounter{enumi}{2}
		\item \textbf{Imperfect market maker competition} (Cournot style)
		\begin{itemize}
			\item Finite number of market makers, $k=1,..., K$
			\item Market maker $k$ supplies $y^k=\phi(p-\mu)$
			\item Market clears at price $p$ with $\sum y^k = q$
			\item Strategic market maker takes into account effect of orders on profits
			\item Now: $p=\mu+\lambda q$ where $\lambda = \alpha (K-1)/(K-2) > \alpha$. Why? 
			\pause \visible<handout:0>{\structure{Increasing price sensitivity reduces $k$'s amount traded, but raises market price, the former outweighs.}}
		\end{itemize}
		\pause
		%\item \textbf{Risk averse market makers}
		%\begin{itemize}
		%	\item Similar to Stoll (1978) from textbook and GM with risk-aversion that we covered; same takeaways
		%	\item See the following slides
		%	%\item See textbook for ver with no asymmetric information. Can add that and imperfect competition.
		%	%\item Allows for imperfect competition: depth is a factor of $(K-1)/(K-2)$ of what we found for Stoll
		%\end{itemize}
		%\pause
		\item \textbf{Trading costs}
		\begin{itemize}
			\item Trivial in GM. Very difficult here, both technically and conceptually.
			\item Don't know how many trades there are, don't know the total volume (not $q$ -- some noise traders' orders could've cancelled each other out)
			\item Taking costs as a function of order imbalance $|q|$ (even constant) makes the normal model intractable (no linear trading strategy, no linear pricing)
		\end{itemize}
	\end{enumerate}
\end{frame}


\begin{frame}{Homework}
	\begin{enumerate}
		\item We will talk about empirical estimation of factors of illiquidity next time (ch.5) and begin talking about LOB markets (without dealers; ch.6)
		\item Solve ex 3 in ch.4 (p.159): Kyle's model with competition among speculators.
	\end{enumerate}
\end{frame}

\end{document} 