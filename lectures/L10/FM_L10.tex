%%% License: Creative Commons Attribution Share Alike 4.0 (see https://creativecommons.org/licenses/by-sa/4.0/)
%%% Slides are based heavily on earlier versions of this course taught by Jesper Rudiger.

\documentclass[english,10pt
%,handout
,aspectratio=169
]{beamer}
%%% License: Creative Commons Attribution Share Alike 4.0 (see https://creativecommons.org/licenses/by-sa/4.0/)
%%% Slides are based heavily on earlier versions of this course taught by Jesper Rudiger and Peter Norman Sorensen.

\DeclareGraphicsExtensions{.eps, .pdf,.png,.jpg,.mps,}
\usetheme{reMedian}
\usepackage{parskip}
\makeatother

\renewcommand{\baselinestretch}{1.1} 

\usepackage{amsmath, amssymb, amsfonts, amsthm}
\usepackage{enumerate}
\usepackage{hyperref}
\usepackage{url}
\usepackage{bbm}
\usepackage{color}

\usepackage{tikz}
\usepackage{tikzscale}
\newcommand*\circled[1]{\tikz[baseline=(char.base)]{
		\node[shape=circle,draw, inner sep=-20pt] (char) {#1};}}
\usetikzlibrary{automata,positioning}
\usetikzlibrary{decorations.pathreplacing}
\usepackage{pgfplots}
\usepgfplotslibrary{fillbetween}
\usepackage{graphicx}

\usepackage{setspace}
%\thinmuskip=1mu
%\medmuskip=1mu 
%\thickmuskip=1mu 


\usecolortheme{default}
\usepackage{verbatim}
\usepackage[normalem]{ulem}

\usepackage{apptools}
\AtAppendix{
	\setbeamertemplate{frame numbering}[none]
}
\usepackage{natbib}



\title{Financial Markets Microstructure \\ Lecture 15}

\subtitle{Value of Liquidity\\
	Chapter 9 of FPR}

\author{Egor Starkov}

\date{K{\o}benhavns Unversitet \\
	Spring 2021}


\begin{document}
\AtBeginSection[]{
\frame<beamer>{
\frametitle{This lecture:}
\tableofcontents[currentsection,currentsubsection]
}}
\frame[plain]{\titlepage}


\begin{frame}{Previously on FMM}
	\begin{itemize}
		\item \structure{Transparency} mostly reallocates welfare across market participants 
		\begin{itemize}
			\item Uninformed traders benefit, so T {helps liquidity}
			\item Insiders may lose, so T {worsens price discovery}
			\item Dealers/exchanges may win or lose
		\end{itemize}
		\item But transparency may also impede risk sharing, foster collusion, and have adverse effects when it is asymmetrically distributed
		\item Opaqueness can be good in limit books
		\begin{itemize}
			\item Hidden limit orders help uninformed traders hedge their positions 
			\\where making these orders visible would by itself create adverse price movements
		\end{itemize}
	\end{itemize}
\end{frame}


\begin{frame}{Today: value of liquidity}
	\begin{itemize}
		\item So far we looked at how illiquidity makes asset's trade price deviate from its fundamental value
		\item But illiquidity may itself affect the asset \emph{value}
		\item Case study -- \structure{U.S. Treasury Notes and Bills}: (\cite{amihud_liquidity_1991})
		\begin{itemize}
			\item notes are long-term (2-10y), bills are short-term ($<12$m) US govt loans
			\item differ only in terms -- so soon-to-mature notes are equivalent to bills
			\only<2-3|handout:0>{
				\item \alert{Bliz Quiz}: which is cheaper? (given same face values)
				\begin{enumerate}
					\item \structure<3>{note with 6m to maturity}
					\item bill with 6m to maturity
					\item neither
				\end{enumerate}
			}
			\pause[4]
			\item but notes trade at a discount relative to bills (i.e., offer higher returns) (as of 1991)
			\item why? Notes are less liquid (larger spread and brokerage fees). Why less liquid though?
		\end{itemize}
	\end{itemize}
\end{frame}


\begin{frame}{Value of liquidity}
	\begin{itemize}
		\item Why does liquidity affect asset value?
		\item Intuitively, an illiquid asset is costlier to transact
		\begin{itemize}
			\item Traders take into account transactions costs
			\item Require a return that compensates for the cost
			\item Liquidity premium: less liquid assets trade at lower prices
		\end{itemize}
		%\item Over-the-counter market model
		%\begin{itemize}
		%	\item The asset is worth less because it is inefficiently allocated in an illiquid market
		%\end{itemize}
		\item Liquidity need not be constant over time
		\begin{itemize}
			\item If illiquidity rises, asset price falls
			\item If future liquidity is random, this is a risk factor
			\item Liquidity risk may be priced
		\end{itemize}
	\end{itemize}
\end{frame}



\section{Toy model}

\begin{frame}{Liquidity premium (\cite{amihud_asset_1986})}
	\begin{itemize}
		\item Before, traders cared only about fundamental value. In this model they care about \structure{resale value} (speculation).
		\item Consider an asset with \emph{constant} relative spread, $s=(a_t-b_t)/m_t$, where $m_t=(a_t+b_t)/2$
		\begin{itemize}
			\item Note that
			\[
			a_t=m_t \left(1+\frac{s}{2}\right) \text{ and } b_t=m_t\left(1-\frac{s}{2}\right)
			\]
		\end{itemize}
		\item Buy at today's ask price $a_t$
		\begin{itemize}
			\item Hold the asset for $h$ periods, sell at $b_{t+h}$
			\item To simplify, suppose zero dividend over this period
		\end{itemize}
	\end{itemize}
\end{frame}


\begin{frame}{Deriving the premium: risk-adjusted return} \label{slide:Rr}
	\begin{itemize}
		%\item Denote the \alert{market's required} risk-adjusted \alert{return} per period as \alert{$r$}, then
		%\begin{equation}\tag{9.3}
		%a_t = \frac{\mathbb{E}(b_{t+h})}{(1+r)^h},
		%\end{equation}
		%i.e.,
		%\begin{equation}\tag{9.4}
		%m_t=\frac{\mathbb{E}(m_{t+h})}{(1+r)^h} \times \frac{1-\frac{s}{2}}{1+\frac{s}{2}}
		%\end{equation}
		\item Let \alert{$r$} denote the risk-adjusted \alert{real return} per period \alert{required by the market}. \hyperlink{slide:reqret}{\beamerbutton{What?}} Then
		\begin{align*}
			a_t &= \frac{\mathbb{E}(b_{t+h})}{(1+r)^h},
			&
			\Rightarrow m_t &= \frac{\mathbb{E}(m_{t+h})}{(1+r)^h} \times \frac{1-\frac{s}{2}}{1+\frac{s}{2}}
		\end{align*}
		\pause
		\item If we estimate the required return $r$ using mid-quotes, there is a bias due to illiquidity. 
		\item Let \structure{$R$} be the \structure{nominal return} rate, estimated from the midquotes:
		\[
			m_t = \frac{\mathbb{E}(m_{t+h})}{(1+R)^h}
		\]
		The observed $R$ is different from $r$!
	\end{itemize}
\end{frame}


\begin{frame}{Deriving the premium: approximation}
	\begin{itemize}
		\item Thus, we have
		\begin{equation}\tag{9.5}
		(1+R)^h = (1+r)^h \times \frac{1+\frac{s}{2}}{1-\frac{s}{2}}
		\end{equation}
		\begin{block}{}
			Thus: (see next slide for derivation)
			\[
			R \simeq r+\frac{s}{h}
			\]
		\end{block}
		\item Essentially, the asset's return needs to be higher by $s/h$ in order to compensate for the liquidity cost
		\begin{itemize}
			\item The difference $R-r$ is a \structure{liquidity premium}
			\item Take $h$ as representative trader's holding period for asset
			%\item Trade in high-spread assets is less frequent, so $R-r$ may be concave in measured relative spread $s$
		\end{itemize}
	\end{itemize}
\end{frame}


\begin{frame}{Appendix on the approximation}
	\begin{itemize}
		\item To get the approximation of the previous slide, we must use the approximation $\ln (1+x) \simeq x$ for small $x$
		\item Recall that $\ln x^h = h \ln x$
		\item So taking logs of (9.5) we get
		\[
		h \ln (1+R) = h\ln (1+r) + \ln(1+\frac{s}{2}) - \ln(1-\frac{s}{2}) 
		\]
		and assuming $r, R,$ and $s$ are small we apply the approximation
		\[
		hR \simeq hr+\frac{s}{2}- \left(-\frac{s}{2}\right)
		\]
		\item Rearranging we get the result
	\end{itemize}
\end{frame}


\begin{frame}{Discussion}
	\begin{itemize}
		\item Note: what about a round-trip starting from a sale?
		\begin{itemize}
			\item From $b_t=\mathbb{E}(a_{t+h})/(1+r)^h$ we get $R \simeq r-s/h$
		\end{itemize}
		\pause
		\only<2-3|handout:0>{
			\item \alert{Bliz quiz}: say we know $R$ from market data. What is the right way to find $r$?
			\begin{enumerate}
				\item $R = r + s/h$ \only<3>{\structure{$\leftarrow$ OK answer for vast majority of assets}}
				\item $R = r - s/h$
				\item \structure<3>{$\#1$ when asset is in positive aggregate supply, $\#2$ when in negative}
				\item $\#2$ when asset is in positive aggregate supply, $\#1$ when in negative
				\item $R = r$
			\end{enumerate}
		}
		\pause[4]
		\item Can expect positive liquidity premium from assets in positive aggregate supply
		\item Refer to more general model and empirics in \citet*{bongaerts_derivative_2011}
		\item Empirical evidence confirms positive liquidity premium for stocks, bonds
	\end{itemize}
\end{frame}



\section{Clientelle effects}

\begin{frame}{Clientelle effects}
	\begin{columns}
		\begin{column}{0.7\linewidth}
			{\setstretch{1.3}
				\begin{itemize}
					\item We obtained $R = r + s/h$ in our toy model
					\item In reality investors differ in $h$, expected holding period
					\item Toy extension of toy model:
					\begin{itemize}
						\item Two types of investors with $h_1 < h_2$
						\item Two assets with $s_1 < s_2$
					\end{itemize}
					\only<2-3|handout:0>{
						\item \alert{Bliz quiz:} what do you think will happen in equilibrium?
						\begin{enumerate}
							\item \structure<3>{$h_1$-investors will only trade in $s_1$-asset and vice versa}
							\item $h_1$-investors will only trade in $s_2$-asset and vice versa
							\item returns will be such as to equalize assets' worth for all investors
						\end{enumerate}
					}
				\end{itemize}
			}
		\end{column}
		\begin{column}{0.3\linewidth}
			\pause[1]
			\includegraphics<handout:0>[width=0.5\linewidth]{pics/patience}
		\end{column}
	\end{columns}
	
\end{frame}


\begin{frame}{Clientelle effects}
	\begin{itemize}
		\item Suppose in eqm $h_1$-investors trade in $s_1$-asset and $h_2$-investors trade in $s_2$-asset
		%\item then $R_1 = r + s_1/h_1$; $R_2 = r + s_2/h_2$.
		\item For this to be an eqm, need $R_1 - s_1/h_1 \geq R_2 - s_2/h_1$ and $R_2 - s_2/h_2 \geq R_1 - s_1/h_2$
		\item The two conditions are equivalent to
		\begin{equation}\tag{9.10}
			\frac{1}{h_2} \leq \frac{R_2 - R_1}{s_2 - s_1} \leq \frac{1}{h_1}
		\end{equation}
		\pause 
		\item There exist $R_1,R_2$ (and resp. $r_1,r_2$) which solve this so all ok
		\begin{itemize}
			\item There would not be a solution if we assumed the opposite kind of segregation
			\item We also cannot have both groups indifferent between both assets (would need two equalities in 9.10)
		\end{itemize}
	\end{itemize}
\end{frame}


\begin{frame}{Clientelle effects: discussion}
	\begin{itemize}
		\item Some investors specialize in illiquid assets / hope to earn the liquidity premium
		\begin{itemize}
			\item Should in equilibrium be those who trade less frequently
		\end{itemize}
		\item (would this explain the case of Treasury Bills vs Bonds?) 
		\item Note: more adverse selection implies larger spread, hence attracts traders with large $h$
		%NOTE: young are encourage to invest in stocks, old in bonds -- this is due to risk-preferences, but also possibly due to this
	\end{itemize}
\end{frame}


\section{Liquidity risk}

\begin{frame}{Liquidity risk}
	\begin{itemize}
		\item Spread $s$ randomly fluctuates over time
		\item Further, liquidity of any given asset may be arbitrarily correlated with that of other assets or the whole market
		\item These are risk factors which can also be priced
		\item Use the Liquidity CAPM model of \cite{acharya_asset_2005}
	\end{itemize}
\end{frame}


\begin{frame}{reminder: regular CAPM}
	\begin{itemize}
		\item The standard CAPM postulates that return on asset $j$ is governed by the correlation of its returns with market returns:
		\begin{equation*}
			\mathbb{E}[r_j] = r_f + \beta_j \left[ \mathbb{E}[r_M] - r_f \right]
		\end{equation*}
		\begin{equation*}
			\text{with } \beta_j = \frac{\mathbb{C}(r_j, r_M)}{\mathbb{V}(r_M)}
		\end{equation*}
		\item In particular, only \emph{systematic} risk enters asset price
		\item \emph{Idiosyncratic} risk of the asset can be diversified away
	\end{itemize}
\end{frame}


\begin{frame}{Liquidity CAPM}
	\begin{itemize}
		\item Investors care for net return $r=R-s$ where $s$ now denotes the liquidity premium
		\begin{itemize}
			\item Let $f$ denote risk-free, $M$ the market
		\end{itemize}
		\item Plugging these into the CAPM equation, we get
		\[
		\mathbb{E}[R_j-s_j]=r_f + \lambda_M \beta_j
		\]
		where $\lambda_M = \mathbb{E}[R_M-s_M]-r_f$ is the risk premium and
		\[
		\beta_j=\frac{\mathbb{C}(R_j-s_j, R_M-s_M)}{\mathbb{V}(R_M-s_M)}
		\]
	\end{itemize}
\end{frame}


\begin{frame}{}
	\begin{itemize}
		\item Expand $\mathbb{C}(R_j-s_j, R_M-s_M)$:
		\[
		=\mathbb{C}(R_j, R_M)+\mathbb{C}(s_j, s_M)-\mathbb{C}(R_j, s_M)-\mathbb{C}(s_j, R_M)
		\]
		to get $\beta_j=\beta_{1j}+\beta^{liq}_j=\beta_{1j}+\beta_{2j}-\beta_{3j}-\beta_{4j}$ with
		\begin{align*}
		\beta_{1j} 	& = \frac{\mathbb{C}(R_j, R_M)}{\mathbb{V}(R_M-s_M)} 	&& \text{: ordinary $\beta$} \\
		\beta_{2j}	& = \frac{\mathbb{C}(s_j, s_M)}{\mathbb{V}(R_M-s_M)}	&& \text{: hedge liquidity with liquidity}\\
		\beta_{3j}	& = \frac{\mathbb{C}(R_j, s_M)}{\mathbb{V}(R_M-s_M)} 	&& \text{: hedge liquidity with returns} \\
		\beta_{4j}	& = \frac{\mathbb{C}(s_j, R_M)}{\mathbb{V}(R_M-s_M)}	&& \text{: hedge returns with liquidity}
		\end{align*}
	\end{itemize}
\end{frame}


%\begin{frame}{Interpretations}
%	\begin{itemize}
%		\item [for high respective betas:]
%		\item $\beta_{2j} \propto \mathbb{C}(s_j, s_M)$: stock is valuable because it stays liquid when 
%	\end{itemize}
%\end{frame}



\section{Arbitrage}

\begin{frame}{Arbitrage}
	\begin{columns}
		\begin{column}{0.7\linewidth}
			{\setstretch{1.3}
				\begin{itemize}
					\item Main tenet of economics and finance: \structure{no arbitrage}
					\begin{itemize}
						\item Assets that generate same cash flows must cost the same
						\item If arbitrage is possible, it is immediately exploited and then there is no more arbitrage.
					\end{itemize}
					\only<2-3|handout:0>{
						\item \alert{Bliz quiz:} how does this relate to our situation? (think bills-vs-bonds)
						\begin{enumerate}
							\item Arbitrage opportunities exist because arbitrage is costly to exercise \only<3>{\alert{$\uparrow$ textbook answer}}
							%NOTE: meant costs of short-selling and leverage (and the associated risks)
							\item Arbitrage opportunities exist because all of economics (except our course) is wrong
							\item No arbitrage opportunities exist \only<3>{\structure{$\leftarrow$ my answer}}
						\end{enumerate}
					}
					\pause[4]
					\item Textbook spends an obscene amount of paper arguing \alert{why arbitrage cannot be realized} in our case (assets with same cash flows but different liquidity)
					\begin{itemize}
						\item Gist: arbitraging is itself a costly activity (due to leverage and short-selling constraints)
					\end{itemize}
					\item But in what we saw, \alert{there are no arbitrage opportunities}
					\begin{itemize}
						\item Arbitrageurs are subject to all the same trading costs
						\item So in the toy model we saw, it looks as if there is arbitrage, but there are no actual opportunities
					\end{itemize}
				\end{itemize}
			}
		\end{column}
		\begin{column}{0.3\linewidth}
			\pause[1]
			\includegraphics<handout:0>[width=\linewidth]{pics/arbitrage}
			\vspace{3em}
		\end{column}
	\end{columns}
\end{frame}



\section{More general model (OTC markets)}

\begin{frame}{Liquidity premium and OTC markets}
	\begin{itemize}
		\item Earlier we derived the midprice keeping the spread fixed
		\item Now let's do both simultaneously.
		\item Use model of \citet*{duffie_over--counter_2005,duffie_valuation_2007}:
		\begin{itemize}
			\item Consider `over-the-counter' (OTC) market, where trade is \textit{decentralized}
			\item Instead of a centralized exchange, traders must search for dealers, and search is costly.
			\item Search costs give market power to dealers and thus produce spread
		\end{itemize}
	\end{itemize}
\end{frame}


\begin{frame}{Model}
	\begin{itemize}
		\item One asset: pays dividends to its holders each period
		\begin{itemize}
			\item Some traders value the dividend at 1 (high value traders), some at $1-c$ (low value traders), where $c \in (0,1)$
			\item Traders can hold \alert{one} or \alert{zero units} of the asset
			\item The alternative is a bank which pays interest $r$
			\item Assume the asset is supplied to fraction $q<1/2$ of the population
		\end{itemize}
		\item Continuum of traders randomly switch between high and low value: $j=h,l$
		\begin{itemize}
			\item This happens each period to a fraction $\psi$ of traders
			\item (In the long run, half the traders have high value)
			\item Switchers would like to trade the asset
			\item Search for a dealer, and find one with chance $\phi$
		\end{itemize}
	\end{itemize}
\end{frame}


\begin{frame}{Model (2)}
	\begin{itemize}
		\item Exact \structure{timing} within a period:
		\begin{enumerate}
			\item investor receives dividend payoff ($\$1$ or $\$(1-c)$)
			\item valuation changes w.p. $\psi$
			\item search ensues; match with a dealer happens w.p. $\phi$
		\end{enumerate}
		\item Dealers have some \structure{bargaining power} because they are hard to find
		\begin{itemize}
			\item Dealers quote ask $a$ and bid $b$; spread $S=a-b$
			\item Bargaining power is measured by a parameter $z \in (0,1)$
			\item Suppose $\bar{a}$ is maximum that buyer will pay,  and $\bar{b}$ is minimum seller will accept
			\item Define midprice $\mu=(\bar{a}+\bar{b})/2$ to be the asset value for the dealers
		\end{itemize}
	\end{itemize}
\end{frame}


\begin{frame}[label=solve]{Solving the OTC search model: setup}
	\begin{itemize}
		\item Recall
		\begin{enumerate}
			\item Since $q<1/2$, buy side of market is restrained (dealers have all market power)
			\item $\bar{a}$ is maximum that buyer will pay,  and $\bar{b}$ is minimum seller will accept
		\end{enumerate}
		\item \alert{Buy side:} restrained $\rightarrow$ mixed equilibrium. Set \structure{$a=\bar{a}$} and set buy probability conditional on finding dealer (can choose this since buyers are indifferent) to
		\[
		p^B = \phi \cdot \frac{\pi_S}{\pi_B},
		\]
		\alert{$\pi_S$}: fraction willing to sell; \alert{$\pi_B$}: fraction willing to buy
		\item \alert{Sell side:} no restraint, sell with prob. 1, bargain over surplus: 
		\[
		\structure{b=z\bar{b}+(1-z)\mu}.
		\]
	\end{itemize}
\end{frame}


\begin{frame}{Solving the OTC search model: setup (2)}
	\begin{itemize}
		\item Method: identify $\bar{a}$ and $\bar{b}$, and then solve for $a$ and $b$
		\item Denote the present discounted cash-flow value (as of beginning of the period) of an asset owner with private valuation $j$ by $V^0_j$ and that of a non-owner by $V^{no}_j$
		\item Since non-owner buys if $V^o_h-a \geq V^{no}_h$ and owner sells if $V^{no}_l+b \geq V^o_l$, we get
		\begin{align}
		\bar{a} 	&= V^o_h-V^{no}_h, \\
		\bar{b}	&= V^o_l-V^{no}_l 
		\end{align}
		\item So $a=V^o_h-V^{no}_h$ and $b=z(V^o_l-V^{no}_l ) + (1-z)\mu$.
		\item Finally, we must calculate the value functions
	\end{itemize}
\end{frame}


\begin{frame}{Solving the OTC search model: setup (3)}
	\begin{itemize}
		\item We will just look at two functions, the rest are similar
		\begin{align*}
		V^o_h 	& = \frac{1}{1+r} + \frac{(1-\psi)V^o_h}{1+r} + \frac{\psi(1-\phi)V^o_l}{1+r}+\frac{\psi\phi(V^{no}_l+b)}{1+r} \\
		V^{no}_h 	& = \frac{\psi V^{no}_l}{1+r} + \frac{(1-\psi)(1-p^B)V^{no}_h}{1+r}+\frac{(1-\psi)p^B (V^o_h-\bar{a})}{1+r}
		\end{align*}
		\item Calculate $V^o_l$ and $V^{no}_l$. Plug back into (1)-(2) to solve for $\bar{a}$ and $\bar{b}$
	\end{itemize}
\end{frame}


\begin{frame}{Conclusion (DGP)}
	\begin{itemize}
		\item Ask price is then
		\[
		a=\frac{1}{r}-\frac{2\psi}{r(1+z)}\left(1-\phi\frac{1-z}{2}\right)S,
		\]
		where $S$ is the spread
		\[
		S=a-b=\frac{(1+z)c}{2(r+2\psi)+(1-2\psi)\phi(1-z)}
		\]
		\item If $\psi>0$ the ask price is less than the $1/r$ which would arise if it were always (efficiently) held by high-value traders
		\item This is due to liquidity costs: buyers anticipate that they will sell in the future and incur the spread cost
	\end{itemize}
\end{frame}





\begin{frame}{Conclusion}
	\begin{itemize}
		\item Empirically the authors find evidence on both a \structure{liquidity premium} and a \structure{liquidity risk premium} on stocks
		\item Further, overall market liquidity may vanish at crisis times when asset prices drop rapidly
		\begin{itemize}
			\item Important risk for investors, especially speculators
			\item Financial institutions are required to hold robustly liquid assets
			\item In general, risky positions require costly collateral, margins
			%\item Section 9.4 considers the optimal allocation over time of capital devoted to costly arbitrage; one prediction is that mis-pricing can persist over time, and that unexpectedly large shocks may trigger `fire sales'
		\end{itemize}
	\end{itemize}
\end{frame}


\begin{frame}{Exercise}
	Ex.1 from ch.9 (p.347)
	%TODO2020: Economist article
\end{frame}


\appendix
\begin{frame}[allowframebreaks]{References}
	\bibliography{../teaching}
	\bibliographystyle{abbrvnat}
\end{frame}


\begin{frame}<handout:0>{Required return} \label{slide:reqret}
	\begin{columns}
		\begin{column}{0.6\linewidth}
			{\setstretch{1.3}
				\begin{itemize}
					\item We have been living in a world with only one asset. In reality, assets ``compete'' for investors' attention.
					\item In market equilibrium, risk-adjusted returns are equalized across assets.
					\item The resulting \structure{``market'' return $r$} is what investors can get by investing in any asset, and any new asset must generate return (at least) $r$ to attract funds.
				\end{itemize}
				\hyperlink{slide:Rr}{\beamerbutton{Back}}
			}
		\end{column}
		\begin{column}{0.4\linewidth}
			\pause[1]
			\includegraphics<handout:0>[width=\linewidth]{pics/portfolio}
		\end{column}
	\end{columns}
\end{frame}

\end{document}