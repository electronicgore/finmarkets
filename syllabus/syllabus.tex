%%% License: Creative Commons Attribution Share Alike 4.0 (see https://creativecommons.org/licenses/by-sa/4.0/)


%%%%%%%%%%%%%%%%%%%%%%%%%%%%%%%%%%%%%%%%%

%----------------------------------------------------------------------------------------
%	PACKAGES AND OTHER DOCUMENT CONFIGURATIONS
%----------------------------------------------------------------------------------------

\documentclass{article}

\usepackage{amssymb}
\usepackage{enumerate}
\usepackage[usenames,dvipsnames]{color}
\usepackage{fancyhdr} % Required for custom headers
\usepackage{lastpage} % Required to determine the last page for the footer
\usepackage{extramarks} % Required for headers and footers
\usepackage[usenames,dvipsnames]{color} % Required for custom colors
\usepackage{graphicx} % Required to insert images
\usepackage{listings} % Required for insertion of code
\usepackage{courier} % Required for the courier font
\usepackage[table]{xcolor}
\usepackage{amsfonts,amsmath,amsthm,parskip,setspace,url}
\usepackage[section]{placeins}
\usepackage[a4paper]{geometry}
\usepackage[USenglish]{babel}
\usepackage[utf8]{inputenc}
\usepackage{hyperref}


% Margins
\topmargin=-0.45in
\evensidemargin=0in
\oddsidemargin=0in
\textwidth=6.5in
\textheight=9.0in
\headsep=0.6in

\linespread{1.1} % Line spacing

%----------------------------------------------------------------------------------------
%	DOCUMENT STRUCTURE COMMANDS
%	Skip this unless you know what you're doing
%----------------------------------------------------------------------------------------

% Header and footer for when a page split occurs within a problem environment
\newcommand{\enterProblemHeader}[1]{
\nobreak\extramarks{#1}{#1 continued on next page\ldots}\nobreak
\nobreak\extramarks{#1 (continued)}{#1 continued on next page\ldots}\nobreak
}

% Header and footer for when a page split occurs between problem environments
\newcommand{\exitProblemHeader}[1]{
\nobreak\extramarks{#1 (continued)}{#1 continued on next page\ldots}\nobreak
\nobreak\extramarks{#1}{}\nobreak
}

\setcounter{secnumdepth}{0} % Removes default section numbers
\newcounter{homeworkProblemCounter} % Creates a counter to keep track of the number of problems

\newcommand{\homeworkProblemName}{}
\newenvironment{ex}[1][Problem \arabic{homeworkProblemCounter}]{ % Makes a new environment called homeworkProblem which takes 1 argument (custom name) but the default is "Problem #"
\stepcounter{homeworkProblemCounter} % Increase counter for number of problems
\renewcommand{\homeworkProblemName}{#1} % Assign \homeworkProblemName the name of the problem
\section{\homeworkProblemName} % Make a section in the document with the custom problem count
\enterProblemHeader{\homeworkProblemName} % Header and footer within the environment
}{
\exitProblemHeader{\homeworkProblemName} % Header and footer after the environment
}

\newcommand{\problemAnswer}[1]{ % Defines the problem answer command with the content as the only argument
\noindent\framebox[\columnwidth][c]{\begin{minipage}{0.98\columnwidth}#1\end{minipage}} % Makes the box around the problem answer and puts the content inside
}

\newcommand{\homeworkSectionName}{}
\newenvironment{homeworkSection}[1]{ % New environment for sections within homework problems, takes 1 argument - the name of the section
\renewcommand{\homeworkSectionName}{#1} % Assign \homeworkSectionName to the name of the section from the environment argument
\subsection{\homeworkSectionName} % Make a subsection with the custom name of the subsection
\enterProblemHeader{\homeworkProblemName\ [\homeworkSectionName]} % Header and footer within the environment
}{
\enterProblemHeader{\homeworkProblemName} % Header and footer after the environment
}


%----------------------------------------------------------------------------------------
%----------------------------------------------------------------------------------------
%----------------------------------------------------------------------------------------
% Set up the header and footer
\pagestyle{fancy}
\lhead[c]{\textbf{{\color[rgb]{.5,0,0} K{\o}benhavns\\Universitet }} \firstxmark} % Top left header
\chead{\textbf{{\color[rgb]{.5,0,0} \Class }}\\ \hmwkTitle  } % Top center head
\rhead{\instructor \\ \theprofessor} % Top right header
\lfoot{\lastxmark} % Bottom left footer
\cfoot{} % Bottom center footer
\rfoot{Page\ \thepage\ of\ \protect\pageref{LastPage}} % Bottom right footer
\renewcommand\headrulewidth{0.4pt} % Size of the header rule
\renewcommand\footrulewidth{0.4pt} % Size of the footer rule

\setlength\parindent{0pt} % Removes all indentation from paragraphs







%----------------------------------------------------------------------------------------
%	NAME AND CLASS SECTION
%----------------------------------------------------------------------------------------

\newcommand{\hmwkTitle}{Course Plan} % Assignment title
\newcommand{\Class}{Financial Markets Microstructure} % Course/class
\newcommand{\instructor}{Spring 2022} % TA
\newcommand{\theprofessor}{Prof. Egor Starkov} % Professor

%\theoremstyle{definition} \newtheorem{ex}{\textbf{\Large{Exercise & #}\\}}
\setlength{\parskip}{0 pt}




















%%%%%%%%%%%%%%%%%%%%%%%%%%%%%%%%%%%%%%%%%%%%%%%%%%%%%%%%%%%%%%%%%%%%%%%%%%%%%%%%%%%%%%


\begin{document}

\begin{center}
	{\huge Financial Markets Microstructure: Course Plan}
\end{center}
\bigskip

This list outlines the plan for the course together with main readings for each topic. Additional readings will be assigned during lectures, whenever needed. The plan is subject to change.
\medskip

\paragraph{Part 1:} Overview
\begin{itemize}
	\item \textit{Week 1}: Introduction: types of financial markets, types of agents in financial markets
	\begin{itemize}
		\item FPR chapters 0, 1
	\end{itemize}
	\item \textit{Week 2}: Liquidity: what it is and why we care
	\begin{itemize}
		\item FPR chapters 0, 2
	\end{itemize}
\end{itemize}

\medskip 
\paragraph{Part 2:} Setting up the models
\begin{itemize}
	\item \textit{Weeks 3-4}: Dealer models 1: {Glosten-Milgrom}, fixed trade size; Bayes' Rule; adverse selection, order costs and inventory risk
	\begin{itemize}
		\item FPR chapter 3
	\end{itemize}
	\item \textit{Week 5}: Dealer models 2: {Kyle}, variable trade size; market depth under adverse selection and inventory risk
	\begin{itemize}
		\item FPR chapter 4
	\end{itemize}
	\item \textit{Week 6}: Empirics of illiquidity: estimating liquidity determinants
	\begin{itemize}
		\item FPR chapter 5
	\end{itemize}
	\item \textit{Weeks 7-8}: Limit order book; {Glosten}: static, random market order demand; market design; {Parlour}: dynamic, endogenous market order demand
	\begin{itemize}
		\item FPR chapter 6
		\item Parlour, Christine A., and Duane J. Seppi. “Limit Order Markets: A Survey.” In Handbook of Financial Intermediation and Banking, 63–96. Elsevier, 2008. \url{https://doi.org/10.1016/B978-044451558-2.50007-6}.
	\end{itemize}
	\item \textit{Week 6}: Problem set 1 out, due week 8
\end{itemize}

\medskip
\paragraph{Part 3:} Applying the models; topics
\begin{itemize}
	\item \textit{Week 9}: Fragmentation costs, {modified Kyle model}; fragmentation benefits, {modified Glosten LOB model}
	\begin{itemize}
		\item FPR chapter 7
	\end{itemize}
	\item \textit{Week 10}: Transparency: search costs, {modified GM model} for order flow transparency
	\begin{itemize}
		\item FPR chapters 8
	\end{itemize}
	\item \textit{Week 11}: Liquidity risk and Illiquidity premia
	\begin{itemize}
		\item FPR chapter 9
	\end{itemize}
	\item \textit{Week 11}: Problem set 2 out, due week 13
	\item \textit{Week 12}: Liquidity and corporate policy; Digital markets
	\begin{itemize}
		\item FPR chapter 10
		\item Kirilenko, Andrei A., and Andrew W. Lo. “Moore’s Law versus Murphy’s Law: Algorithmic Trading and Its Discontents.” The Journal of Economic Perspectives 27, no. 2 (2013): 51–72. \url{https://dx.doi.org/10.1257/jep.27.2.51}
		\item Nica, Octavian, Karolina Piotrowska, and Klaus Reiner Schenk-Hoppé. “Cryptocurrencies: Economic Benefits and Risks.” SSRN Scholarly Paper. Rochester, NY: Social Science Research Network, Oct 2017. \url{https://doi.org/10.2139/ssrn.3059856}
	\end{itemize}
	\item \textit{Week 13}: High-frequency trading
	\begin{itemize}
		\item Beason, Tyler, and Sunil Wahal. “The Anatomy of Trading Algorithms.” SSRN Scholarly Paper. Rochester, NY: Social Science Research Network, December 2, 2019. \url{https://doi.org/10.2139/ssrn.3497001}.
		\item Biais, Bruno, Thierry Foucault, and Sophie Moinas. “Equilibrium Fast Trading.” Journal of Financial Economics 116, no. 2 (May 1, 2015): 292–313. \url{https://doi.org/10.1016/j.jfineco.2015.03.004}
		\item Budish, Eric, Peter Cramton, and John Shim. “The High-Frequency Trading Arms Race: Frequent Batch Auctions as a Market Design Response.” The Quarterly Journal of Economics 130, no. 4 (2015): 1547–1621. \url{https://doi.org/10.1093/qje/qjv027}
	\end{itemize}
	\item \textit{Week 14}: Public information; optimal disclosure policies; public announcements and trade volumes
	\begin{itemize}
		\item Kondor, Péter. “The More We Know about the Fundamental, the Less We Agree on the Price.” The Review of Economic Studies 79, no. 3 (2012): 1175–1207. \url{https://doi.org/10.1093/restud/rdr051}
	\end{itemize}
	\item \textit{Week 15}: Bubbles; herding; common knowledge
	\begin{itemize}
		%\item Smith, Lones, and Peter Sørensen. “Pathological Outcomes of Observational Learning.” Econometrica 68, no. 2 (2000): 371–398. \url{https://doi.org/10.1111/1468-0262.00113}
		\item Smith, Lones, and Peter Norman Sørensen. “Observational Learning.” The New Palgrave Dictionary of Economics Online Edition, 2011, 29–52. \url{http://www.econ.ku.dk/Sorensen/papers/observational-learning.pdf}
		\item Bikhchandani, Sushil, and Sunil Sharma. “Herd Behavior in Financial Markets.” IMF Staff Papers 47, no. 3 (2000): 279–310. \url{https://doi.org/10.2307/3867650}
		\item Abreu, Dilip, and Markus K. Brunnermeier. “Bubbles and Crashes.” Econometrica 71, no. 1 (2003): 173–204. \url{https://doi.org/10.1111/1468-0262.00393}
	\end{itemize}
	%\item \textit{Week 15}: Auction models
	%\begin{itemize}
	%	\item Krishna, Vijay. Auction theory (Chapters 2 and 6). Academic press, 2009. ISBN: 978-0-12-374507-1. \url{https://doi.org/10.1016/C2009-0-22474-3} 
	%\end{itemize}
\end{itemize}

%\bigskip
%The list will be expanded and updated as we progress through the course.


%%-----------------------------------------------------------------------------------------------------

\end{document}
